\newcommand{\lat}{}%\selectlanguage{latin}{}}
\newcommand{\old}{\textsc{old}}
\newcommand{\lp}{\textsc{lp}}


\newcommand{\bra}{\selectlanguage{brazilian}{}} 
%\newcommand{\oi}{º}
%\newcommand{\ai}{ª} \newcommand{\dr}{\emph{De oratore}\ }
\newcommand{\ot}{\emph{Orator}\ }

\chapterspecial{Posfácio: Salústio e a conjuração}{}{}


\section{Salústio na história} 

O discurso brilhante e inusitado que caracteriza
as obras de Salústio, bem como a narrativa vívida e intensa por meio da qual
discorre sobre política e moral, teriam contribuído para justificar a fama
duradoura que o historiador alcançaria ao longo dos séculos\footnote{As abreviações das obras antigas seguem o padrão do \emph{Oxford Latin Dictionary} e do \emph{Greek English Lexicon}, de Liddell \& Scott. Todas as datas são a. C., salvo quando observado o contrário.}. Marcial, não sem
razão, profetizaria: \emph{Hic erit, ut perhibent doctorum corda uirorum,/
primus Romana Crispus in historia}  (“Este Crispo será o primeiro na história
romana, como afirmam os juízos dos homens doutos”)\footnote{Marcial, 14, 191.
Tradução de \versal{AMBRÓSIO} (2005: 105).}. 

Tal modo de escrita único convidou à imitação historiadores como Veleio
Patérculo, Valério Máximo, Quinto Cúrcio e, sobretudo,
Tácito\footnote{\versal{OSMOND} \& \versal{ULERY}  (2003: 188).}. Veleio
Patérculo vê em Salústio o “êmulo de Tucídides” (2, 36). Também Quintiliano o
considera um digno correspondente romano para Tucídides, como Tito Lívio o
seria para Heródoto (10, 1, 101).  Ao comparar os dois historiadores latinos,
Quintiliano, embora julgue a obra de Tito Lívio mais apropriada para a educação
dos jovens, por ser mais acessível aos iniciantes, considera Salústio um
historiador maior (2, 5, 19).  Tácito, demonstrando sua admiração por Salústio,
refere-se a ele como \lat \emph{rerum Romanarum florentissimus auctor} \bra
(\emph{Ann.} 3, 30) (“ilustríssimo autor de história romana”); em suas obras,
sobretudo nas \emph{Histórias},  os retratos de personagens, os discursos
inseridos nas narrativas e o modo como faz uso da linguagem são exemplos
inequívocos de imitação de Salústio\footnote{\versal{SYME} (1964: 293--94).}.
Escrevendo no final da Antiguidade, Agostinho  demonstra ser leitor e admirador
de Salústio, ao apontá-lo como um  “historiador de notável verdade”
(\emph{nobilitatae veritatis historicus}, \emph{C. D.} 1, 5) e homem
extremamente eloquente (\emph{vir disertissimus Sallustius} (\emph{id.} 7, 3). 

No período imperial, sobretudo na dinastia dos Antoninos (96--193 d. C), os
textos de Salústio foram frequentemente estudados, discutidos e comentados,
conforme atestam abundantes registros do período\footnote{\versal{SYME} (1964: 300).}.
Quanto à Idade Média, a grande quantidade de manuscritos supérstites revela o
lugar privilegiado que as obras de Salústio ocupavam nas escolas da
época\footnote{\versal{OSMOND} \& \versal{ULERY} (2003: 192).}. A Renascença, período em que
Salústio alcançou enorme popularidade\footnote{\versal{OSMOND} \& \versal{ULERY} (2003: 197).},
iria incluí-lo no cânone dos historiadores antigos, ao lado de César, Tito
Lívio e Tácito. Uma razão para explicar tal grau de aceitação seria a
versatilidade e a adaptabilidade de seu pensamento às variadas épocas e
interesses; outro aspecto relevante seria a particularidade de sua
linguagem\footnote{ \versal{OSMOND} \& \versal{ULERY} (2003: 186).}. 

Contudo, como ocorre em geral com os inovadores, Salústio não recebeu só
elogios, mas também críticas. Dois aspectos, na Antiguidade, eram os alvos mais
frequentes de censura: sua vida particular --- considerada incoerente com a
moral que advogava em suas obras, sobretudo nos prefácios --- e seu modo de 
escrita arcaizante.  O ataque mais duro que Salústio recebe é o de um liberto
de Pompeu Magno, de nome Leneu, que, para retaliar uma referência desabonadora
do historiador acerca de seu mestre, dirige-lhe duras ofensas, dentre elas, a
de ladrão das palavras de Catão\footnote{Cf. Suetônio, \emph{Gram.} 15: 

\emph{priscorum Catonisque verborum ineredutissimum furem}


(``ladrão extremamente grosseiro das palavras dos antigos e de Catão''). Cf. \versal{OSMOND} \&
\versal{ULERY} (2003: 186).}. Tal acusação pode ter dado origem a um epigrama que
Quintiliano refere como sendo então bem conhecido: 

\begin{verse}
\emph{et verba antiqui
multum furate Catonis,\\ Crispe, Iugurthinae conditor historiae.} 

\end{verse}
 


(“Tu, Crispo,
autor da história de Jugurta, que roubaste as palavras do antigo
Catão.)”\footnote{Quintiliano, \emph{Inst.} 8, 3, 29. \emph{Fr. Incert.}16, Morel. Courtney
(1963: 145), \emph{apud }\versal{RUSSELL} (2009: 357, n. 32).}.

O historiador Asínio Polião\footnote{Asínio Polião (76 a. C.--4 d. C.), cônsul
em 40, orador e historiador.}, segundo indica Suetônio, teria escrito um
tratado em que condenava os arcaísmos de Salústio\footnote{Suetônio, \emph{Gram.}, 10. Mas, de acordo com Aulo Gélio,  \emph{elegantia orationis
Sallustii verborumque fingendi et novandi studium cum multa prorsus invidia
fuit} (“a elegância do discurso de Salustio e seu gosto por forjar e renovar
palavras foi motivo de enorme indignação [\ldots{}]”) (4, 15,
1).}. Ainda de acordo com Suetônio, Polião teria acusado Salústio de empregar
um estudioso para colecionar palavras antigas, já em desuso, para utilizar em
suas obras\footnote{ Referência a Ateio Pretextato, \versal{SUETÔNIO}, \emph{Gram.}, 10, (cf. Woodman (1988: 149).)}.  O historiador Pompeu Trogo critica Salústio ---
e também Tito Lívio --- por intercalar discursos nas narrativas. Contudo, seu
próprio modo de escrita trairia marcas salustianas\footnote{\versal{SYME} (1964: 289).}. Tito Lívio, conforme testemunha Sêneca, o Rétor, manifesta sua
desaprovação a oradores que recorrem a \emph{verba antiqua et sordida}
(``palavras antigas e vulgares'')\footnote{ Cf. Sêneca, o Rétor, \emph{Con.}
9, 2, 26.}. Para criticar Salústio, Lívio teria chegado a depreciar a adaptação
que este fizera de uma frase grega, atribuída a Tucídides\footnote{A frase
criticada é: \emph{secundae res mire sunt vitiis obtentui} (“A prosperidade
serve maravilhosamente para encobrir os defeitos”, 1, 55, 24.
A frase, na verdade, era de Demóstenes  (\versal{XI}, 13), não de Tucídides, como supõem
Lívio e Sêneca, cf. \versal{SYME} (1964: 289, n. 62).}.  Sêneca, o Rétor, que reporta
o caso, considera que Lívio, na oportunidade, teria sido injusto com Salústio,
e acrescenta que, na adaptação de tal frase, este teria abatido Tucídides em
seu próprio acampamento, ou seja, o teria superado naquela que era sua
principal virtude, a brevidade\footnote{Sêneca, o Rétor,  \emph{Con.} 9, 1,
14.}. Mas, apesar das críticas, Tito Lívio não se teria furtado à imitação de
Salústio\footnote{\versal{SYME} (1964: 289); \versal{WOODMAN} (1988: 130; 140).}.  As
obras de Salústio continuaram sendo amplamente lidas, debatidas e mesmo
imitadas na modernidade\footnote{ \versal{OSMOND} \& \versal{ULERY} (2003: 213).}. Alguns
historiadores do século \versal{xix}, porém, baseados em teorias positivistas e
desconsiderando o contexto retórico-literário em que as obras de Salústio foram
escritas, questionariam o valor histórico de seus relatos, por não
corresponderem aos paradigmas factuais e objetivos então estimados\footnote{Cf. \versal{FUNARI} \& \versal{GARRAFONI} (2007: 65--66, e n. 1 e 2).}.  Por volta da segunda
  metade do século \versal{xx}, contudo, restabelece-se o valor de Salústio
  como historiador\footnote{Cf. \versal{LEEMAN} (1967: 57--61).}; suas obras
  continuam a ser estudadas hoje sob variados pontos de vista.


\section{Vida de Salústio}

Caio Salústio Crispo foi político e historiador do século I a. C., no período
final da República romana.  Contemporâneo de Cícero, era, porém, vinte anos
mais jovem. Teria nascido na cidade de Amiterno, na alta região da Sabina, em
86, e morrido em Roma em 35\footnote{Cf. Jerônimo,  151, H:
\emph{Sallustius Crispus scriptor historicus in Sabinis Amiterni nascitur}; p.
159 H: \emph{Sallustius diem obiit quadriennio ante Actiacum bellum}.
\emph{Apud} \versal{SYME} (1964:  13, n. 30).}. Amiterno, cidade situada
a aproximadamente 80 quilômetros de Roma, já desfrutava de pleno direito à
cidadania romana por ocasião do nascimento de Salústio\footnote{\versal{RAMSEY} (2007: 2)}. Considera-se que sua família era provavelmente constituída de
aristocratas locais (\emph{domi nobiles}), o que não queria dizer muito, pois,
naquele tempo, uma aristocracia municipal pouco contava diante da aristocracia
romana\footnote{\versal{RAMSEY} (2007: 2).}. 

Salústio, assim como Cícero, era um \emph{homo nouus}, isto é, o primeiro da
família a adentrar a carreira política. Tal passo, para um homem de origem
municipal e sem ancestrais ilustres, não se dava sem a superação de barreiras
impostas pela tradição romana, que considerava o poder e o prestígio
prerrogativas hereditárias\footnote{\versal{RAMSEY} (2007: 2).}.  Salústio, no prefácio
da \emph{Conjuração de Catilina}, afirma que, ainda muito jovem, teria se
sentido atraído pela política. Contudo, quaisquer que tenham sido suas
ocupações anteriores, o primeiro registro que temos de sua carreira política é
como tribuno da plebe, em 52,  conforme o testemunho de Ascônio Pediano,
comentador de alguns discursos de Cícero\footnote{Quinto Ascônio Pediano (9
a.C-- 76 d.C) escreveu comentários a vários discursos de Cícero, dos quais
restam-nos fragmentos de \emph{In Pisonem, Pro Scauro, Pro Cornelio, In toga
Candida e Pro Milone}. Além dos registros históricos, Ascônio consultava os
\emph{acta diurna e os acta senatus}.}. Antes disso, é razoável supor que tenha
sido questor, uma vez que esse era o primeiro cargo da carreira política
(\emph{cursus honorum}), o qual, normalmente, habilitava seu detentor à
admissão no senado. Infelizmente, a única evidência de sua suposta questura se
encontra na \emph{Invectiva contra Salústio}, obra atribuída a Cícero que hoje
se considera um texto forjado nas escolas de retórica do período
imperial\footnote{Atribuída por
Diomedes a Dídio (1.387. 4K); \emph{Apud} \versal{RAMSEY}  (2007: 2, n. 2).}.   

Durante seu tribunado, Salústio evidencia-se no cenário político em meio à
crise precipitada pelo assassinato de Públio Clódio Pulcro por Tito Ânio Milão.
Ascônio nos fornece ampla documentação desse período, no texto de comentário
que escreveu sobre o discurso de Cícero, \emph{Defesa de Milão}. Em seu relato,
Ascônio revela em várias oportunidades as atividades políticas de Salústio, não
só no senado, mas também nas assembleias populares (\emph{contiones}), o que
nos permite conhecer seu posicionamento político na época, como também suas
ligações\footnote{Ascônio refere-se às atividades de Salústio nos parágrafos
38, 45, 49, 50 e 51 da Edição de \versal{CLARK} do \emph{Pro Milone}, Oxford, 1895.
Todas as referências a Ascônio são tomadas a essa edição.}.
De acordo com Ascônio, o ano de 52 começara sem cônsules, uma vez que as
eleições não haviam podido ser realizadas em virtude dos violentos tumultos que
se alastravam pela cidade. Havia três candidatos a disputar o consulado: Tito
Ânio Milão, apoiado pela ala mais conservadora dos \emph{optimates}\footnote{\emph{Optimates}: termo raramente encontrado no latim antes de Cicero, que o
usa para designar os conservadores que apoiavam a dominância senatorial na vida
política romana; frequentemente equivalente a \emph{boni}. O termo não tem
correspondência alguma com o sentido moderno de partido político, apenas
designa uma categoria de pessoas com certas convicções e atitudes. Cf. \versal{LEWIS}
(2006: 310-11).}, e dois candidatos que tinham o apoio de Pompeu: Quinto
Metelo Cipião e Públio Pláucio Hipseu. Públio Clódio, ligado à facção dos
\emph{populares}\footnote{O termo não tem conotação partidária, apenas
refere-se comumente a um grupo, indefinido formalmente, que buscava no povo, e não na nobreza, os alicerces de seu poder.}, candidato a pretor e acerbo inimigo de Cícero e Milão,  apoiava
ativamente os candidatos de Pompeu contra este, segundo testemunha
Ascônio\footnote{Asc. 31.}.  Milão e Clódio eram líderes rivais de bandos
armados, utilizados por eles como instrumento de intimidação política.  O
primeiro, com o apoio tácito de Pompeu e dos \emph{optimates}, o segundo, de
César e dos \emph{populares}. No dia 18 de janeiro de 52, os dois líderes e
seus séquitos se encontraram na Via Ápia e, no embate que se seguiu, Clódio foi
morto pelos escravos de Milão\footnote{Asc. 33.}. Seu cadáver foi levado a
Roma e exposto no Fórum, enquanto tribunos discursavam incitando o povo contra
o assassino. Na agitação que se formou, a Cúria foi incendiada\footnote{Asc. 34.}.
No início do relato introdutório de Ascônio, dois tribunos são mencionados como
os agitadores: Quinto Pompeu Rufo e Tito Munácio Planco Bursa.  No entanto, na
parte em que comenta o parágrafo 45 da \emph{Defesa de Milão}, Ascônio menciona
Salústio ao lado de Pompeu Rufo\footnote{ Asc. 49: \emph{sunt autem contionati
eo die, ut ex Actis apparet, Sallustius et Q. Pompeius, utrique et inimici
Milonis et satis inquieti} (``Discursaram perante o povo, naquele dia, como consta das \emph{Atas} do senado, Salústio e Quinto Pompeu, ambos inimigos de Milão e bastante
agitados'').}.

A desordem continuava sem que pudesse ser contida. O senado aprova um decreto
nomeando Pompeu cônsul único em 25 de fevereiro de 52.  Pompeu então toma
medidas enérgicas para conter a crise e põe em marcha o processo de julgamento
de Milão e também dos responsáveis pelos tumultos que castigaram a cidade. Os
três tribunos, Rufo, Bursa e Salústio, passam então a utilizar as assembleias
populares para atacar violentamente Milão e também Cícero, por ter assumido sua
defesa\footnote{Asc. 38: \emph{Inter primos et Q. Pompeius et C Sallustius et
T. Munatius Plancus tribuni plebis inimicissimas contiones de Milone habebant,
invidiosas etiam de Cicerone, quod Milonem tanto studio defenderet} (``Quinto
Pompeu, Caio Salústio e Tito Munácio, tribunos da plebe, foram os principais
organizadores de assembleias inimicíssimas a  Milão, e mesmo hostis contra
Cícero, que o defendia com todo o empenho'').}. Na \emph{Defesa de Milão},
parágrafo 47, vemos Cícero queixar-se com indignação dos ataques a ele
dirigidos\footnote{\emph{Pro Milone}, 47: \emph{me videlicet latronem et
sicarium abiecti homines et perditi describebant} (``É evidente que esses
indivíduos vis e perversos me representavam como bandido e assassino.'')}. Ele
não menciona os nomes dos responsáveis, mas Ascônio o faz: Quinto Pompeu Rufo e
Caio Salústio Crispo\footnote{Asc. 50: \emph{Q. Pompeius Rufus et C. Sallustus
tribuni fuerunt quos significat. Hi enim primi de ea lege ferenda populum
hortati sunt et dixerunt a manu Milonis occisum esse Clodium, consilio vero
maioris alicuius}. (``É aos tribunos Quinto Pompeu Rufo e
Caio Salústio Crispo que ele [sc. Cícero] se refere. De fato, eles foram os primeiros a incitar  o povo a aprovar aquela proposta de lei e a dizer que Clódio havia sido morto pelo
bando de Milão, embora por instigação de alguém maior”).}.  Cícero não obteve a
absolvição de Milão, mas, no ano seguinte, conseguiu condenar Planco Bursa,
embora este recebesse apoio explícito de Pompeu Magno. O tribuno Pompeu Rufo
também foi condenado, tendo como acusador Marco Célio, amigo de Cícero. Sobre
Salústio, Ascônio nada menciona. Não há evidências de que tenha ido a
julgamento. Mas, se escapou incólume desse episódio, não logrou o mesmo em 50,
quando foi expulso do senado pelo censor Ápio Cláudio\footnote{ Cf. Dião
Cássio, 40, 63, 4.}. 

Expulsões desse gênero costumavam ocorrer por improbidade, quer na vida
privada, quer na política\footnote{\versal{SYME} (1964: 34).}. Contudo, o historiador
Dião Cássio (c. 155--229 d. C.), ao relatar o fato, não menciona o motivo,
apenas observa que, juntamente com Salústio, outros foram expulsos do senado
naquela ocasião, inclusive homens pertencentes à nobreza. Pelo modo como o
episódio é narrado, subentende-se que motivações políticas tenham guiado a
decisão do censor.

Os detratores antigos de Salústio, contudo, iriam comprazer-se em contrastar o
tom moralizante de suas obras com sua própria conduta. Uma anedota que se
tornou famosa reporta que Salústio, flagrado em adultério com a esposa de
Milão, Fausta, filha do ditador Sula, fora açoitado pelo marido ultrajado e só
teria conseguido sair com vida do incidente trocando sua liberdade por certa
soma de dinheiro\footnote{ \versal{AULO} \versal{GÉLIO},  17, 18, \emph{Apud.} \versal{SYME} (1964: 
278, n. 15).}.  Mas a credibilidade de tais registros é questionável. O tema do
adultério, bem como outras referências ao comportamento sexual do historiador
que aparecem na mencionada \emph{Invectiva contra Salústio}, são temas
recorrentes nos discursos do gênero retórico demonstrativo ou
epidítico\footnote{\versal{ERNOUT} (1996: 9).}, do qual a invectiva é uma espécie,
assim como o é seu oposto, o elogio. De resto, deve-se ainda lembrar que nem
Cícero, na \emph{Defesa de Milão}, nem Ascônio, em seus comentários, fazem
qualquer alusão ao suposto caso de Salústio com a esposa de Milão.  Depois de
sua expulsão do senado, Salústio pode ter buscado refúgio no acampamento de
César\footnote{\versal{RAMSEY} (2007: 4).}. Com efeito, pode-se encontrá-lo algum
tempo depois no comando de uma das legiões de César na Ilíria, ao final de 49,
nos princípios da guerra civil. Na ocasião, duas legiões, uma comandada por
Minúcio Básilo, outra comandada por Salústio, são enviadas para auxiliar Caio
Antônio, cujo exército estava cercado na ilha de Curita. Mas a missão fracassa.
Como os pompeianos detinham o comando do mar Adriático, a rendição foi
inevitável e César teve que amargar sérias perdas\footnote{Orósio, \versal{VI}, 15, 8,
\emph{apud} \versal{SYME} (1964: 36, n. 34).}.

O nome de Salústio reaparece no ano de 47, desta vez como enviado de César para
apaziguar uma rebelião das tropas reunidas na Campânia, em preparação para a
invasão da África. Salústio não teve sucesso na operação, mas teve sorte:
escapou com vida, sendo que dois outros enviados posteriores, senadores com o
cargo de pretores, não o conseguiram; foi necessário que o próprio César
interviesse para subjugar os amotinados\footnote{Apiano, \emph{\versal{BC}} \versal{II}, 92,
387; Dião Cássio, 42, 52, 1. Cf. \versal{SYME} (1964: 36-37).}. 

Na campanha de César na África, Salústio não teria tido outro comando em frente
de batalha. Contudo, César se serviria dele para reunir e transportar
suprimentos para as tropas, o que ele realizou, ao que parece, com
sucesso\footnote{\versal{SYME} (1964: 37).}. Depois da vitória de Tapso, na África,
em 46, César anexou a maior parte da Numídia, tornando-a província romana com o
nome de África Nova. Para governá-la, escolheu Salústio\footnote{\emph{ Id.
ibid.}.}. De acordo com Dião Cássio (43, 9, 20),  Salústio, durante seu
governo, teria saqueado a província em proveito próprio. Ao voltar a Roma, em
45, foi acusado de extorsão e só não teria sofrido outra expulsão do senado por
intervenção do próprio César, que, segundo refere Dião Cássio, ficou com a
suspeita de ter protegido Salústio em troca da divisão do lucro (43, 47, 4). O
escândalo, de qualquer modo, mais uma vez deve ter afetado a reputação de
Salústio.

Se ele decidiu afastar-se da vida pública nesse momento, ou depois do
assassinato de César, em março de 44, como parece mais provável, não se pode
saber ao certo. O fato é que Salústio decide renunciar à política, atribuindo o
motivo a várias adversidades sofridas, e declara sua intenção de dedicar-se a
escrever história, conforme testemunha o prefácio da \emph{Conjuração de
Catilina} (\versal{IV}, 1--3). A vila que ele ou seu herdeiro por adoção construíra,
cercada pelos seus famosos jardins (os \emph{horti Sallustiani}), situada entre
os montes Píncio e Quirinal, que mais tarde seria propriedade da família
imperial, era uma indicação de que seus recursos eram capazes de garantir seu
ócio\footnote{ \versal{RAMSEY} (2007: 5).}. 


\section{A obra de Salústio}

De Salústio chegaram até nós duas monografias completas, a \emph{Conjuração de
Catilina}, que ora se apresenta, e a \emph{Guerra de Jugurta}, bem como uma
obra de estrutura analística pertencente ao subgênero da história universal que
cobria os anos de 78 a 67, as \emph{Histórias}, que o autor, com sua
morte, deixou incompleta, chegando ela até nós de maneira apenas fragmentária.
A \emph{Conjuração} foi publicada depois da morte de César (44), como se
deduz do capítulo \versal{LIII}, em que Salústio usa o verbo “ser” no passado para se
referir a este e a Catão, e é tradicionalmente considerada anterior à
\emph{Guerra de Jugurta}, embora o único dado de que podemos ter alguma certeza
seja que as monografias precederam as \emph{Histórias}. 

Outras obras a ele atribuídas são duas \emph{Epístolas a César} e uma
\emph{Invectiva contra Cícero}, sobre as quais pesa a dúvida da autenticidade.
Desde a Renascença se tem debatido sobre a legitimidade de tais obras, embora,
até hoje, não se tenha conseguido provar definitivamente seu caráter autêntico
ou espúrio. Os argumentos em favor da legitimidade, porém, têm pouco apoio
entre os estudiosos. Como afirmam \versal{OSMOND} e \versal{ULERY} (2003: 187), estas duas obras,
caso autênticas, forçosamente precederiam as monografias e as \emph{Histórias}. 

\section{Salústio e a historiografia romana}

 Ponderando acerca da escrita da história entre gregos e romanos, Salústio
 deplora que em Roma nunca tivesse havido homens de engenho que se dedicassem a
 eternizar os feitos de seus cidadãos ilustres. Observa que, diferentemente dos
 gregos, os romanos mais sagazes preferiam notabilizar-se por suas próprias
 realizações a escrever sobre as realizações alheias (\emph{Cat.} \versal{VIII}, 5). Assim,
 Salústio reconhece que Roma não produzira até então um escritor de história à
 altura dos historiadores gregos. Tal julgamento coincide com a opinião de
 Cícero em \emph{Leg.}, I, 5\footnote{Também em \emph{Brut.}, 228, e
 \emph{de Orat.} 2, 52--55.}.  Nesse diálogo, escrito entre
 50--46, em que as personagens em cena são o próprio Cícero, Quinto, seu irmão,
 e o amigo Ático, há uma passagem em que este tenta convencer Cícero a
 dedicar-se ao gênero histórico, uma vez que suas qualidades como orador o
 credenciariam a tal tarefa.  Ático então argumenta que a história, do modo
 como ele próprio e Cícero a concebiam, ainda estava ausente das letras romanas
 (“[\ldots{}] a história está ausente de nossas letras, segundo eu próprio entendo e
 inúmeras vezes ouvi de ti [\ldots{}]”;  \emph{Leg.}. I, 5). 

Tal afirmação, contudo, não queria dizer que os romanos não tinham interesse
por sua própria história. Desde provavelmente o século \versal{IV} ou \versal{V}, o  
pontífice máximo consignava os eventos importantes da cidade em um quadro branco
(\emph{tabula dealbata}), que era afixado no átrio de sua casa. Esses quadros
foram afixados durante séculos, até 130,  quando o  pontífice máximo Múcio
Cévola mandou publicar os documentos anteriores em oitenta volumes, que
passaram a constituir os \emph{Anais Máximos} (\emph{Annales
Maximi})\footnote{\versal{FRIER} (2002: xv; 163; 171).}. 

Efetivamente, havia em Roma condições favoráveis para o desenvolvimento da
historiografia: a tradição firmemente estabelecida do registro oficial dos
eventos importantes; o interesse pessoal dos homens ilustres na divulgação
escrita dos feitos de seus ancestrais, requisitos fundamentais para propugnar
suas próprias carreiras políticas; e, ao final do século \versal{III}, o estímulo
e a influência dos modelos helenísticos\footnote{\versal{USHER} (1969: 130).}.    A
primeira forma de historiografia em prosa parece ter surgido durante a Segunda
Guerra Púnica (218--201) com a escrita de tipo analístico, modo de narrar
característico dos anais, que apenas forneciam um relato cronológico dos
eventos, sem qualquer tipo de análise. Fábio Píctor (final do século \versal{III}) 
é tradicionalmente considerado o primeiro romano a dedicar-se à escrita da história,
embora o tivesse feito em grego, como alguns de seus sucessores. Membro de uma
notável família patrícia, Píctor teria sido também o primeiro de uma longa
lista de senadores que devotaram seu tempo livre (\emph{otium}) a compor a
própria versão de eventos que apresentavam algum tipo de
interesse\footnote{\versal{USHER} (1969: 131).}.

No diálogo \emph{Do Orador} (55), obra retórica em que se discute acerca
do orador ideal, Cícero relaciona Píctor, Catão e Pisão como escritores que
compuseram história à maneira dos anais, pois não dominavam ainda “os meios com
que se orna o discurso” (2, 52--53). Apenas Célio Antípatro\footnote{Célio
Antípatro, que escreveu a história da segunda Guerra Púnica em sete livros,
parece ter sido o primeiro romano a ter abandonado a escrita analística em
favor da monografia histórica, para tratar de um tema único.  Cf. \versal{RAMSEY}
(2007: 8, n. 10).}, afirma Cícero, teria se elevado um pouco acima destes,
tendo acrescentado “maior grandiloquência à história”, mas, mesmo assim,
careceria de polimento de elocução e de variedade (2, 54).  A combinação de
serviços à República com historiografia tornou-se uma tradição fortemente
estabelecida em Roma\footnote{\versal{RAMSEY}
(2007: 132).}. Tal foi o caso, além de
Fábio Píctor, de Marco Pórcio Catão (234--149). Cônsul em 195, censor em 184,
Catão é o homem que, entre as gerações posteriores, é tido como a
personificação de toda espécie de excelência particularmente romana\footnote{\versal{RAMSEY}
(2007: 132).}. Ao retirar-se da vida pública, passou a escrever história.
Sua obra, \emph{Origens}, em sete livros, talvez a primeira do gênero em latim,
foi composta entre 168 e 149. Cobre o mesmo período dos analistas --- que
geralmente remontavam seu relato à época de fundação da Cidade ---, embora
introduza uma nova maneira de escrever história: não só Roma é alvo de sua
atenção, mas também as comunidades que lhe são vizinhas; não se limita a
reproduzir analisticamente os eventos, mas discute questões de ordem
geográfica, etnográfica e ainda econômica\footnote{\versal{CHASSIGNET} (1999: 10).};
e, apesar do notório patriotismo de seu autor, a obra não deixa de mostrar
sinais de influência grega, como, por exemplo, de Xenofonte\footnote{ \versal{USHER}
(1969: 135).}.

Em Roma, no século \versal{I}, embora a oratória tivesse atingido elevado grau de
excelência com Cícero, que se notabilizou não só por praticá-la como por
teorizá-la, a história ainda não havia tido a mesma sorte. Em contexto romano,
as primeiras reflexões teóricas sobre a escrita da história a chegarem até nós
encontram-se no próprio Cícero, que, embora não tenha praticado o gênero,
elaborou uma concepção de história, a de história ornada, que teve larga
fortuna entre os escritores posteriores. Tal conceito envolve, segundo suas
próprias palavras, “a variedade de tópicos”; “o arranjo das palavras”; “a
cadência regular e uniforme do discurso” (\emph{de Orat.}, 2, 54). Segundo
Cícero, a escrita da história devia ser sobretudo obra de oradores (\emph{opus
oratorium maxime}, \emph{Leg.} I, 5), ou seja, de quem detivesse o domínio
da arte retórica e estivesse, portanto, credenciado para escrever bem.  No \emph{Do
orador}, Cícero apresenta, em certo momento, a personagem Antônio refletindo
sobre a causa de os romanos não possuírem até então (91, data dramática
do diálogo) um representante no gênero historiográfico à altura dos gregos.
Antônio conclui que, ao contrário destes, os romanos eloquentes preferiam
brilhar defendendo causas no fórum (2, 55), enquanto, na Grécia, historiadores
ilustres, como Heródoto, Tucídides, Xenofonte e outros haviam sido  eloquentes
sem nunca ter praticado o gênero judiciário.  A razão oferecida por Antônio é
próxima à que Salústio oferece na passagem que citamos acima (\emph{Cat.} \versal{VIII},
5);  tal argumento reflete o ideal de praticidade romano, segundo o qual
haveria mais virtude na ação do que no trabalho intelectual, reservado para as
horas de ócio.  

 A crítica de Cícero e de Salústio de que Roma não tinha ainda um escritor que
 pudesse ser comparado aos historiadores gregos repousa principalmente no fator
 da elocução. A elocução da história é uma, a do gênero judiciário é outra bem
 distinta, afirma Cícero por meio de Antônio. Aquela deve ``buscar uma conformação das palavras e um gênero de discurso amplo e contínuo, que flua uniformemente e com alguma placidez, sem a aspereza própria dos tribunais e sem os aguilhões forenses dos pensamentos  [\ldots]” (\emph{de Orat.}, 2, 64). 
 
O que Cícero preconiza para a história é uma elocução capaz não só de ensinar,
mas também de deleitar o leitor. Em 56, ele envia uma carta ao amigo Luceio,
historiador seu contemporâneo\footnote{ \emph{Fam.}, 5, 12.}, pedindo-lhe que
narre o evento que, a seu ver, era o mais notável de seu consulado: sua atuação
ao suprimir a conspiração de Catilina, bem como os eventos que se seguiram até
seu retorno do exílio, em 57. Desejando ter seus feitos celebrados ainda em
vida, Cícero manifesta sua preferência de que isto ocorra por meio de um gênero
de escrita que hoje chamamos monografia, que lhe propiciaria maior destaque,
por tratar-se de obra que gira em torno de tema único, centrado em um
protagonista principal. Essa carta nos é valiosa do ponto de vista das
concepções historiográficas nela expostas. Permite-nos entrever mais claramente
a concepção de Cícero de história ornada e, além disso, que tipo de escritor
ele considerava habilitado para escrever história. 

No parágrafo 7 da carta, Cícero justifica seu desejo de ver seus feitos
narrados particularmente por Luceio, que seria dotado da “autoridade de um homem
muitíssimo ilustre e distinto, reconhecido e mais que todos aprovado nas causas
mais importantes e sérias da República” (\emph{Fam.}, 5, 12, 7). Cícero
considera que, para a escrita de história, a política é fonte de
\emph{auctoritas} (autoridade).  Nesse contexto, tal \emph{auctoritas}
significa a capacidade analítica de apreensão do fato, bem como os conhecimentos
adquiridos que permitam ao escritor traduzir em palavras a experiência
vivida\footnote{ \versal{SEBASTIANI} (2007: 79).}. Ter vivenciado o período que narra
era considerado fundamental já na historiografia grega clássica. Assim, Luceio
é o homem ideal para escrever sobre Cícero porque, além de dotado de
autoridade, vivenciou o período sobre o qual devia escrever\footnote{\versal{SEBASTIANI} (2007: 77-78).}.



Não sabemos se Luceio chegou a realizar o desejo de Cícero, mas, caso o tenha
feito, tal escrito não chegou até nós. Contudo, ironicamente, 10 anos mais
tarde, um homem dotado de experiência nos assuntos políticos e militares, bem
como de habilidade retórica, decide empreender a tarefa de narrar o mesmo evento antes
sugerido por Cícero a Luceio, e no mesmo formato: a monografia. Trata-se de
Salústio, que leva a cabo a tarefa, porém, de modo próprio: em primeiro lugar,
centra sua narrativa na \emph{persona} de Catilina, cabendo a Cícero um papel
incidental; em segundo, afasta-se diametralmente da elocução proposta por
Cícero para a historiografia, escolhendo uma escrita breve e pontilhada de
reminiscências da antiga prosa dos historiadores antigos.

 
\section{Os prefácios}


Os prefácios dos escritos históricos eram, tradicionalmente, a parte em que os
historiadores expunham o programa que pretendiam seguir em suas obras. Assim, a
começar por Heródoto, os prefácios desenvolveram, com o passar dos séculos,
lugares-comuns que funcionavam como indicadores do projeto historiográfico de
cada autor\footnote{Leia-se, particularmente, o artigo de \versal{EARL} (1972) para uma
síntese da questão.} --- ou, pelo menos, daquilo que o autor nos queria fazer
crer como tal. Os principais lugares-comuns diziam respeito: à enunciação do
tema e do subgênero adotados; à justificativa da escrita da história e da
escolha do tema; ao desejo de alcançar a glória; e à declaração de
imparcialidade por parte do historiador e de sua consequente busca da verdade.  

No prefácio da \emph{Conjuração de Catilina}, Salústio procura justificar sua
retirada da vida pública e dedicação à escrita do gênero histórico. Por
escrever uma obra de caráter moralizante, encontrava-se diante de um dilema: o
historiador, como se disse, fora expulso do senado em 50, possivelmente por
improbidade, e enfrentara um processo de extorsão após sua administração da
província da África Nova, em 45. Sendo assim, tinha, de antemão, de enfrentar o
problema da falta de \emph{auctoritas} ao compor seu \emph{ethos} (“caráter”)
de historiador moralista, com consequente falta de \emph{fides}
(“credibilidade”). A solução encontrada foi extremamente engenhosa: imitando a
Carta 7 de Platão\footnote{ O conceito antigo de imitação nada tem que ver com
nossas ideias de originalidade, que remetem, em última instância, a uma
concepção romântica de escrita. Para os antigos, a imitação de um antecessor
era uma estratégia de escrita perfeitamente legítima, contanto, bem entendido,
que apresentasse um elemento de novidade em relação ao texto imitado (leia-se, por exemplo, Quint. \emph{Inst.} 10,~2).}, em que o filósofo explica seu entusiasmo
inicial pela vida pública e posterior desilusão, a que se seguiu a dedicação à
filosofia, Salústio coloca-se na posição de Platão romano da historiografia,
apresentando-se como vítima da corrupção vigente na época de sua juventude,
quando se lançara, quer nos fazer crer, com ardor à vida pública e à política.

%, e como oferecendo maior préstimo à república em seu ócio do que o fazia
%quando nos cargos públicos. Excluir

O prefácio da \emph{Conjuração de Catilina} apresenta todos os elementos que
constituem os tópicos protocolares do gênero. Contudo, é absolutamente singular
na maneira como está arranjado: Salústio não aborda de imediato esses tópicos
tradicionais, mas começa seu exórdio por meio de reflexões de caráter
filosófico sobre o \emph{corpus} (“corpo”) e o \emph{animus} (“espírito”)
humanos, que remetem, em última instância, a Platão. 

No primeiro parágrafo, Salústio começa pela comparação entre seres humanos e
animais, e pela enunciação do que é preciso fazer para que aqueles superem a
estes:

\begin{quote} 1. A todos os homens que anseiam superar os restantes animais,
  cumpre empenhar-se com o máximo afinco para que não atravessem a vida no
  silêncio, tal como o gado, que a natureza moldou curvado e obediente ao
  ventre. 2. Ora, nosso poder, como um todo, reside na mente e no corpo: da
  mente, exercemos antes o mando, do corpo, a escravidão; compartilhamos uma
  com os deuses, o outro, com as feras. 3. Daí parecer-me mais correto buscar a
  glória pelos recursos da inteligência do que pelos da força e, uma vez que a
  própria vida que gozamos é breve, tornar o mais duradoura possível a
  recordação de nós mesmos. 4. De fato, a glória do dinheiro e da beleza é
  fugaz e frágil, a virtude é uma posse brilhante e eterna.

\end{quote} 
 
A tópica da comparação do homem com os animais já comparece no
\emph{Panegírico} de Isócrates, que observa que, se nos diferenciamos dos
animais pela fala, pela eloquência superamos as pessoas naquilo mesmo em que
superamos os animais\footnote{Isócrates, 4, 48.}. Entre os
latinos, a ideia fora retomada por Cícero em seu \emph{Da invenção}, também no
contexto da eloquência. No §\versal{I} do prefácio, porém, Salústio efetua uma imitação
com variação, introduzindo, em lugar desta, a historiografia, ainda que de
maneira oblíqua: apenas mais adiante, no §\versal{III}, ficará claro que a referência é à
historiografia e à sua prática por parte de Salústio. 

Os homens que pretendem superar os outros animais, então, devem, para Salústio,
empregar seus esforços para não atravessar a vida \emph{silentio}. Esta
palavra, no original latino, pode ter dois sentidos, ativo e
passivo\footnote{Cf., acerca do termo e da passagem, o breve e excelente artigo
de \versal{WOODMAN} (1973).}. Pelo primeiro, poderíamos parafrasear a palavra por “sem
que se fale a seu respeito”, ou, como mais concisamente propõem comentadores e
tradutores, “na obscuridade”. Pelo segundo, teríamos a idéia de “em silêncio”,
“sem falar”. Ora, como aponta Woodman, a concisão salustiana parece estar em ação na passagem, uma vez que as duas leituras possíveis não são
alternativas que tenhamos de escolher, mas complementam-se uma à outra: de um
lado, o termo diz respeito à fama que os homens devem lutar por adquirir, de
outro, à maneira pela qual Salústio decidiu granjear tal fama: por meio de uma
atividade que diz respeito não ao corpo, mas ao espírito --- a escrita da
história.

Ainda no primeiro parágrafo, Salústio introduz a bipartição corpo/espírito,
ligando, ao primeiro, o mando e a atribuição ao domínio divino, ao segundo, a
escravidão e a atribuição ao domínio animal. A partir disso, enunciará o
objetivo que concerne, em geral, a todo homem que deseje superar os restantes
animais, em particular, a ele, Salústio, na qualidade de historiador: a busca
da glória e da recordação da posteridade. Salústio apropria-se dessa tópica
tradicional
 e a insere num contexto de reflexão filosófica acerca da condição humana, onde
 reside a novidade de seu tratamento. Para se colocar no domínio do espírito,
 do comando e dos deuses, o historiador aponta como mais correto o uso da
 inteligência ou engenho (\emph{ingenium}) do que o da força (\emph{vires}), e
 é por isso que se dedicará, não à busca da glória proveniente das riquezas e
 da beleza corporal, que é efêmera, mas à da virtude, brilhante, eterna e
 imortal.

O segundo parágrafo introduz um elemento que só será explicitado na primeira
digressão da obra, concernente ao passado glorioso de Roma e a sua decadência a
partir da destruição de Cartago (no ano de 146): a ideia de que as grandes conquistas
dizem respeito não ao domínio da força, mas ao da inteligência, e de que um
domínio se mantém apenas quando atuam as mesmas qualidades que estavam em ação
quando de sua conquista. Quando estas mudam, mudam com ela a Fortuna e os
costumes, e o poder passa, então, a alguém melhor. 
É com essa alusão à decadência dos costumes, com a substituição do
trabalho, da temperança e da equidade pela indolência, pelo desejo e pela
soberba, que Salústio passa a separar os que se dedicam aos prazeres do corpo
daqueles que buscam a glória pelos feitos da inteligência: ``Parece-me viver e
desfrutar da vida apenas quem, ocupado com alguma atividade, busca a glória de
feito ilustre ou de bom dote'' (\versal{II}, 9). Neste passo é desenvolvido e explicitado
o sentido passivo do termo \emph{silentio}: igualam-se aos animais aqueles que
se entregam aos prazeres do corpo, apresentados concretamente em “entregues ao
ventre e ao sono” (\versal{II}, 8) --- lembremos da imagem inicial do §\versal{I}, que caracteriza
o gado como curvado e obediente ao ventre ---, e, por não buscarem a glória,
não terão fama ou renome e acerca deles a posteridade calará (\versal{II}, 8).  Com essa
discussão, Salústio pode então retomar, no §\versal{III}, o tema da escrita da história e
de sua justificativa ao escrevê-la. Argumenta que é digno de louvor não só
aquele que realiza boas ações pela República, mas também aqueles que narram
tais feitos. As duas ocupações podem tornar um homem ilustre, mas a tarefa do
escritor parece-lhe “sobremaneira árdua [\ldots{}]: % Aqui, Jorge, as reticências não estão começando a linha
primeiro, porque se devem
igualar feitos e ditos; em seguida, porque a maioria considera os delitos que
se censuram fruto de malevolência e inveja” (\versal{III}, 2). 


Para o público contemporâneo de Salústio, uma comparação entre aquele que
realiza e aquele que escreve seria, muito provavelmente, favorável ao primeiro,
e o historiador tenta, neste passo, equilibrar os dois lados ou mesmo inverter
essa relação de valores. Para tal, faz referência às dificuldades inerentes à
escrita da história, que dizem respeito tanto a sua produção como a sua
recepção: de um lado, é preciso que se igualem as palavras aos feitos; de
outro, a obra pode provocar no leitor a suspeita de parcialidade e
fabricação\footnote{Cf. \versal{GRETHLEIN} (2006: 136).}.


Nos parágrafos \versal{III-IV}, ocorre a imitação da Carta 7 de Platão. Comparemos as duas
passagens: \begin{quote}

1)	(\emph{Cat.} \versal{III}, 3--4, 2) Mas eu, ainda bem jovem, de início, tal como a
maioria, lancei-me com ardor à vida pública, e enfrentaram-me aí muitas
adversidades. Pois imperavam, em lugar do pudor, da integridade, da virtude, a
audácia, a largueza, a avidez.  Embora meu ânimo, desafeito das más condutas,
as desprezasse, em meio a tamanhos vícios minha idade vacilante mantinha-se
corrompida pela ambição; e a mim, divergindo embora dos maus costumes dos
demais, atormentava-me, pela fama e pela inveja, o mesmo desejo de honra que
aos restantes. Então, quando meu ânimo teve descanso das muitas misérias e
perigos e decidi manter o restante de minha vida longe da vida pública, não
quis exaurir meu bom ócio na apatia e na indolência, ou mesmo passar a vida
voltado ao cultivo do campo ou à caça, tarefas de escravos; mas, tornando
àquele mesmo projeto e estudo de que me apartara a má ambição, decidi minuciar
os feitos do povo romano por partes, conforme cada um parecesse digno de
recordação [\ldots{}].  \end{quote}

\begin{quote} 2) Platão (\emph{Ep.} 7, 325d--326a) [\ldots{}] a legislação e a moralidade
  estavam corrompidas a tal ponto, que eu, inicialmente pleno de ardor para
  trabalhar a favor do bem público, considerando esta situação e vendo como
  tudo caminhava à deriva, acabei por ficar confuso. Não deixei, entretanto, de
  procurar nos acontecimentos e especialmente no regime político os possíveis
  indícios de melhoras, mas esperei sempre o bom momento para agir. Acabei por
  compreender que todos os Estados atuais são mal governados, pois a sua
  legislação é praticamente incurável sem enérgicos preparativos coincidindo
  com felizes circunstâncias. Fui então irresistivelmente conduzido a louvar a
  verdadeira filosofia e a proclamar que somente à sua luz se pode reconhecer
  onde está a justiça na vida pública e privada\footnote{Tradução de \versal{DA SILVA}
  e \versal{MELO} (2002: 50).}.  \end{quote}
%manter as citações, por causa da importância da comparação

Parece-nos que, na passagem em questão, Salústio está a se colocar, conforme
dissemos, como o Platão romano da historiografia: assim como este estaria
plenamente justificado em abandonar a vida pública e dedicar-se à filosofia, o
mesmo se daria com aquele em relação à história. O cotejo entre os dois
contextos, no entanto, mostra que há uma imperfeição nessa relação de
proporção: trata-se justamente do fato de Salústio se haver deixado tomar pela
corrupção reinante\footnote{Embora pretenda nos fazer crer que isso foi motivado
pela inocência da idade, Salústio contava cerca de 36 anos quando foi
expulso do senado e 41 quando sofreu o processo por extorsão.}, elemento que,
como se nota, não se encontra no relato de Platão.



Somente após suas reflexões sobre o corpo  e o espírito e sobre a escrita da
história é que Salústio faz uso dos tópicos protocolares dos prefácios do
gênero historiográfico --- enunciar o subgênero da obra, o seu tema e a
justificativa deste, a intenção de imparcialidade do autor, a busca da
fidedignidade (\versal{IV}, 3--5).

Ao dizer que minuciará os feitos do povo romano (a expressão em latim,
\emph{res populi Romani}, também pode ser traduzida por “a história do povo
romano”), Salústio insere-se no gênero histórico; ao qualificar essa atividade
com a expressão “por partes” (\emph{carptim}), promove a escolha do subgênero
da monografia histórica, que não tinha uma denominação própria na Antiguidade.
Repare-se que uma nova qualificação, nessa mesma linha, é feita longo em
seguida, quando diz “concisamente”, termo que parece carregado de dupla
significação: de um lado, remete, tal como \emph{carptim}, ao subgênero da
monografia; de outro, à maneira como Salústio a escreverá, concisa e breve,
como demonstraremos adiante. Por outro lado, quando acresce “conforme cada um
parecesse digno de recordação”, está a atualizar um tópico dos prefácios
historiográficos assim preceituado por Luciano:

%talvez colocar o que os manuais dizem, e na forma de paráfrase.

\begin{quote} Sempre que [\emph{sc.} o historiador] fizer uso de um prefácio,
  começará por apenas dois elementos, não três, como fazem os oradores:
  deixando de lado a captação da benevolência, fornecerá a seus ouvintes o que
  é digno de atenção e aprendizado\footnote{ \versal{LUCIANO}, \emph{Hist. Conscr.} 53.}. 

\end{quote}

O mesmo pode-se dizer da afirmação de que a conjuração é “sobremaneira
memorável pela novidade do crime e do perigo”, conforme a sequência do preceito
de Luciano mostra:

\begin{quote}

Estes [\emph{sc.} os ouvintes] atentarão a ele se der mostras de que irá tratar
de temas grandiosos, necessários, familiares ou úteis\footnote{\emph{Idem,
ibidem}.}.

\end{quote}

 
Por “novidade do crime”, pode-se supor que Salústio esteja se referindo a dois
aspectos inusitados da conjuração: o fato de cidadãos da nobreza estarem
lutando contra a República, e esta contra aqueles\footnote{\versal{CHIAPPETTA} (1996:
28, n. 32).}; e o fato de Catilina ter se colocado à frente de um exército
particular, uma vez que não estava investido de nenhuma magistratura
pública\footnote{ \versal{MACKAY} (1962: 184).}.

\section{Os retratos}

A biografia, que, entre os antigos, recebia o nome de “vida” (\emph{bíos} entre
os gregos, \emph{uita} entre os latinos), era um dos subgêneros da
historiografia antiga em que o louvor e o vitupério dos biografados tinham
prioridade sobre a imparcialidade e a verdade, conforme afirma explicitamente
Políbio, historiador grego do período helenístico\footnote{ Cf. Políbio, 10, 21. \emph{Apud} \versal{AMBRÓSIO} (2005: 76).}. 
Esse subgênero é empregado em miniatura, de maneira ancilar, por assim dizer, nos retratos das
figuras históricas, que constituem uma das partes tradicionais da narrativa
histórica antiga.

Na \emph{Conjuração de Catilina}, tais retratos, em que as personagens são
apresentadas à luz da dicotomia \emph{corpus/ animus} introduzida no prefácio,
mostram-se perfeitamente adequados à ideia de Salústio de colocar a
\emph{virtus} [“virtude”] como o fulcro de toda a sua obra. Na formulação de
Kraus e Woodman:

\begin {quote}

A principal preocupação de Salústio, ao longo de suas obras, é com a
\emph{uirtus}, ou seja, “o funcionamento do \emph{ingenium} [o talento inato de
uma pessoa e suas habilidades intelectuais] para realizar \emph{egregia
facinora} [“feitos egrégios”], obtendo, assim, \emph{gloria}, pelo exercício das
\emph{bonae artes} --- incluindo-se energia (\emph{industria}), trabalho árduo
(\emph{labor}), integridade (\emph{fides}), modéstia (\emph{pudor}) e
autocontrole (\emph{continentia}). Sua concepção de \emph{uirtus }é aparentada
à tradicional noção aristocrática romana --- com a exceção de que a de Salústio
é uma qualidade produzida pela própria pessoa, não dependendo de seu
nascimento, atividade escolhida ou classe, nem sendo garantida por eles. Ele
explora tal preocupação de diversas maneiras: por meio da discussão filosófica
nos prefácios; da apresentação da \emph{uirtus} corrompida ou pervertida (como
no caso de Catilina e Jugurta) ou da \emph{uirtus} misturada a elementos
baixos, por vezes desviando completamente um homem de seu curso adequado (como
no caso de Mário, Sula e Pompeu); e de uma comparação entre duas grandes
figuras, cada uma das quais encarnando um aspecto da \emph{uirtus} (César e Catão)\footnote{\versal{KRAUS} \&\versal{WOODMAN} (1997: 11).}.

\end{quote}

Há três retratos na \emph{Conjuração de Catilina}: o de seu próprio
protagonista, imediatamente após o prefácio (§\versal{V}), o de Semprônia, personagem
que participa da trama de maneira apenas indireta (\versal{XXV}), e o de César em
contraposição ao de Catão, um duplo retrato, comparativo, que recebe o nome
técnico de síncrise (\versal{LIV}). Catilina e Semprônia são exemplos de personagens
viciosas (o primeiro era “dotado de índole má e depravada”, a segunda colocava
tudo acima da honra e do pudor); César e Catão, de personagens virtuosas,
embora cada um de uma maneira diferente (eram homens “de grande virtude, de
caráter diverso”), como Salústio aponta em \versal{LIII} e a síncrise procurará
demonstrar verossimilmente.

\section{As digressões}

A digressão é um dos mecanismos da historiografia antiga usados para evitar a
monotonia da narrativa, conferindo-lhe variedade e propiciando deleite a sua
leitura. Apesar de sua denominação, cumpre notar que a digressão pretende-se, e
efetivamente o é, intrínseca ao assunto tratado, não podendo ser retirada da
obra sem prejuízo de sua unidade. Por sinal, o próprio Salústio sinaliza que
assim seja no parágrafo final do capítulo \versal{V}, que antecede e prenuncia a
digressão: \begin {quote} Já que a ocasião nos lembrou os costumes da cidade, o
próprio tema parece exigir que remontemos ao passado [\ldots{}].  \end{quote} Na
\emph{Conjuração de Catilina}, a principal digressão, que trata do passado
grandioso de Roma e da decadência dos costumes, encontra-se nos capítulos \versal{VI--XIII};
a segunda digressão, em \versal{XVIII-XIX}, trata da suposta primeira conjuração de
Catilina; por fim, de \versal{XXXVI}, 4 a \versal{XXXIX}, 5, Salústio discorre, na terceira digressão,
sobre os fatores que tornaram a conjuração possível.  \section{Os discursos}
 
Os discursos das personagens históricas presentes na Conjuração de Catilina
podem parecer, a nossos ouvidos modernos, fora de lugar numa obra de caráter
histórico: acostumados, desde o começo do século \versal{XX}, a ver e ouvir, em
gravações ou ao vivo, as declarações mesmas dos agentes da história, e afeitos
a um conceito de verdade que sem dúvida era estranho aos antigos, poderíamos
desqualificar a própria presença dos discursos na monografia como mera ficção.
Contudo, cabe observar que a escrita da história, entre os antigos, inseria-se
no domínio das letras, não no das ciências, e a criação do verossímil, que
consideraríamos vício, por aproximá-lo do falso, seria por eles julgada
virtude, como marca de um escritor engenhoso.  É exatamente por isso que
Salústio, quando reconstrói o que suas personagens teriam verossimilmente dito
em determinada situação, marca tal verossimilhança antes de começar cada
discurso com o emprego do termo \emph{huiuscemodi}, que traduzimos, ao longo de
todo o livro, por “do seguinte teor”. Observemos alguns exemplos: No capítulo
20, Salústio apresenta o discurso de Catilina aos conjurados em sua casa,
introduzindo-o pela seguinte observação: \begin{quote} Catilina, quando vê
  reunidos os que há pouco mencionei (\emph{sc.} os conjurados), embora
  houvesse não raro tratado de diversos pormenores com cada um deles, julgando
  que seria útil falar e exortar a todos juntamente, retira-se para uma parte
  oculta de sua casa e ali, distante de qualquer testemunha, proferiu um
  discurso do seguinte teor (\emph{orationem huiuscemodi habuit}).  \end{quote}
  Ora, mais do que nunca é necessário ao historiador observar de antemão que as
  palavras que se seguirão são uma reconstrução verossímil do que deve ter
  acontecido, uma vez que o leitor facilmente perceberia a impossibilidade de
  Salústio saber exatamente o que fora dito naquela circunstância, dado o
  caráter secreto do encontro e a ausência de qualquer testemunha que tivesse
  participado da conjuração --- trata-se de uma maneira de conferir
  credibilidade ao discurso e ao próprio historiador. Idêntica é a razão para o
  uso da mesma expressão, com pequena variação na ordem dos termos
  (\emph{huiuscemodi orationem habuit}), em \versal{LVII}, que precede o discurso
  derradeiro de Catilina ao exército de conjurados, antes da batalha final.

No caso dos discursos proferidos por César (capítulo \versal{LI}) e Catão (capítulo \versal{LII})
durante a deliberação dos senadores para decidir o destino dos conjurados que
haviam sido capturados, a mesma estratégia e a mesma expressão, ou quase
idêntica, são empregadas: \begin{quote} Ora, César, quando chegou sua vez e o
  cônsul solicitou seu parecer, falou palavras do seguinte teor
  (\emph{huiuscemodi verba locutus est});

\end{quote} e

\begin{quote} Depois que César terminou seu discurso, os demais concordavam
  verbalmente com um ou outro, de maneira diversa. Porém, Marco Catão,
  consultado acerca de sua proposta, proferiu um discurso do seguinte teor
  (\emph{huiuscemodi orationem habuit}).  \end{quote} Nestes dois casos, porém,
  podemos postular uma explicação um pouco diversa daquela que propusemos para
  os discursos de Catilina: como o próprio Salústio informa, o senado em peso
  estava presente à mencionada deliberação, e, pode-se supor, parte do público
  leitor da \emph{Conjuração de Catilina}, apesar de passados cerca de 20 anos
  do evento, teria ainda alguma lembrança do que fora dito na ocasião. Salústio
  necessitaria, então, apontar a tais leitores que o que está prestes a referir
  não é exatamente o que foi dito na ocasião, mas uma reconstrução verossímil
  do mesmo, baseado no caráter de César e Catão, cuja síncrise apresentará logo
  após o discurso deste.  


\section{Os “documentos''}

 Também no caso daquilo que denominaríamos  “documentos”, como as cartas
 citadas na \emph{Conjuração}, Salústio faz uso da mesma estratégia empregada
 para os discursos, à exceção de dois casos em que pretende estar a transcrever
 sua cópia exata (\versal{XXXIV}, 3; \versal{XL}, 4). Assim, no final do capítulo 32, faz menção à
 carta enviada por Mânlio, um dos cabeças da conjuração, com as seguintes
 palavras: \begin{quote} Enquanto isso ocorre em Roma, Caio Mânlio manda a
 Márcio Rex enviados de seu contingente com ordens do seguinte teor (\emph{cum
 mandatis huiuscemodi}).  \end{quote} Já quando se trata das cartas do próprio
 Catilina aos senadores, Salústio, construindo o seu \emph{ethos} de
 imparcialidade, apresenta duas versões diferentes (\versal{XXXIV}, 2--3): \begin{quote} 2.
   Porém, de seu trajeto, Catilina envia cartas à maioria dos consulares,
   sobretudo a todos os bons cidadãos, em que afirma estar cercado de falsas
   acusações e que, por não ter conseguido resistir à facção dos inimigos,
   cedia à fortuna, partia para o exílio em Massília, não por ser cúmplice de
   crime tamanho, mas para que a República tivesse paz e que de sua resistência
   não nascesse uma sedição. 3. Quinto Cátulo leu no senado uma carta bastante
   diferente desta, que afirmava haver recebido da parte de Catilina. Uma cópia
   dela é transcrita abaixo (\emph{Earum exemplum infra scriptum est}).
 \end{quote}

\section{As versões alternativas dos fatos e  a imparcialidade do historiador}

 A imparcialidade do historiador é construída também em algumas passagens em
 que Salústio apresenta mais de uma versão do mesmo fato, por vezes suspendendo
 seu julgamento a respeito. É o que acontece, por exemplo, no capítulo 14, em
 que o historiador, logo após sua digressão acerca da decadência dos costumes
 de Roma, exemplifica-a precisamente pela enunciação dos tipos de homens que se
 haviam associado a Catilina. Em \versal{XIV}, 7, aludindo com toda probabilidade a
 Cícero, faz menção a um rumor que, quer nos fazer notar, considerou e
 descartou em sua pesquisa: \begin{quote} Sei que houve quem julgasse que a
 juventude frequentadora da casa de Catilina tinha o pudor em pouca conta; mas
 tal rumor tinha força antes por motivos outros que por alguém o ter
 descoberto.  \end{quote} Conforme aponta Ramsey, “Cícero alude às relações
 sexuais entre Catilina e seus associados em diversas ocasiões [\ldots{}]. Tais
 acusações são uma característica padrão da invectiva, e Salústio, muito
 corretamente, tem reservas quanto a um julgamento”\footnote{ \versal{RAMSEY} (2007:
 99).}. % Jorge, aqui a Marlene pensa que talvez seja o caso de colocar esta citação de Ramsey em itálico, por ter mais de 3 linhas. Não sei o padrão.
   O que está em jogo aqui é o reforço da afirmação de
 imparcialidade por parte do historiador, de um lado, e de busca da maior
 veracidade possível, de outro, ambas programaticamente apresentadas no
 prefácio da \emph{Conjuração}, como vimos: quando Salústio aponta que havia
 outras motivações por trás de tais acusações, indiretamente quer dizer que, na
 qualidade de historiador, não as levará em conta em seu relato, porque
 constituiriam parcialidade, que, se na invectiva é virtude, na historiografia
 é vício. 
 Dessa forma, mina a autoridade
 de Cícero, cuja alusão o leitor contemporâneo facilmente identificaria, e
 amplifica a sua; já quando diz que as alegações são infundadas e não foram
 efetivamente descobertas, pretende mostrar que promoveu uma investigação
 factual independente, não se fiando cegamente nas fontes de que dispunha.  A
 mesma tática pode ser observada em \versal{XXII}, 3, passagem em que Salústio comenta o
 rumor de que Catilina teria feito os conjurados beberem sangue humano depois de
 lhe jurarem lealdade: \begin{quote} Alguns julgavam que essa e, de resto, muitas
   outras alegações haviam sido forjadas por aqueles que acreditavam abrandar o
   ódio que nascera posteriormente contra Cícero por meio da atrocidade do
   crime dos que foram executados. A nós, tal fato, em virtude de sua
   magnitude, não tem comprovação suficiente.  \end{quote} Mesmo não tendo
   comprovação suficiente, Salústio não deixa de apresentar o rumor. A nosso
   ver, isso tem uma dupla motivação: o relato, por sua natureza escabrosa,
   confere vivacidade e variedade à narrativa, cumprindo a função do deleite;
   por outro lado, oferece a Salústio nova oportunidade de reforçar seu
   \emph{ethos} de historiador imparcial e pesquisador da verdade. Esta segunda
   observação, por sinal, também explica a enumeração das diversas versões do
   suposto envolvimento de Marco Licínio Crasso\footnote{Marco Licínio Crasso
   (115--53), o senador que formara, a partir de 60, uma aliança de
 poder e influência com Pompeu e César, a que os historiadores tradicionalmente
 dão o nome de “primeiro  triunvirato”.} na conjuração, no capítulo \versal{XLVIII}, bem
 como a observação de que o próprio Salústio ouvira, em pessoa, uma explicação
 da boca de Crasso, o que coloca o historiador no centro dos acontecimentos e,
 consequentemente, confere-lhe maior credibilidade: \begin{quote} Houve,
   naquela época, quem julgasse que aquilo fora maquinado por Públio Autrônio
   para que, no caso de se acusar Crasso, ficasse mais fácil, com sua
   influência, proteger os demais com sua cumplicidade. Outros afirmavam que
   Tarquínio fora instigado por Cícero, a fim de que Crasso não perturbasse a
   ordem pública ao assumir, como era seu costume, a defesa de homens
   desonestos. Eu ouvi o próprio Crasso afirmar em público, posteriormente, que
   aquela terrível injúria fora-lhe imputada por Cícero.  \end{quote} Idêntica
   estratégia de construção do \emph{ethos} de historiador, por fim, parece
   motivar, no capítulo 53, as observações acerca das muitas leituras,
   pesquisas e reflexões empreendidas por Salústio, numa passagem em que está a
   examinar as causas das grandes façanhas do povo romano no passado:
   \begin{quote} Ora, quanto a mim, que muito li, muito ouvi sobre as façanhas que o
     povo romano realizou na paz como na guerra, em mar e em terra, calhei de
     achar por bem observar que fatores sobretudo contribuíram para que se
     sustentassem tamanhas empresas. [\ldots{}] E para mim, que muito refletia, era
     evidente que o egrégio valor de uns poucos cidadãos levara tudo a cabo, e
     por isso acontecera que a pobreza superasse as riquezas, que um pequeno
     número vencesse uma multidão.  \end{quote}



\section{A elocução da \emph{Conjuração de Catilina}}

 Passando ao domínio da elocução, a característica mais marcante da escrita de
 Salústio, a nosso ver, é sua tentativa de conferir um tom arcaizante a suas
 obras, que se observa em diversos níveis: ortográfico, morfológico, lexical,
 sintático. Comentando o aspecto lexical, Ernout assim explica a sua função: 
 
 % Jorge, este ldots da citação não está aparecendo no texto de saída.
 
   \begin{quote} [\ldots{}] esse recurso aos termos antigos
   parecia-lhe também necessário numa obra que exaltava as virtudes da Roma de
   outrora, opondo-as à corrupção contemporânea; enfim, ela conferia a sua obra
   aquela \emph{grauitas} [“gravidade”] em que via o equivalente da
   \emph{semnótes} [“solenidade”] tucididiana. Assim, Catão e o historiador
   grego foram os dois modelos que ele se propôs: foi a eles que emprestou esse
   colorido antigo, esforçando-se, geralmente com sucesso, para evitar a
   imperícia do primeiro ou a obscuridade do segundo\footnote{\versal{ERNOUT} (1996: 27). Tradução nossa.}.  \end{quote}

Goodyear, por outro lado, pensa em três possíveis motivações para esse tom:

\begin{quote} Sua motivação não pode ser explicada de maneira simples. Sem
  dúvida ele busca ampliar sua expressão, conferir-lhe uma dignidade
  particular: os romanos acreditavam (e não temos razão para duvidar) que o
  arcaísmo podia ter esse efeito. Ademais, quer mostrar que ele, pelo menos,
  fará uso das abundantes riquezas do latim mais antigo, que outros estavam
  cada vez mais renegando. Mas, acima de tudo, ao empregar a parataxe e evitar
  os padrões rítmicos caros aos oradores, ele de bom grado pareceria direto e
  honesto, como os menos artísticos dos antigos analistas. As armadilhas da
  retórica não cabem a um homem que professa dizer a verdade de maneira direta
  e breve\footnote{\versal{GOODYEAR} (1996: 104). Tradução nossa. O autor parece empregar
    anacronicamente a palavra “retórica”, aqui, em sentido pejorativo.}.
 
\end{quote} No nível ortográfico, Salústio emprega as formas antigas em u em
lugar das mais recentes, em i, como em \emph{maxume} em lugar de \emph{maxime}
[“sobretudo”], \emph{lubido} em lugar de \emph{libido} [“desejo”], \emph{lubet}
em lugar de \emph{libet} [“apraz”], entre muitos outros termos; no nível
morfológico, emprega a desinência arcaica de terceira pessoa do plural do
pretérito perfeito, \emph{–ere}, em lugar de \emph{–erunt}, como
\emph{inuasere} em lugar de \emph{inuaserunt} [“tomaram de assalto”]; no nível lexical, temos o
emprego de um vocabulário arcaizante e poético em diversas ocasiões, como no
caso de \emph{algor} [“frio”] em lugar de \emph{frigus} [“frio”], por exemplo,
ou o uso de um termo em seu sentido mais antigo, como no caso de
\emph{uenenum}, usado em sentido neutro, “droga”, apesar de, na época de
Salústio, ter já adquirido o sentido negativo de “droga maléfica”, “veneno”;
por fim, no nível sintático, sobretudo a parataxe\footnote{Cf. texto de
\versal{GOODYEAR} citado acima.}.  A \emph{inconcinnitas} (“assimetria”) é uma das
características da elocução das obras de Salústio que mais se afasta da
preceituação ciceroniana para o gênero historiográfico, que demandava uma
cadência regular e uniforme para o discurso historiográfico.  Assim, por
exemplo, em lugar de \emph{alii\dots alii} (“uns\ldots outros”), Salústio
emprega, ainda que de maneira não sistemática (o que tornaria o efeito
previsível), \emph{pars\dots alii} (“parte\dots outros”), \emph{alii\ldots{}pars}
(“uns\ldots parte”) e mesmo \emph{alii\dots pars\dots plerique}
(“uns\ldots{}parte\ldots{}a maioria”): \begin{quote} Então de início os reis --- pois tal
  foi, sobre as terras, o primeiro nome do poder ---, opostos, parte
  [\emph{pars}] exercitava a inteligência, outros [\emph{alii}], o corpo (\versal{II},1).

 
De fato, para dizer a verdade em poucas, quantos, a partir dali, perturbaram a
República sob pretextos honrosos, uns [\emph{alii}], como a defender os
direitos do povo, parte [\emph{pars}], para que a autoridade do senado fosse a
maior possível, lutavam, cada um, por seu próprio poder, simulando o bem
público (\versal{XXXVIII}, 3).
 
Porém, quando Tarquínio mencionou Crasso [\ldots{}], alguns [\emph{alii}]
consideraram o caso indigno de crédito, parte [\emph{pars}],  embora o
julgasse verdadeiro, por lhe parecer que em tais circunstâncias cumpria antes
abrandar que provocar o enorme poder daquele homem, a maioria
[\emph{plerique}], devedora a Crasso por negócios particulares, conclamam que a
denúncia é falsa [\ldots{}] (\versal{XLVIII}, 5).

\end{quote} Outras vezes, um advérbio é contraposto a uma locução adverbial com
preposição, como em \versal{XLII}, 2, \emph{inconsulte ac ueluti per dementiam}
(“irrefletidamente e como que por demência”), ou um substantivo é contraposto a
uma oração, como em \versal{IX}, 3, \emph{duabus his artibus, audacia in bello, ubi pax
euenerat aequitate, seque remque publicam curabant} (“Com estas duas
qualidades, a audácia na guerra, quando a paz chegara, a equidade, cuidavam de
si e da república”).

A concisão salustiana, característica que parece tomar a Tucídides, é um dos
traços da escrita da obra do historiador que mais chamou a atenção dos leitores
antigos, se podemos julgar pela quantidade de comentários críticos a esse
respeito que chegaram até nós. Quintiliano, por exemplo, faz menção à
\emph{breuitas Sallustiana} (“concisão salustiana”; 4,~2,~45 e 10,~1,~32), a  seu \emph{abruptum sermonis genus }(“estilo abrupto de linguagem”; 4, 2, 45) e,
expressão que se tornaria célebre, à \emph{immortalis Sallustiana uelocitas}
(“velocidade imortal de Salústio”; 10, 1, 102); Aulo Gélio denomina nosso autor
\emph{subtilissimus breuitatis artifex} (“um artífice da concisão extremamente
preciso”: 3, 1, 6); Sêneca, o Rétor, como já apontado, compara a concisão de
Salústio com a de Tucídides, pronunciando-se pela superioridade do primeiro
(\emph{Con.} 9, 1, 13); e Sêneca, o filósofo, enfim, numa passagem de sua
Epístola 114 em que está a tratar dos erros cometidos por Lúcio Arrúncio,
imitador de Salústio, assim caracteriza a concisão salustiana: \begin{quote}
% Jorge, aqui também o ldots não está aparecendo no texto de saída
  [\ldots{}] quando Salústio estava na moda considerava-se o suprassumo do estilo:
  as sentenças abruptas, as frases inesperadamente sincopadas, a concisão
  obscura\footnote{ \versal{SÊNECA}, \emph{Ep.}, 114, 17
(tradução de J. A. \versal{SEGURADO E} \versal{CAMPOS}.}  \end{quote} Quatro são os fatores que contribuem para a concisão de
Salústio\footnote{Cf. \versal{RAMSEY} (2007: 12), de cuja lista se toma boa parte dos
exemplos dados a seguir. O autor  pensa, na verdade, em cinco fatores, mas o
quarto, o uso de pares diametralmente opostos para evitar enumerações
fastidiosas, não nos parece convincente.}: 1) elipse; 2) assíndeto; 3) uso
frequente de infinitivos históricos; e 4) parataxe.  


1) Uma célebre passagem em
que ocorre a elipse do verbo \emph{esse} (“ser”) é a do retrato de Catilina, em \versal{V},
3-4: \begin{quote} \emph{Corpus patiens inediae, algoris, uigiliae [sc. erat],
  supra quam cuiquam credibile est. Animus audax, subdolus, uarius, cuius rei
lubet simulator ac dissimulator [sc. erat]; alieni adpetens, sui profusus [sc.
erat]; ardens in cupiditatibus [sc. erat]; satis eloquentiae, sapientiae parum
[sc. erat].}

Seu corpo suportava, mais do que se pode crer, a fome, o frio, o sono. Seu
ânimo era ousado, astuto, versátil, simulador e dissimulador do que quer que
fosse; desejoso do alheio, dissipador do que era seu, ardente nas paixões;
grande era sua eloquência, sua sabedoria, parca.  \end{quote}

2) Dos muitos exemplos de assíndeto na Conjuração de Catilina, o mais extenso e
ousado parece ser o que se encontra em \versal{X}, 1: \begin{quote} \emph{Sed ubi
  labore atque iustitia res publica creuit, reges magni bello domiti, nationes
ferae et populi ingentes ui subacti, Carthago, aemula imperi Romani, ab stirpe
interiit, cuncta maria terraeque patebant, saeuire fortuna ac miscere omnia
coepit.}

Porém, quando a república cresceu com o labor e a justiça, grandes reis foram
domados pela guerra, gentes feras e povos poderosos foram submetidos à força,
Cartago, rival do poderio romano, foi aniquilada pela raiz, todos os mares e
terras estavam abertos, a fortuna passou a se enfurecer e a pôr tudo em
desordem.  \end{quote} 3) O uso dos infinitivos históricos, ou seja, verbos no
infinitivo assumindo o papel de verbos flexionados, traduzidos
sistematicamente, aqui, pelo nosso presente histórico, confere vivacidade a
diversas passagens da Conjuração de Catilina, como ocorre em \versal{VI}, 4-5:
\begin{quote} \emph{Igitur reges populique finitumi bello temptare, pauci ex
  amicis auxilio esse; nam ceteri, metu perculsi, a periculis aberant. At
Romani, domi militiaeque intenti festinare, parare, alius alium hortari,
hostibus obuiam ire, libertatem, patriam parentesque armis tegere.}

Então os reis e os povos vizinhos os põem à prova com a guerra, poucos, dentre
os aliados, vêm-lhes em auxílio, pois os demais, abatidos pelo medo,
mantinham-se apartados dos perigos. Porém, os romanos, atentos na paz como
na guerra, de pronto agem, preparam-se, encorajam-se uns aos outros, avançam
contra os inimigos, defendem a liberdade, a pátria e os pais com suas armas.
\end{quote} Por vezes ocorre a alternância entre infinitivo histórico e verbo
flexionado dentro de um mesmo período, acrescentando-se, à vivacidade, um
efeito de estranheza e assimetria, como em \versal{XXI}, 4: \begin{quote} \emph{Ad hoc
  maledictis increpabat omnis bonos; suorum unumquemque nominans laudare;
admonebat alium egestatis, alium cupiditatis suae, complures periculi aut
ignominiae, multos uictoriae Sullanae, quibus ea praedae fuerat.} 

Além disso,
atacava a todos os bons cidadãos com impropérios, elogia cada um dos seus
chamando-os pelo nome; relembrava, a um, sua falta de recursos, a outro, sua
cobiça, a diversos, o perigo ou a ignomínia, a muitos, a vitória de Sila, que
lhes propiciara espólios.  \end{quote} 4) O uso da parataxe, ou seja, a
tendência a usar preferencialmente a justaposição de orações coordenadas em
lugar da subordinação, além de conferir maior concisão à escrita da \emph{Conjuração},
é também traço arcaizante, como já dissemos. Um exemplo dessa estratégia
sintática ocorre logo no início da obra, quando encontramos uma oração
introduzida por \emph{nam} (“pois”) em lugar de uma subordinada relativa:
\begin{quote} \emph{Igitur initio reges – nam in terris nomen imperi id primum
  fuit – diuorsi, pars ingenium, alii corpus exercebant.}

Então de início os reis – pois tal foi, sobre as terras, o primeiro nome do
poder –, opostos, parte exercitava a inteligência, outros, o corpo (\versal{II},1).
\end{quote}

\section{A tradução}

 A presente tradução, baseada na edição crítica de Alfred Ernout para a coleção
 Les Belles Lettres (1996), procura manter o tom arcaizante do original
 sobretudo pela escolha do léxico. Daí o uso de termos como “viajores” em lugar
 de “viajantes”, “súplices” em lugar de “suplicantes”, “urbe” em lugar de
 “cidade”, “gentes” em lugar de “povos”, “labor” em lugar de “trabalho”,
 “gestas” em lugar de “feitos”, “sofrer” em lugar de “suportar”, entre outros
 que deixamos ao leitor atento descobrir.  Das demais características da
 elocução salustiana, algumas são vertidas literalmente, dada a possibilidade
 oferecida pela língua portuguesa de se manter o mesmo efeito, como ocorre no
 caso do uso dos assíndetos, por exemplo, ou da assimetria, conforme ilustram
 os exemplos citados acima. Outras, porém, como os quiasmos e as figuras
 etimológicas, nem sempre são mantidas na tradução. Optamos, em lugar de uma
 literalidade que seria servil, pois que não funcional, adotar a chamada
 compensação tradutória, procurando resgatar o mesmo efeito em outras
 passagens.  Em algumas passagens procuramos imitar, no sentido técnico
 e antigo do termo, a tradução de Barreto Feio, tradutor do século \versal{XVIII}
 recentemente redescoberto entre nós com a publicação de sua versão da \emph{Eneida},
 de Vergílio, organizada e anotada por Paulo Sérgio de Vasconcellos. O leitor
 curioso fará a comparação entre os contextos das passagens que imitam e as
 imitadas, podendo daí tirar suas conclusões acerca das escolhas feitas. 

\section{Agradecimentos do tradutor}

 
Agradeço a Marlene Lessa Vergílio Borges pelo apoio, incentivo e,
particularmente,  pelas  incontáveis sugestões de melhorias na tradução; aos
meus professores de latim da \versal{USP}, que me apresentaram Salústio e a
\emph{Conjuração de Catilina}; aos meus alunos, que em vários
cursos de língua latina criticaram e melhoraram os primeiros esboços da tradução; a Adriano Aprigliano, Marcelo Vieira Fernandes e Breno Battistin Sebastiani pela leitura cuidadosa de uma primeira versão da tradução, há quase dez anos; a
minha família, pelo apoio, suporte e carinho incondicionais; a meus colegas-amigos/amigos-colegas e, particularmente, a Santo
Edi, Saint Lévy, Sainte Hélène e Saint Pierre,  salvadores.

E à Adriana, sempre, por tudo. 

\bigskip
\hfill Adriano Scatolin e Marlene L.V.~Borges


\begin{bibliohedra}

\section{Primária}

\tit{AGOSTINHO}. \emph{A Cidade de Deus. (Contra os pagãos)}. Parte I (Livros I a X). 6ª Ed. Trad. Oscar Paes Leme. Petrópolis: Vozes, 2001.

\tit{APIANN}. \emph{Roman History}. Vol. III. English translation by Horace White. London, Cambridge, Massachusetts: Harvard University Press, 2002.

\tit{CLARK, A. C.} \emph{M. Tulli Ciceronis Pro T. Annio Milone: Ad Iudices Oratio}. Oxford, Clarendoon Press, 1895. Oxford, 1895.


\tit{CICERO.} \emph{As Catilinárias, Defesa de Murena, Defesa de Árquias, Defesa de Milão}. 1974. Intr. Trad. notas Sebastião Tavares de Pinho. Lisboa: Verbo, 1974.

\tit{CÍCERO}. ``\emph{Ad. fam. 12}, tradução de Adriano Scatolin'', \emph{in Lingua e Literatura}, 27 (2002/2003), São Paulo: Humanitas, 2010.

\tit{CICERO.} \emph{Letters to Friends}, Vol. II (edited and translated by D. R. SHACKLETON BAILEY). Cambridge/London, Harvard University Press, 2001.

\tit{CICERO.} \emph{Select Letters} (edited by D. R. SHACKLETON-BAILEY). Cambridge, Cambridge University Press, 2000.

\tit{CICERONE.} \emph{Dell’oratore} (con un saggio de Emanuele Narducci). Milano, Biblioteca Universale Rizzoli, 2006.

\tit{DIO CASSIUS}.  \emph{Dio’s Roman History}.Vol. III. With an English Translation by Earnest Cary.  Cambridge, Massachussetts: Harvard University Press; London: William Heinemann Ltd, 1984. 

\tit{DIO CASSIUS}.   \emph{Dio’s Roman History}.Vol. IV. With an English Translation by Earnest Cary.  Cambridge, Massachusetts: Harvard University Press; London: William Heinemann Ltd, 1987. 

\tit{DYCK, A. R.} \emph{A Commentary on Cicero, \emph{De Legibus}.} Ann Arbor, The University of Michigan Press, 2004.


\tit{ERNOUT, A.} \emph{Salluste --- Catilina, Jugurtha, Fragments des Histoires}. Les Belles Lettres, 1996.

\tit{LEWIS, R. G.} \emph{Asconnius Commentaries on Speeches by Cicero}. Oxford University Press, New York, 2006.


\tit{LUCIAN.} “How to write History”. In: Lucian, Vol. VI (with an English translation by K. KILBURN). Cambridge/London, Harvard University Press, 1990, pp. 1-73.

\tit{OROSE}. \emph{Histoires (Contre les Païens)}. Livres I-III. Texte établi et traduit par Marie-Pierre Arnaud-Lindet. Paris: Les Belles Lettres, 1990. 

\tit{RAMSEY, J. T.} \emph{Sallust’s Bellum Catilinae}. 2ª ed., Oxford University Press, Oxford, New York, 2007,   p. 2. 

\tit{VELLEIUS PATERCULUS e FLORUS}.  \emph{Histoire Romaine}. Texte revu et traduit par P. Hainsselin et H. Watelet. Paris: Librairie Garnier Frères, 1932.  


\tit{PINHO, S. T.} de. (introdução, tradução e notas das \emph{Catilinárias}). Cicero. Vol. 1. Lisboa/São Paulo, Verbo, 1974. 

\tit{PLINIUS.} \emph{Epistularum Libri Decem} (recognouit breuique adnotatione critica instruxit R. A. B. MYNORS). Oxonii, e typographeo Clarendoniano, 1968. 

\tit{PLINY THE YOUNGER.} \emph{The Letters of Pliny the Younger} (translated with an introduction by Betty Radice). London, Penguin Books, s/d.


\tit{SALLUST.} \emph{Bellum Catilinae} (edited, with introduction and commentary, by J. T. RAMSEY, 2nd edition). Oxford University Press, 2007.

\tit{SALLUSTE.} \emph{La Conjuration de Catilina}. Texte établi et traduit par Alfred Ernout; introduction et notes par Martine Chassignet. Paris, Les Belles Lettres, 1999.

\tit{SALLUSTE.} \emph{La Conjuration de Catilina, La Guerre de Jugurtha, Fragment des Histoires}. Texte établi et traduit par Alfred Ernout, revu et corrigé para J. Hellegouarc’h. Paris, Les Belles Lettres, 1996.

\tit{SALLUSTIO.} \emph{Sallustio em Portuguez} (por J. V. BARRETO FEIO). Paris, Livraria Nacional e Estrangeira, 1825.

\tit{SALLUSTIUS.} \emph{De Catilinae coniuratione} (a cura di E. MALCOVATI).Torino, Paravia, 1956.


\tit{SALUSTIO.} \emph{Conjuración de Catilina, Guerra de Jugurta, Fragmentos de las “Historias”}. Madrid, Gredos,

1997.

\tit{SALÚSTIO.} \emph{A Conjuração de Catilina, A Guerra de Jugurta} (tradução de Antônio da Silveira Mendonça). Petrópolis, Vozes, 1990.


\tit{SÊNECA.} \emph{Cartas a Lucílio} (tradução, prefácio e notas de J. A. SEGURADO E CAMPOS). Lisboa, Fundação Calouste Gulbenkian, 1991.

\tit{SÉNECA, EL VIEJO}. \emph{Controversias} (libros VI-X) - \emph{Suasorias}. Trad. y notas de Ignacio Javier A. Lajara, Esther Artigas Alvarez y Alejandra de Riquer Permanyer. Madrid: Editorial Gredos, 2005. 

\tit{SUETONIUS}. \emph{The lives of the Caesars (continued), The lives of illustrious men}. Vol. II, English translation by J. C. Rolfe. Cambridge, Massachusetts: Harvard University Press; London: William Heinemann, Ltd, 1979.

\tit{QUINTILIAN}. \emph{Institutio Oratoria of Quintilian}. Vol. I (Books I, II, III). Translation H. E. Butler. Cambridge, Massachusetts, London, 1989.

\titidem \emph{The Orator’s Education}. Books 3-5. Edited and translated by Donald A. Russell. Cambridge, Massachusetts, London: Harvard University press, 2001.

\titidem \emph{The Orator’s Education}. Books 6-8. Edited and Translated by Donald A. Russel. Cambridge, Massachusetts, London: Harvard University press, 2001.


\titidem \emph{The Orator’s Education}. Books 9-10. Edited and Translated by Donald A. Russel. Cambridge, Massachusetts, London: Harvard University press, 2001.

\tit{QUINTILIANO.} \emph{La Formazione dell’Oratore}, Volume Terzo, libri IX-XII (traduzione e note di C. M. CALCANTE. Milano, Biblioteca Universale Rizzoli, 1997.


\tit{TACITE}.  \emph{Annales} Livres 1-III. Texte établi et traduit par Henri Goelzer. Paris: Les Belles Lettres, 1953. 
\tit{WOODMAN, A. J.} “A Note on Sallust, Catilina 1.1”. In: The Classical Quarterly, New Series, Vol. 23, No. 2, 1973, p. 310.

\section{Secundária}



\tit{AMBRÓSIO, R.} \emph{De rationibus exordiendi: os princípios da história em Roma}. São Paulo, Humanitas, 2005.



\tit{BAKER, R. J.} ``Sallustian Silence”. \emph{Latomus}, T. 41, Fasc. 42, 1982, pp. 801–802.

\tit{CHASSIGNET, M.} ``Introduction'' \emph{in Salluste, La Conjuration de Catilina}. Paris: Les Belles Lettres, 1999.  

 
\tit{CHIAPPETTA, A.} ``Não Diferem o Historiador e o Poeta … Texto Histórico como Instrumento e Objeto de Trabalho”. \emph{Língua e Literatura}, 22, 1996, pp. 15-34.

 \tit{EARL, D.} “Prologue-Form in Ancient Historiography”. In: \emph{Aufstieg und Niedergang der Römischen Welt – Von den Anfängen Roms bis zum Ausgang der Republik}, v. 2. Berlin / New York, Walter de Gruyter, 1972, pp. 842- 856.

 

\tit{FRIER, B.} \emph{Libri Annales Pontificum Maximorum – The Origins of the Annalistic Tradition}. The University of Michigan Press, 1999-2002. 

\tit{KRAUS, C. S. \& WOODMAN, A. J.} “Sallust”. In: \emph{Greece \& Rome}, New Surveys in the Classics No. 27, Latin Historians. Oxford, Oxford University Press, 1997.

\tit{KRISTELLER, P. O.  \&  BROWN, V. ( org.)} \emph{Catalogus Translationum et Commentariorum: Mediaeval and Renaissance Latin Translations and Commentaries : Annotated Lists and Guides}. Washington: CUA Press, 2003, 188.

\tit{GOODYEAR, F. D. R}. “Sallust”. In: \emph{The Cambridge History of Classical Literature}, Vol. II, Part 2 – The Late Republic (edited by E. J. KENNEY). Cambridge, Cambridge University Press, pp. 94-106.

\tit{GRETHLEIN, J.} “Nam quid ea memorem: The Dialectical Relation of Res Gestae and memoria rerum gestarum in Sallust’s Bellum Iugurthinum. In: \emph{Classical Quarterly}, 56, 2006, pp. 135-148.

\tit{HARTOG, F.} (org.). \emph{A História de Homero a Santo Agostinho}. Belo Horizonte, Editora \versal{UFMG}, 2001. 


\tit{LEEMAN, A. D.}, “Review of \emph{Sallust} by Ronald Syme”, \emph{Gnomon}, 39. Bd., H. (mar. 1967), pp. 57-61.

 
 

\tit{MARINCOLA, J.} \emph{Authority and Tradition in Ancient Historiography}. Cambridge, Cambridge University Press, 1999.

 

\tit{OSMOND, P. J.  \& ULERY, R. W.} “Sallustius Fortuna”, \emph{in, Catalogus Translationum et Commentariorum: Mediaeval and Renaissance Latin Translations and Commentaries: Annotated Lists and Guides}. Washington: CUA Press, 2003.

 
 

\tit{RENEHAN, R.} “A Traditional Pattern of Imitation in Sallust and His Sources”. In: \emph{Classical Philology}, Vol. 71, 1976, pp. 97-105.

\tit{SEBASTIANI, B. B.} “A política como objeto de estudo: Tito Lívio e o pensamento historiográfico romano do século I. a. C.”  in \emph{História e retórica: ensaios sobre historiografia antiga.} JOLY, Fabio D. (org.). São Paulo: Alameda, 2007, p. 79. 
 
 \tit{SYME, R.} \emph{Sallust}. London: University Of California Press, 1964.

 

\tit{VASCONCELLOS, P. S. de}. \emph{Efeitos intertextuais na Eneida de Virgílio}. São Paulo, Humanitas, 2001.



\tit{USHER, S.} \emph{The historians of Greece and Rome}. University of Oklahoma Press, Norman/ Bristol Classsical Press (1969) 1985.


\tit{WOODMAN, A. J.} “A Note on Sallust, Catilina 1.1”. In: \emph{The Classical Quarterly}, New Series, Vol. 23, No. 2, 1973, p. 310.

---------------------- \emph{Rhetoric in Classical Historiography}. Routledge, London/New York: 1988.


\end{bibliohedra}

 






 
