\part{\titulo}


\chapter*{}


I. 1. \versal{A todos os homens que} anseiam superar os restantes animais, cumpre
empenharem-se com o máximo afinco para que não atravessem a vida no
silêncio\footnote{``No silêncio'', como tradução de \emph{silentio}, em vez de
simplesmente ``em silêncio'', para tentar passar o duplo sentido do original:
``em silêncio", sim, porque não se manifesta, mas também ``em meio ao
silêncio'', porque ninguém fala a seu respeito (ideia explicitada em II, 8).
Cf. Woodman (1973) e Baker (1982).}, tal como o gado, que a natureza criou
curvado e obediente ao ventre. 2. Ora, todo o nosso poder reside na
mente\footnote{\emph{Animus} é termo polissêmico: pode traduzir 
\grego ψυχή\selectlanguage{brazilian}, tendo então o significado de ``alma'', mas
também pode significar ``ânimo'', ``espírito'' \ e ``mente''. Adotamos esta
última opção por dois motivos: para diferenciar de  \emph{anima}, que Salústio
também usa para traduzir \grego ψυχή
\selectlanguage{brazilian} (\emph{Catilina},
I, 8 e \emph{Jugurta}, II, 1: ``De fato, tal como a raça humana é composta de
corpo e alma (\emph{anima}), assim também todas as nossas atividades e
aspirações seguem, umas, a natureza do corpo, outras, a da mente
(\emph{animus})''.); e o contexto do prefácio da \emph{Conjuração} parece
concernir sobretudo ao aspecto do pensamento e da inteligência.} 
e no corpo: da mente, valemo-nos mais para comandar, do corpo, para servir; compartilhamos uma com
os deuses, o outro, com as feras. 3. Daí parecer-me mais correto buscar a
glória pelos recursos da inteligência\footnote{\emph{Ingenium}, além de
``inteligência'', pode  significar ainda ``engenho'', ``talento'', noções que
também parecem presentes no contexto.} do que pelos da força e, uma vez que a
própria vida que gozamos é breve, tornar o mais duradoura possível a recordação
de nós mesmos. 4. De fato, a glória do dinheiro e da beleza é fugaz e frágil, a
virtude\footnote{\emph{Virtus} é a qualidade do \emph{vir}, ''varão'', podendo
significar, além de ``virtude'', ``bravura'', ``coragem'' e, tal como o grego
\grego ἀρετή\selectlanguage{brazilian}, ``excelência''. Nossa opção por
``virtude''\ deve-se ao caráter moralizante do prefácio.} é uma posse brilhante
e eterna. 5. Ora, largo tempo disputou-se com ardor entre os
mortais\footnote{Ao longo da \emph{Conjuração}, Salústio utiliza poeticamente a
forma \emph{mortalis}, ``mortal'', para falar dos humanos e do que é humano.}
se a arte militar teria melhor êxito pela força do corpo ou pelo valor da
mente. De fato, é preciso, antes de passar à ação, deliberar e, deliberado,
logo agir. Desse modo, ambos, insuficientes por si só, carecem um do auxílio do
outro.

II. 1. Então de início os reis --- pois tal foi, no mundo, o primeiro
nome do poder ---, opostos, parte exercitava a inteligência, outros, o corpo;
até então a vida dos homens transcorria sem cobiça, a cada um bastava o que era
seu\footnote{Do ponto de vista moderno, este retrato idealizado do passado nada
tem de histórico. Na economia da obra, entretanto, ele é coerente com a visão
de decadência dos costumes que fundamenta o entendimento e a apresentação dos
fatos e dos agentes históricos na  \emph{Conjuração}.}. 2. Porém, só depois que
Ciro\footnote{Fundador do Império Persa, que governou de c. 559--530.}, na Ásia, os lacedemônios e os atenienses, na Grécia,
passaram a subjugar urbes e gentes, a ter, como pretexto de guerra, o desejo de
domínio, a depositar a maior glória na maior potência, é que se notou, pela
experiência e pela prática, que, na guerra, sobretudo a inteligência tem poder.
3. É que se o vigor da mente de reis e comandantes tivesse tanta força na paz
como na guerra, seriam mais uniformes e constantes as humanas coisas, nem se
veria cada uma arrastada para um lado, nem tudo a se mudar e confundir. 4. Pois
fácil se mantém um domínio pelos dotes com que foi de início conquistado. 5.
Porém, quando acometem, em lugar do labor, a indolência, em lugar da temperança
e da equidade, o desejo e a soberba, muda-se com os costumes a fortuna. 6.
Assim, o poder sempre passa do menos bom a alguém melhor. 7. O que aos homens
vem do arar, navegar, edificar obedece inteiramente à virtude. 8. Mas muitos
mortais, entregues ao ventre e ao sono, indoutos e incultos, atravessam a vida
tal como viajores. Para eles, em clara oposição à natureza, o corpo é fonte de
prazer, a alma, um fardo. Deles eu julgo vida e morte próximas, pois que de
ambas se cala. 9. Parece-me viver e desfrutar da vida apenas quem, ocupado com
alguma atividade, busca a glória de feito ilustre ou de bom dote\footnote{A
glória que Salústio busca advirá do bom uso de seu engenho, convertido na
escrita da \emph{Conjuração}.}.

III. 1. Porém, na imensa soma dos acontecimentos, a cada um a natureza mostra o
seu caminho. É belo agir bem pela República, mas não destoa o dizer bem. Na paz
como na guerra é possível tornar-se ilustre; os que realizaram, os que
escreveram as realizações alheias, em grande número se louvam\footnote{``Os que
realizaram'', como os comandantes de guerra; ``os que escreveram'', como os
historiadores que abordaram os feitos de tais comandantes. Exemplos célebres,
na época de Salústio, eram os historiadores de Alexandre, o Grande, e Teófanes
de Mitilene, historiador de Pompeu Magno. Eles são mencionados conjuntamente
por Cícero na \emph{Defesa de Árquias}, 24.}. 2. E a mim, pelo menos, embora
seja absolutamente diversa a glória que acompanha o escritor e o realizador dos
feitos, parece-me sobremaneira árduo escrever as gestas: primeiro, porque se
devem igualar feitos e ditos\footnote{Entenda-se: para feitos ou fatos
grandiosos será preciso encontrar palavras adequadas. Salústio tem em mente o
\emph{modus scribendi} do historiador, a elocução decorosa e apta que deverá
empregar para fazer jus à grandiosidade do tema.}; em seguida, porque a maioria
considera os delitos que se censuram fruto de malevolência e inveja; quando se
rememoram o grande valor e a glória dos homens de bem, cada qual recebe com
indiferença o que julga fácil fazer, o que lhe está acima, toma por falso, como
forjado\footnote{Salústio refere-se aos dois aspectos complementares da
história escrita com viés moral: a crítica dos vícios e o louvor das
virtudes.}. 3. Mas eu, ainda bem jovem, de início, tal como a maioria,
lancei-me com ardor à vida pública, e enfrentaram-me aí muitas adversidades.
Pois imperavam, em lugar do pudor, da integridade, da virtude, a audácia, a
largueza, a avidez. 4. Embora meu ânimo, desafeito das más condutas, as
desprezasse, em meio a tamanhos vícios minha idade vacilante\footnote{Também
nos processos judiciais a defesa de um réu, ou, pelo menos, a atenuação de uma
admissão de culpa, baseavam-se muitas vezes na pouca idade do acusado. É o que
acontece, por exemplo, na defesa que Cícero fez de seu pupilo Marco Célio, em
56 a.C. Cf. \emph{Cael.} 11, 75 e 77.} mantinha-se corrompida pela ambição; 5. e a mim, divergindo embora dos
maus costumes dos demais, atormentava-me, pela fama e pela inveja, o mesmo
desejo de honra que aos restantes.

IV. 1. Então, quando meu ânimo teve descanso das muitas misérias e perigos e
decidi manter o restante de minha vida longe da vida pública, não quis exaurir
meu bom ócio na apatia e na indolência, ou mesmo passar a vida voltado ao
cultivo do campo ou à caça, tarefas de escravos\footnote{Ou seja, não querendo
ou não podendo mais servir à República pela participação ativa na política,
Salústio será útil a ela por meio da escrita da história.}; 2. mas, tornando
àquele mesmo projeto e estudo de que me apartara a má ambição, decidi minuciar
os feitos do povo romano por partes, conforme cada um parecesse digno de
recordação\footnote{``Por partes'' indica a adoção da forma da monografia
histórica; ''cada um''\ parece apontar para um projeto historiográfico em
várias monografias (Salústio comporia apenas mais uma, a \emph{Guerra de
Jugurta}); ``digno de recordação''\ é o critério de escolha do tema
histórico.}; tanto mais que meu ânimo estava livre das expectativas, receios,
facções da política\footnote{Trata-se da tópica da imparcialidade do
historiador.}. 3. Assim, concisamente\footnote{``Concisamente'' refere-se tanto
à adoção da monografia histórica como à elocução do autor, ou seja, sua maneira
de escrever.}, descreverei a conjuração de Catilina\footnote{Enunciação do tema
e do recorte da monografia, passagem de que os estudiosos também derivam o
título da obra.} da maneira mais verídica possível\footnote{A  busca da verdade
também era uma tópica tradicional, quase protocolar, nos prefácios dos
historiadores. Ela está intimamente relacionada à imparcialidade do
historiador, permitindo-lhe não distorcer os fatos por simpatia ou antipatia
pelos agentes históricos.}, 4. pois considero tal feito sobremaneira memorável
pela novidade do crime e do perigo\footnote{Outra tópica, preceituada nos
manuais de retórica para os exórdios dos discursos em geral e adotadas em
particular no gênero histórico, é a da grandiosidade do tema. A finalidade é
tornar o leitor atento ao assunto. Cf. \emph{Rhet. Her.} 1, 7; Cícero, \emph{Inv.} 1, 23.}. 5. É preciso que fale umas poucas
palavras acerca do caráter desse homem antes que dê início à
narração\footnote{Dado o viés moral da obra, torna-se essencial um retrato do
protagonista da trama. O retrato explica e antecipa características de Catilina
ao longo da \emph{Conjuração}, como apontaremos em nota.}.

V. 1. Lúcio Catilina, oriundo de linhagem nobre\footnote{A linhagem da
\emph{gens Sergia}, de antiquíssima tradição em Roma.}, tinha grande vigor
intelectual e físico, mas uma índole perversa e depravada\footnote{É exatamente
assim que Salústio representará Catilina na descrição da batalha final, em LX,
4.}. 2. Desde a adolescência eram-lhe caras as guerras internas, as matanças,
as pilhagens, a discórdia civil\footnote{Alusão à participação de Catilina na
guerra civil dos anos 80, na facção sulana, e de seu envolvimento nas
proscrições que se seguiram.}, e nisso ocupou sua juventude\footnote{Ocorre
aqui um contraste implícito: enquanto Salústio abandonara as práticas
corrompidas, restringindo-as à juventude, para depois buscar a glória na
escrita da história, Catilina mantém, quando adulto, a índole que já
demonstrara em seus anos mais jovens.}. 3. Seu corpo suportava, mais do que se
pode crer, a fome, o frio, o sono\footnote{Cf. Cícero, \emph{Catil.} 1,
26:  ``Tens aí onde possas ostentar essa tua famosa capacidade em suportar a
fome, o frio, a carência de tudo; e em breve perceberás que foi isso que deu
cabo de ti''; e 3, 16: ``podia suportar o frio, a sede e a fome'' (referências
em Ramsey (2007) \emph{ad locum}; tradução de Pinho (1974)).}. 4. Seu ânimo era
ousado, astuto, versátil, simulador e dissimulador do que quer que fosse;
desejoso do alheio, dissipador do que era seu, ardente nas paixões; grande era
sua eloquência\footnote{Como será exemplificado no
discurso de Catilina às tropas, antes da batalha final contra as forças do
lugar-tenente Marco Petreio (LVIII, 1--21).}, sua sabedoria, parca. 5. Seu ânimo insaciável sempre
desejava o desmedido, o inacreditável, o inatingível. 6. Depois da tirania de
Lúcio Sula, assaltara-o um enorme desejo de tomar a República, não medindo os
meios para consegui-lo, contanto que obtivesse a monarquia. 7. Agitava-se mais
e mais, com os dias, seu ânimo bravio em virtude do esgotamento do patrimônio
familiar e do remorso dos crimes, alimentados, um e outro, pelas qualidades a
que fiz menção anteriormente. 8. Incitavam-no, de resto, os costumes
corrompidos da cidade, que eram movidos por males terríveis e opostos, o luxo e
a avidez. 9. Já que a ocasião nos lembrou os costumes da cidade, o próprio tema
parece exigir que remontemos ao passado\footnote{Salústio busca evidenciar que
a digressão sobre o passado de Roma que fará em seguida está íntima e
adequadamente ligada ao tema em questão, não se tratando de desvio ornamental
do assunto.} e, breves\footnote{Novamente acena, a um só tempo, para a extensão
da digressão e para o seu modo conciso de escrita.}, discutamos as instituições
de nossos antepassados na paz e na guerra, de que modo governaram a República e
com que extensão a legaram, como, mudando aos poucos, de bela e grandiosa que
era, tornou-se extremamente perversa e dissoluta.

VI. 1. A cidade de Roma, segundo a tradição, fundaram e habitaram de início os
troianos, que, fugitivos, sob o comando de Eneias, erravam sem morada certa, e,
com eles, os aborígines, raça agreste, sem leis, sem governo, livre e
desregrada. 2. Depois que se reuniram dentro de uma única muralha, incrível é
mencionar a facilidade com que, malgrado a diversidade de raças, a diferença
entre as línguas, vivendo cada qual segundo seu costume, acabaram por se
fundir: \textless assim, breve, uma multidão diversa e errante tornou-se numa
cidade\footnote{Sérvio Honorato, comentando \emph{Eneida}, 1, 6, aponta a
filiação de Salústio às \emph{Origens} de Catão no que concerne ao papel dos
aborígenes na história de Roma: ``[...] Catão, nas \emph{Origens}, cuja
autoridade Salústio segue na \emph{Guerra de Catilina}, afirma o seguinte:
`Originalmente, habitavam a Itália certos homens que eram denominados
aborígines. Posteriormente, com a chegada de Eneias, eles se misturaram aos
frígios e foram denominados latinos.' '' (referência em Ramsey (2007) \emph{ad
locum.})}.\textgreater

3. Mas depois que sua cidade cresceu em virtude de seus cidadãos, seus
costumes, suas terras, e parecia já bastante próspera, bastante poderosa, tal
como se dá na maior parte do que é mortal, da opulência nasceu a inveja. 4.
Então os reis e os povos vizinhos os põem à prova com a guerra, poucos, dentre
os aliados, vêm-lhes em auxílio, pois os demais, abatidos pelo medo,
mantinham-se apartados dos perigos. 5. Porém, os romanos, atentos na paz como
na guerra, de pronto agem, preparam-se, encorajam-se uns aos outros, avançam
contra os inimigos, defendem a liberdade, a pátria e os pais com suas armas.
Depois, quando haviam afastado já os perigos com seu valor, levavam auxílio aos
aliados e amigos e forjavam alianças antes concedendo que recebendo benefícios.
6. Tinham um governo legítimo, como nome do governo, a monarquia. Os
escolhidos, que tinham o corpo debilitado já pelos anos, o entendimento
saudável pela sabedoria, velavam pela república; esses, fosse pela idade, fosse
pela semelhança do encargo, eram denominados “pais”. 7. Depois, quando o
governo monárquico, que de início servira para preservar a liberdade e ampliar
a república, converteu-se em soberba e tirania\footnote{Alusão a Tarquínio, o
Soberbo, o último rei de Roma.}, mudaram a tradição\footnote{Com a derrubada da
Monarquia, em 510 a.C., por obra de Lúcio Júnio  Bruto.}, estabelecendo
governos anuais e dois governantes\footnote{O governo anual era o consulado. A
cada ano, dois novos cônsules assumiam o cargo. A contagem dos anos passou a
ser por isso feita com base no nome dos cônsules de cada ano.}: desse modo,
julgavam mínimas as possibilidades de o espírito humano assoberbar-se pelo
poder\footnote{Em teoria, o fato de sempre haver dois cônsules dividindo  o
poder em pé de igualdade serviria como garantia de que um não se projetaria
sozinho acima dos demais cidadãos, por uma espécie de equilíbrio de forças.}.

VII. 1. Ora, naquele tempo, eles passaram, cada qual, a distinguir-se e a pôr
mais à mostra seu engenho. 2. De fato, para os reis, os homens de bem são mais
suspeitos do que os maus, e o valor dos outros lhes é sempre temível. 3. Mas a
Urbe, incrível é mencionar com que rapidez cresceu depois de conseguir a
liberdade\footnote{Com o fim da Monarquia e o estabelecimento da República.},
tamanho desejo de glória a tomara. 4. Agora, por primeiro, a juventude, tão
logo se tornava apta a suportar a guerra, aprendia a arte militar na prática,
com as operações nos acampamentos, e encontravam\footnote{Atente-se à
construção \emph{ad sensum}: como ``juventude'' pressupõe uma ideia de
coletivo, equivalendo aqui a ``jovens'', Salústio alterna o tempo verbal do
singular para o plural.} maior prazer em luzentes armas e cavalos de guerra do
que em rameiras e banquetes. 5. Então, para tais homens nenhuma fadiga era
estranha, nenhum lugar era difícil ou inacessível, nenhum inimigo armado,
temível: a bravura superara todos os obstáculos. 6. Ora, havia, entre eles, uma
enorme disputa pela glória: cada qual esforçava-se por golpear o inimigo,
escalar muralhas, ser visto ao fazer tal façanha. Julgavam essas as riquezas,
essa a boa reputação e a grande nobreza. Eram ávidos de louvor, pródigos de
dinheiro; desejavam glória enorme, riqueza honrosa. 7. Enumerara em que
ocasiões o povo romano dispersou vastas tropas de inimigos com um pequeno
contingente, que cidades naturalmente protegidas tomou em combate, se isso não
me afastasse de minha empresa.

VIII. 1. Ora, claramente, a Fortuna em tudo é soberana: ela, antes por capricho
que justiça, a tudo torna célebre ou obscuro. 2. As façanhas dos atenienses,
segundo julgo, foram bastante grandiosas e magníficas, mas não tão importantes,
todavia, quanto afirma a tradição. 3. Porém, por haverem florescido ali
grandiosos engenhos de escritores, os feitos dos atenienses são celebrados como
os mais importantes pelo mundo. 4. Assim, o valor dos que os realizaram é
considerado tanto maior quanto os engenhos ilustres os puderam exaltar com
palavras. 5. Mas o povo romano nunca teve tal oportunidade\footnote{Salústio
pretende preencher a alegada lacuna com a escrita de sua obra.}, porque os mais
sagazes eram os mais ocupados: ninguém exercitava a inteligência sem o corpo,
os melhores preferiam agir a falar, que outros louvassem suas grandes empresas
a eles próprios narrarem as alheias.

IX. 1. Assim, na paz como na guerra cultivavam-se os bons costumes; suma era a
concórdia, a avidez, quase nenhuma. Entre eles, a justiça e a retidão não
vigoravam mais pelas leis do que pela natureza\footnote{Entenda-se: os antigos
eram justos e corretos por sua própria natureza, sem terem de se ver coagidos a
tais virtudes pela lei e, como está implícito, pelos castigos impostos por
esta.}. 2. Entregavam-se às rixas, às discórdias, às rivalidades com os
inimigos, cidadãos disputavam com cidadãos pelo mérito\footnote{Observação que
deveria repercutir fortemente entre os leitores contemporâneos de Salústio, já
que a escrita da obra deu-se em época de turbulência e de guerras civis: a
lembrança da guerra entre César e Pompeu ainda era recente e, dependendo da
data de escrita --- certamente depois dos Idos de Março, embora não se saiba ao
certo quando ---, já se aproximavam ou já estavam
em andamento novos conflitos. Cf. Ramsey (2007: 6--7).}. Nas oferendas aos deuses eram magníficos, em
casa, comedidos, aos amigos, fiéis. 3. Com estas duas qualidades, a audácia na
guerra, quando a paz chegara, a equidade, cuidavam de si e da República. 4. Tenho,
como as maiores provas disso\footnote{Com a enumeração das provas de sua
afirmação, Salústio busca consolidar sua autoridade de historiador e a
credibilidade de seu relato.}, os seguintes fatos: o de, na guerra, terem-se
punido com maior frequência os que enfrentaram os inimigos desobedecendo às
ordens e os que, com ordens de se retirar do combate, tardaram a deixá-lo, que
os que ousaram abandonar os estandartes ou, expulsos de sua posição, bater em
retirada; 5. já na paz, o fato de exercerem o poder antes por benefícios que
pelo medo, e preferirem perdoar as ofensas recebidas a procurar vingança. 

X. 1. Porém, quando a República cresceu com o labor e a justiça, grandes reis
foram domados pela guerra, gentes feras e povos poderosos foram submetidos à
força, Cartago, rival do poderio romano, foi aniquilada pela raiz, todos os
mares e terras estavam abertos, a Fortuna passou a se enfurecer e a pôr tudo em
desordem\footnote{Salústio estabelece, com esta observação, o fim da Terceira
Guerra Púnica, em 146, como o momento crucial, na história de Roma, em que os
costumes da Cidade começaram a corromper-se.}. 2. Os que haviam facilmente
suportado as fadigas, os perigos, as situações incertas e difíceis, a eles o
ócio e a riqueza, desejáveis em outras circunstâncias, foram fardo e miséria.
3. Então, primeiro cresceu o desejo de dinheiro, depois, o de poder: estes
foram como que a substância de todos os males. 4. De fato, a avidez fez caírem
por terra a lealdade, a integridade e os demais bons dotes; em lugar deles,
ensinou a soberba, a crueldade, a descurar os deuses, a tudo ter na conta de
venal. 5. A ambição constrangeu muitos mortais a se tornarem falsos, a ter uma
coisa encerrada no peito, outra à mostra nos lábios, a julgar amizades e
inimizades não por si mesmas, mas por conveniência, e a ter antes a fisionomia
que a índole boa. 6. Tais males, de início, crescem aos poucos, por vezes se
punem. Depois, quando a infecção se espalhou como uma peste, a cidade mudou, o
governo, de mui justo e bom que era, tornou-se cruel e intolerável.

XI. 1. Ora, de início a ambição, mais do que a avidez, inquietava os ânimos dos
homens, embora tal vício estivesse bem próximo à virtude. 2. É que tanto o
honesto como o indolente buscam para si a glória, a honra, o poder, mas aquele
avança pelo reto caminho; por faltarem a este bons dotes, luta por meio de
enganos e mentiras. 3. A avidez contém em si o desejo de dinheiro, o que sábio
nenhum anseia; ela, como que impregnada de drogas maléficas, efemina o corpo e
o ânimo viris, é sempre infinita, insaciável, não diminui com a riqueza ou a
pobreza. 4. Ora, depois que Lúcio Sula tomou o poder pelas armas e fez que a
bons inícios se seguissem maus resultados, todos roubam, saqueiam, este deseja
uma casa, terras, aquele; vencedores, não têm moderação ou medida, cometem
crimes terríveis e cruéis contra os cidadãos\footnote{Salústio refere-se à
guerra civil de 83--82, à ditadura de Sula que se seguiu e às proscrições e
abusos que decorreram de seu mando.}. 5. Somava-se a isso o fato de Lúcio Sula,
contra os costumes ancestrais, manter o exército que comandara na Ásia em meio
ao luxo e à excessiva libertinagem, a fim de garantir sua lealdade. Paragens
amenas, aprazíveis haviam facilmente debilitado, na paz, os bravos ânimos dos
soldados. 6. Ali, pela primeira vez, o exército do povo romano acostumou-se a
fornicar, beber, admirar estátuas, pinturas, vasos entalhados, saqueá-los
privada e publicamente, espoliar os santuários, macular tudo o que é sacro ou
profano. 7. Tais soldados, então, depois de alcançar a vitória, nada deixaram
aos vencidos. Se a prosperidade atormenta os ânimos dos sábios, tampouco eles,
com seus costumes corrompidos, poriam limite à vitória.

XII. 1. Depois que as riquezas passaram a ser fonte de honra e a elas seguiam a
glória, o poder, a influência, começou a enfraquecer-se a virtude, a posse de
pouco, a ser tida em desprezo, a honestidade, a passar por malevolência. 2.
Então, pelas riquezas, o luxo e a avidez, mais a soberba, assaltaram a
juventude: saqueiam, gastam, pouco apreciam o que é seu, desejam o alheio,
consideram indiferentes o recato, o pudor, o divino e o humano, não têm
qualquer escrúpulo ou moderação. 3. Vale a pena, ao conhecer as casas e as
vilas construídas à maneira de cidades, visitar os templos dos deuses, que
nossos antepassados, mortais profundamente religiosos, construíram. 4. Estes,
porém, ornavam os santuários dos deuses com sua devoção, suas casas, com sua
glória, e não tomavam aos vencidos senão a possibilidade de ofensa. 5. Já
aqueles, em contrapartida, homens por demais indolentes, tomam aos aliados, no
maior dos crimes, tudo o que os vencedores, bravíssimos varões, deixaram ---
como se empregar o poder consistisse precisamente em cometer ofensas.  XIII. 1.
De fato, por que mencionar aquilo em que não podem crer senão os que o viram:
montanhas postas abaixo, mares aterrados por diversos cidadãos?\footnote{Ernout
(1996: 67, n. 2) e Chassignet (1999: 23, n. 40) apontam Luculo como
possível referência desta alusão. De fato, a acreditar no relato do historiador
Veleio Patérculo (2, 33), Luculo recebera de Pompeu Magno, por conta justamente
desse tipo de prática de construção que avançava sobre o mar e de nivelamento
de terras montanhosas, a alcunha de ``Xerxes de toga''. Evidentemente, não está
descartado que outros aristocratas também fossem visados pelo moralista.} 2. A
eles, parece-me, as riquezas eram motivo de escárnio, uma vez que se
apressavam, pela torpeza, em abusar das que lhes era concedido ter
honestamente. 3. Ora, espalhara-se um desejo não menor de adultério, orgias e
demais dissipações: os homens comportam-se como mulheres, estas põem à venda a
castidade; tudo examinam em terra e mar em busca de diferentes iguarias; dormem
antes de ter sono; não aguardam fome ou sede, nem frio ou cansaço, mas
antecipam-nos todos pelo excesso. 4. Tais coisas incitavam a juventude aos
crimes, quando os bens familiares faltavam: não era fácil, a um ânimo
impregnado de vícios, privar-se dos prazeres; por isso entregara-se por todos
os meios aos ganhos e aos gastos, sem qualquer moderação.

XIV. 1. Numa cidade tão grande e tão corrompida, Catilina mantinha a seu redor,
como guardas --- o que era extremamente fácil de fazer ---, bandos de
escândalos e delitos de toda espécie\footnote{Uma imagem forte: escândalos e
delitos são personificados, para compor a guarda de Catilina!}. 2. De fato,
todo aquele que, devasso, adúltero, glutão, dilacerara os bens paternos com os
dados, o ventre, o pênis\footnote{Outra imagem bastante forte, direta e
concreta, como referência ao jogo, à gula e à fornicação.}, bem como os que contraíram enormes dívidas para obter a absolvição
de um escândalo ou delito; 3. demais, os assassinos todos de toda parte,
sacrílegos, condenados em julgamento ou no temor deste por seus atos; além
disso, aqueles a quem a mão e a língua alimentavam com perjúrio ou sangue
civil, todos, por fim, a quem atormentavam a desonra, a pobreza, a culpa ---
esses eram íntimos e familiares de Catilina\footnote{Assim como Catilina, seus
cúmplices de conspiração são pintados com as grossas tintas da invectiva.}. 4.
É que mesmo alguém isento de culpa, se incorria em sua amizade, fácil
tornava-se igual e semelhante aos demais em virtude do hábito e dos atrativos
cotidianos. 5. Porém, ele buscava sobretudo a intimidade dos jovens: seus
ânimos, maleáveis ainda e vacilantes [pela idade], eram capturados sem
dificuldade por seus enganos. 6. De fato, conforme ardia o desejo de cada um, 
por conta da idade, a uns fornece prostitutas, a outros, compra cães e cavalos; por
fim, não poupava dinheiro ou dignidade, contanto que os tornasse submissos e
leais a si. 7. Sei que houve quem julgasse que a juventude frequentadora da
casa de Catilina tinha o pudor em pouca conta\footnote{Conforme apontam os
comentadores, temos aqui uma alusão a Cícero, que retrata a depravação dos
conspiradores em várias passagens de seu \emph{corpus} oratório:
\emph{Catil.} 2, 8; 2, 23--24; \emph{Red. Sen.} 10; \emph{Dom.} 62 (referências em Ernout
(1996: 69, n. 1) e Ramsey (2007: 99), em nota a \emph{nonnullos qui}).};
mas tal rumor tinha força antes por motivos outros\footnote{``Motivos outros''\
seriam o interesse de Cícero em denigrir seu adversário, tática retórica que o
orador adota, em seus discursos, para tratar de todos os seus grandes inimigos:
além de Catilina, Verres, Clódio, Vatínio, Pisão, Gabínio e Marco Antônio. Salústio, em contrapartida, busca uma apresentação mais isenta, em conformidade com seu ofício de historiador.} que por
alguém o ter efetivamente descoberto.

XV. 1. Já de início o jovem Catilina cometera diversos adultérios abomináveis:
com uma virgem nobre, com uma sacerdotisa de Vesta, outros do gênero contra a
lei humana e divina. 2. Por fim, tomado de amor por Aurélia Orestila, de quem
um homem honesto jamais louvou senão a beleza, passa por certo que, como
ela hesitava em desposá-lo, temendo um enteado em idade adulta, Catilina matou
seu filho e esvaziou sua casa para as núpcias criminosas. 3. Parece-me que esse
foi o principal motivo de apressar o crime. 4. De fato, seu ânimo impuro,
infesto a deuses e homens, não conseguia ter descanso, fosse na vigília, fosse
no sono, de tal modo o remorso devastava sua mente aterrorizada. 5. Então sua
cor era pálida, turvos, seus olhos; seu passo, rápido por vezes, por vezes,
lento: em suma, a insânia estava em sua face e em seu rosto. 

XVI. 1. Ora, à juventude, que, como dissemos acima, seduzira, ele ensinava, por
muitos meios, crimes terríveis. 2. Dentre eles\footnote{Ou seja, os jovens, ideia que se depreende de ``juventude''. Nova \emph{constructio ad sensum}.} fornece testemunhas e
signatários falsos; ordenava que tivessem em pouca conta a lealdade, as
fortunas, os processos\footnote{Repare-se na alternância de tempos verbais
(``fornece'' --- ``ordenava'').}; depois, quando esmagara sua reputação e
pudor, outros delitos ainda mais graves. 3. Se faltava momentaneamente um
pretexto para o erro, mesmo assim os fazia cercar, matar os inculpáveis como
culpados.  Sem dúvida, para que as mãos ou o ânimo não enfraquecessem pela ociosidade,
era de preferência perverso e cruel gratuitamente. 4. Fiando-se em tais amigos
e aliados, a um tempo porque eram enormes as dívidas por toda a parte e porque
a maioria dos soldados de Sula, depois de dissipar o que era seu, lembrada das
rapinas e da antiga vitória, desejava a guerra civil, concebeu o plano de
aniquilar a República. 5. Na Itália não havia nenhum exército, Gneu Pompeu
promovia a guerra nos confins do mundo\footnote{Pompeu detinha então um comando
militar para enfrentar o rei Mitridades do Ponto, no oriente.};  ele tinha
grande esperança em sua candidatura ao consulado\footnote{Entenda-se:
Catilina.}, o Senado estava absolutamente desatento: tudo estava seguro e
tranquilo, mas isso era de todo propício a Catilina.

XVII. 1. Então, por volta das calendas de junho, no consulado de Lúcio César e
Caio Fígulo\footnote{Ou seja, cerca de 1\oi\ de junho de 64.}, de início
convoca-os todos, exorta a uns, a outros, sonda, informa-os de seus recursos,
do despreparo da República, das grandes recompensas da conjuração. 2. Depois de
assegurar o que pretendia, convoca a um único lugar todos os que tinham enorme
necessidade e audácia inda maior. 3. Ali reuniram-se, da ordem senatorial,
Públio Lêntulo Sura, Públio Autrônio, Lúcio Cássio Longino, Caio Cetego, Públio
e Sérvio Sula, filhos de Sérvio, Lúcio Vargunteio, Quinto Ânio, Marco Pórcio
Leca, Lúcio Béstia, Quinto Cúrio; 4. demais, da ordem equestre, Marco Fúlvio
Nobílior, Lúcio Estatílio, Públio Gabínio Cápito, Caio Cornélio; além disso,
muitos nobres locais das colônias e municípios. 5. Havia, de resto, diversos
nobres participando desse plano de maneira mais velada, encorajados mais pela
esperança de tirania do que por pobreza ou alguma precisão\footnote{Salústio
deixa à imaginação do leitor a identidade de tais membros da nobreza:
tratar-se-ia de Marco Crasso? De Júlio César?}. 6. Além disso, a maior parte da
juventude --- embora sobretudo da nobreza --- era favorável aos desígnios de
Catilina: pudera embora viver no ócio, já na suntuosidade, já na lascívia,
preferiam\footnote{Construção \emph{ad sensum} (``pudera'' --- ``preferiam''),
uma vez mais com a palavra ``juventude''.} o incerto ao certo, a guerra à paz.
7. Houve também, naquela época, quem cresse que Marco Licínio Crasso não
desconhecia seu plano: que, pelo fato de Gneu Pompeu, a quem odiava, comandar
um grande exército, desejava que os recursos de quem quer que fosse crescessem
contra a influência deste, ao mesmo tempo convicto de que, se a conjuração
ganhasse força, facilmente se tornaria o líder entre eles\footnote{Outra
construção \emph{ad sensum}: o termo ``eles'', referente aos conjurados, é
sugerido pelo substantivo ``conjuração''.}.

XVIII. 1. Ora, do mesmo modo, anteriormente, uns poucos conjuraram contra a República, dentre os quais estava Catilina. Dela\footnote{Nova construção \emph{ad
sensum}, desta vez em sentido inverso da que se encontra em XVII, 7: o verbo
``conjuraram''\ sugere a ideia de ``conjuração'', a que se refere o termo
``dela''.} falarei da maneira mais verídica possível\footnote{Retoma a tópica,
já explorada no prefácio (IV, 3), da busca da verdade, decorrente da
imparcialidade do historiador.}. 2. No consulado de Lúcio Tulo e Mânio
Lépido\footnote{Isto é, em 66.}, Públio Autrônio e Públio Sula, designados
cônsules\footnote{Salústio inverte a formulação tradicional, ``cônsules
designados''.}, foram acusados de acordo com as leis de corrupção eleitoral e
condenados. 3. Pouco tempo depois, Catilina, acusado de concussão, fora
impedido de disputar o consulado porque não conseguira apresentar sua
candidatura dentro do prazo legal. 4. Havia, na mesma época, Gneu Pisão, um
jovem nobre de extrema audácia, pobre, sedicioso, a quem a penúria e o mau
caráter estimulavam à perturbação da ordem pública. 5. Com ele, Catilina e
Autrônio, por volta das nonas de dezembro\footnote{Ou seja, por volta de 5 de
dezembro.}, depois de informarem seu plano, preparavam-se para assassinar os
cônsules Lúcio Cota e Lúcio Torquato no Capitólio, nas calendas de
janeiro\footnote{Isto é, em 1\oi\ de janeiro.}, tomar eles próprios os
fasces\footnote{Os fasces eram feixes de varas presos por tiras vermelhas que
eram usados como insígnias do poder dos magistrados.} e enviar Pisão com um
exército para ocupar as duas Hispânias. 6. Conhecida a trama, adiaram novamente
o plano do assassinato para as nonas de fevereiro\footnote{Isto é, para 5 de
fevereiro.}. 7. Já então maquinavam a ruína não apenas dos cônsules, mas da
maior parte dos senadores. 8. É que se Catilina não se tivesse precipitado em
dar o sinal a seus cúmplices diante da Cúria\footnote{A Cúria, localizada no
fórum, era a sede do senado e lugar de boa parte de suas sessões (o senado
também podia se reunir em lugares consagrados, como templos, por exemplo. É o
que acontece na sessão do senado convocada por Cícero para enfrentar a crise,
realizada no Templo da Concórdia. Cf. XLVI, 5.).}, naquele dia teria sido
perpetrado o mais terrível crime desde a fundação da cidade de Roma. Por não se
terem reunido homens armados em número suficiente, tal fato frustrou o plano.
XIX. 1. Posteriormente, Pisão foi enviado à Hispânia citerior como questor
propretor por influência de Crasso, porque o sabia inimigo infesto a Gneu
Pompeu. 2. O Senado, contudo, não lhe concedera a província de mau grado, já
que desejava longe da República esse homem abominável; ao mesmo tempo, muitos
dos bons cidadãos julgavam que ele representava segurança, e já então era
temível o poderio de Gneu Pompeu. 3. Mas esse Pisão foi morto na província
pelos cavaleiros hispanos que comandava no exército, durante uma marcha. 4. Há
quem diga que os bárbaros não puderam suportar suas ordens injustas,
arrogantes, cruéis; outros, por sua vez, que esses cavaleiros, clientes antigos
e fiéis de Gneu Pompeu, atacaram Pisão com o consentimento daquele; que os
hispanos, além disso, nunca cometeram um crime tal, mas toleraram muitos
governos duros anteriormente. Quanto a nós, deixemos tal questão em
suspenso\footnote{Coerentemente com o protocolo de imparcialidade do
historiador, Salústio pesa as possibilidades e, na falta de evidências
conclusivas, suspende seu julgamento.}. 5. Da primeira conjuração já se falou o
bastante\footnote{Há quem considere fictícia a existência da primeira
conjuração de Catilina. Leiam-se, a respeito, as palavras de Ramsey (2007: 
237--239).}.  XX. 1. Catilina, quando vê reunidos os que há pouco mencionei,
embora houvesse não raro tratado de diversos pormenores com cada um deles,
julgando que seria útil falar e exortar a todos juntamente, retira-se para uma
parte oculta de sua casa e ali, distante de qualquer testemunha, proferiu um
discurso do seguinte teor:

2. “Se não houvesse eu mesmo comprovado vosso valor e lealdade, inutilmente se
mostrara oportuna a ocasião; vã fora a grande esperança, a tirania já em nossas
mãos, e eu não tomara, em virtude da covardia ou das falsas naturezas, por
certo o que era incerto. 3. Porém, porque vos reconheci fortes e fiéis a mim em
muitas e graves crises, meu ânimo ousou conceber o maior e mais belo dos
feitos, ao mesmo tempo por perceber serem, para vós, os bens e os males os
mesmos que para mim: 4. de fato, querer e não querer o mesmo, eis,
precisamente, o que é uma sólida amizade. 5. Mas todos vós já ouvistes
anteriormente, em diversas ocasiões, o que tenho revolvido em minha mente. 6.
Além disso, meu ânimo inflama-se mais a cada dia, quando considero qual há de
ser nossa condição de vida se nós mesmos não nos libertarmos. 7. De fato,
depois que a República cedeu à lei e ao domínio de uns poucos poderosos, sempre
os reis, os tetrarcas são-lhes tributários, os povos, as gentes pagam-lhes
impostos; todos os demais, ativos, honestos, nobres e não nobres, formamos o
vulgo, sem crédito, sem autoridade, submissos àqueles a quem, fizera a
República jus ao nome, causaríamos terror. 8. E assim, todo crédito,
influência, honraria, riqueza estão entre eles ou onde pretendem que estejam; a
nós deixaram as derrotas, os perigos, os processos, a pobreza. 9. Até quando,
então, o suportareis, bravíssimos varões?\footnote{Paródia do célebre começo da
primeira \emph{Catilinária} (1,1): \emph{quousque tandem, Catilina, abutere
patientia nostra?} [``Afinal, até quando, Catilina, abusarás de nossa
paciência?'']} Não é preferível morrer com valor a perder, com infâmia, uma
vida miserável e desonrosa, em que se foi motivo de escárnio da soberba alheia?
10. Mas, a bem da verdade, pela fé de deuses e homens, a vitória está em nossas
mãos, é valerosa a idade, o ânimo, forte; em contrapartida, tudo o que têm
envelheceu com os anos e as riquezas. É preciso apenas o primeiro passo, o
resto virá por si. 11. E, de fato, quem, dentre os mortais de natureza viril,
pode sofrer que lhes abundem as riquezas, para dilapidarem-nas construindo
sobre o mar e aplainando as montanhas\footnote{As palavras de Catilina ecoam as
de Salústio, em sua digressão sobre a derrocada dos costumes de Roma (XIII,
1).}, que a nós falte o patrimônio familiar mesmo para o necessário? Que
conjugem duas ou mais casas, enquanto em parte alguma temos um lar familiar?
Enquanto compram quadros, estátuas, vasos, derrubam construções novas,
constroem outras, em suma, por todos os meios pilham, devastam, não conseguem,
todavia, apesar de seu imenso desejo, dar-lhes fim. 13. Nós, porém, temos a
penúria em privado, as dívidas em público, uma realidade terrível, uma
perspectiva muito mais dura: enfim, que nos resta senão um miserável sopro de
vida? 14. Por que, então, não despertais? Eis que aquela liberdade, aquela que
sempre desejastes, de resto, as riquezas, a honra, a glória encontram-se diante
de vossos olhos; a Fortuna colocou-as todas como prêmios aos vencedores. 15. A
situação, a ocasião, os riscos, a pobreza, os magníficos espólios de guerra vos
servem de maior encorajamento do que meu discurso. 16. Servi-vos de mim, seja
como comandante, seja como soldado\footnote{Catilina cumprirá sua promessa,
conforme o relato de Salústio em LX, 4.}: nem meu ânimo nem meu corpo se
apartarão de vós. 17. Farei tais coisas junto convosco, segundo espero, como
cônsul, a não ser que me engane, e estejais prontos antes a servir que a
comandar”.

XXI. 1. Depois de ouvirem tais palavras, embora aqueles  homens, sobre quem pesava todo tipo de desgraça, mas não possuíam  nenhum recurso ou boa perspectiva,
considerassem um grande ganho perturbar a ordem, exigiram, em sua maior parte, que
explicitasse quais eram os termos da guerra, que prêmios pretendiam pelas
armas, que recursos ou perspectivas tinham em cada lugar. 2. Então Catilina
promete o perdão das dívidas, a proscrição dos ricos, magistraturas,
sacerdócios, pilhagens, tudo mais que trazem a guerra e o desejo dos
vencedores\footnote{As promessas de Catilina ecoam os eventos que Salústio
expôs sobre a ditadura de Sula (XI, 4).}. 3. De resto, Pisão encontra-se na
Hispânia citerior, Públio Sítio, de Nucéria, na Mauritânia com seu exército,
cúmplices de seu plano; disputa o consulado Caio Antônio, a quem espera ter
como colega, homem não apenas seu amigo, mas também cercado de toda espécie de
necessidades; cônsul, com ele dará início à empresa\footnote{Cf. LIX, 4 e nota \emph{ad locum}.}. 4.
Além disso, atacava a todos os bons cidadãos com impropérios, elogia cada um
dos seus chamando-os pelo nome; relembrava, a um, sua falta de recursos,
a outro, sua cobiça, a diversos, o perigo ou a ignomínia, a muitos, a vitória
de Sula, que lhes propiciara espólios\footnote{Atente-se à alternância das formas
verbais: ``atacava'' --- ``elogia'' --- ``relembrava''.}. 5. Depois que vê os ânimos ansiosos de
todos, exortou-os a cuidar de sua candidatura e encerrou a reunião.

XXII. 1. Houve, naquele tempo, quem dissesse que Catilina, depois de seu
discurso, ao ligar os cúmplices de seu crime por juramento, teria servido, em
páteras, sangue de um corpo humano misturado com vinho; 2. daí, depois de todos
jurarem e o provarem, como é uso nos sacrifícios solenes, teria revelado
seu plano, assim agindo a fim de que fossem mais leais uns aos outros, como
cúmplices de tão grave feito. 3. Alguns julgavam que essa e, de resto, muitas
outras alegações haviam sido forjadas por aqueles que acreditavam poder
abrandar o ódio que nascera posteriormente contra Cícero, em virtude da
atrocidade das execuções\footnote{Cícero acabaria sendo exilado por cerca de 18
meses, em 58--57, por conta da execução sem julgamento dos 5 conspiradores
capturados.}. 4. A nós, tal fato, em virtude de sua magnitude, não tem
comprovação suficiente\footnote{Novo aceno à imparcialidade do autor, à busca
da verdade e à ponderação das evidências.}.  XXIII. 1. Ora, nessa conjuração
estava Quinto Cúrio, oriundo de estrato não obscuro, coberto de escândalos e
delitos, a quem os censores haviam expulsado do Senado por improbidade. 2. A
frivolidade desse homem não era menor do que sua audácia: não ponderava calar o
que ouvia, nem ocultar seus próprios crimes, nem, enfim, o que dizia ou fazia.
3. Tinha com Fúlvia, mulher nobre, uma antiga relação de adultério; como já não lhe agradasse, uma vez que não podia presenteá-la pela falta de recursos,
repentinamente, vangloriando-se, pôs-se a prometer mares e montes e, por vezes,
a ameaçá-la de morte, caso não se submetesse a ele; por fim, passou a agir de
maneira mais feroz do que de costume. 4. Porém, Fúlvia, ao tomar conhecimento
da causa da arrogância de Cúrio, não manteve tal perigo oculto à República,
mas, omitindo sua fonte, contou a diversas pessoas o que ouvira, e de que modo,
acerca da conjuração de Catilina. 5. Sobretudo tal fato inflamou nos homens o
desejo de conferir o consulado a Marco Túlio Cícero. 6. De fato, antes disso a
maior parte da nobreza ardia de inveja, e acreditavam que o consulado seria
como que maculado se um homem novo\footnote{``Homem novo'', entre os romanos,
designava o primeiro representante de uma família a adentrar o Senado, como era
o caso de Cícero.}, ainda que egrégio, o obtivesse. Porém, quando se aproximou
o perigo, a inveja e a soberba ficaram em segundo plano.

XXIV. 1. Então, realizadas as eleições\footnote{Em 64, para o mandato de  63.},
são proclamados cônsules Marco Túlio e Caio Antônio; esse fora o primeiro fato
a abater os cúmplices da conjuração. 2. No entanto, não diminuía a loucura de
Catilina, mas a cada dia maquina mais planos, apresta os exércitos pela Itália
em posições estratégicas, leva a Fésulas dinheiro tomado de empréstimo em seu
nome, ou no dos amigos, a certo Mânlio, que foi depois o primeiro a promover a
guerra. 3. Conta-se que, naquela época, aliciara inúmeros homens de toda
estirpe, mesmo algumas mulheres que, de início, haviam suportado enormes gastos
pela venda do corpo, depois, quando a idade pusera medida aos ganhos, mas não
ao luxo, haviam contraído grandes dívidas. 4. Por meio delas, Catilina
acreditava poder sublevar a escravaria urbana, incendiar a Urbe, aliciar seus
maridos ou matá-los.

XXV. 1. Ora, entre elas estava Semprônia\footnote{Discute-se a relevância do
retrato de Semprônia na economia da \emph{Conjuração de Catilina}. É de notar
que o próprio Salústio, tão ansioso por demonstrar a adequação do retrato de
Catilina e da digressão sobre o passado de Roma, nada diz aqui nesse sentido.
Cf. IV, 5; V, 9; VII, 7.}, que não raro cometera diversos crimes de uma audácia
varonil. 2. Essa mulher era bastante afortunada por sua estirpe e beleza, além
disso, por seu marido e filhos; versada nas letras gregas e latinas, na cítara,
na dança com mais graça do que o necessário a uma mulher honesta, em muitos
outros dons que são instrumentos da luxúria. 3. Mas qualquer coisa sempre lhe
foi mais cara do que a honra e o pudor; não seria fácil discernir se poupava
menos dinheiro ou reputação; sua libido era de tal modo inflamada que mais
vezes procurava os homens do que estes a procuravam. 4. Ora, ela não raro
traíra, anteriormente, a confiança recebida, negara um empréstimo sob
juramento, fora cúmplice de assassinato, derrocara pela luxúria e privação\footnote{O latim \emph{luxuria atque inopia praeceps abierat}"\ também pode ser traduzido por ``derrocara pelo luxo e pela penúria". Neste caso, os gastos provocados pelo primeiro é que teriam acarretado o segundo.}. 5. 
Sua inteligência, contudo, não era vulgar: capaz de compor versos, provocar o
riso, empregar uma linguagem ora moderada, ora insinuante, ora obscena --- em
suma, havia nela muitos encantos e muita graça.

XXVI. 1. Mesmo com tais preparativos, Catilina disputava o consulado do ano
seguinte\footnote{Em 63, para o mandato de 62.}, na esperança, caso fosse
eleito, de manipular Antônio facilmente, à sua vontade. E não se aquietara
nesse ínterim, mas por todos os meios tramava contra Cícero. 2. Não faltavam a
este, porém, habilidade e astúcia para precaver-se. 3. De fato, desde o começo
de seu consulado, com muitas promessas conseguira, por meio de Fúlvia, que
Quinto Cúrio, que há pouco mencionei, revelasse os desígnios de Catilina. 4.
Além disso, impelira seu colega Antônio, com um acordo sobre as
províncias\footnote{Ao deixar suas magistraturas, os cônsules partiam para
governar as províncias que lhes eram designadas por sorteio. No sorteio deste
ano, coubera a Cícero a Macedônia, e a Gália a Antônio. A manobra de Cícero
consistiu em trocar de província com Antônio, ao mesmo tempo em que abdicava do
governo da Gália Cisalpina. A Macedônia seria mais interessante para Antônio porque mais
lucrativa.}, a não se posicionar contra a República; mantinha a seu redor,
veladamente, escoltas de amigos e clientes. 5. Depois que chegou o dia das
eleições e nem a candidatura, nem os atentados que fizera contra os cônsules no
Campo de Marte tiveram êxito, Catilina decidiu promover a guerra e experimentar
todos os extremos, já que o que maquinara em segredo tivera consequências
desfavoráveis e vergonhosas.

XXVII. 1. Então enviou Caio Mânlio a Fésulas e àquela região da Etrúria, certo
Septímio de Camerino, ao campo do Piceno, Caio Júlio, à Apúlia; além disso, os
demais, a lugares diversos --- quem e onde, segundo acreditava, lhe seria
oportuno. 2. Enquanto isso, em Roma, maquina diversos planos a um tempo, almeja
atentados contra os cônsules, apresta incêndios, ocupa posições estratégicas
com homens armados, ele mesmo arma-se, ordena aos mais o mesmo, exorta-os a
estarem sempre atentos e prontos, dia e noite lida, vela, não cede à vigília ou
ao labor. 3. Por fim, quando nenhuma de suas diversas ações surtiu efeito, alta
noite convoca novamente, na casa de Marco Pórcio Leca, os líderes da
conjuração, e lá, depois de muito queixar-se de sua apatia, informa que enviou
já Mânlio à multidão que aprestara para pegar em armas, bem como cada um dos
demais a uma posição estratégica para que dessem início à guerra, e que deseja
partir para o exército caso consiga, antes, eliminar Cícero --- que  é um
grande obstáculo a seus planos.

XXVIII. 1. Então, enquanto os mais se aterram e hesitam, Caio Cornélio,
cavaleiro romano, prometeu sua colaboração e decidiu, juntamente com Lúcio
Vargunteio, senador, que naquela noite, dali a pouco, adentrariam a casa de
Cícero com homens armados a pretexto de o saudar\footnote{Salústio refere-se ao
costume romano da \emph{salutatio}, segundo o qual os homens mais importantes
de Roma eram saudados, no período da manhã, por seus clientes. A prática
conferia prestígio ao patrono visitado. De acordo com o plano de Catilina,
Cornélio e Vargunteio se misturariam aos clientes de Cícero, sob pretexto de o
saudar, para assassiná-lo de surpresa.} e súbito o apunhalariam, despreparado
em sua própria casa. 2. Cúrio, quando nota quão grande perigo pende sobre o
cônsul, rapidamente informa Cícero, por meio de Fúlvia, da cilada que se
prepara. 3. Assim, impedidos de entrar, empreenderam tamanho crime em vão. 4.
Enquanto isso, Mânlio, na Etrúria, alicia a plebe, sedenta de revolução pela
miséria e pelo ressentimento da injustiça sofrida (pois perdera as terras e
todos os bens com a tirania de Sula), além de bandidos de toda espécie, de que
havia grande quantidade naquela região, e alguns dos colonos de Sula, a quem o
desejo e o luxo nada haviam deixado das grandes pilhagens.

XXIX. 1. Quando a nova chega a Cícero, ele, perturbado por um duplo mal, porque
nem podia, por iniciativa própria, defender mais tempo a cidade das insídias,
nem tinha bem claros a dimensão ou os desígnios do exército de Mânlio, submete
ao Senado a questão, já antes discutida em virtude dos rumores públicos. 2.
Assim, como é uso na maioria das situações de perigo, o senado decretou que os
cônsules lidassem para que a República não sofresse
detrimento\footnote{Trata-se do 
``senátus-consulto último'', explicado em seguida por Salústio.}. 3. Esse é o
maior poder conferido a um magistrado pelo Senado segundo a tradição romana ---
aprestar um exército, promover a guerra, por todos os meios coagir aliados e
cidadãos, ter máxima autoridade e arbítrio na paz como na guerra. De outro
modo, sem o mando do povo, o cônsul não tem direito a nenhum desses poderes. 

XXX. 1. Poucos dias depois, o senador Lúcio Sênio leu no Senado uma carta que
afirmava haver recebido em Fésulas, na qual estava escrito que Caio Mânlio
pegara em armas com uma grande multidão no sexto dia antes das calendas de
novembro\footnote{Isto é, em 27 de outubro.}. 2. A um tempo, como é uso em tal
situação, alguns anunciavam portentos e prodígios, outros, que acontecia uma
reunião, portavam-se armas, movia-se um levante de escravos em Cápua e na
Apúlia.

3. Então, por decisão do Senado, Quinto Márcio Rex foi enviado a Fésulas,
Quinto Metelo Crético, à Apúlia e cercanias --- 4. os dois encontravam-se às
portas da Urbe como comandantes, impedidos de realizar um triunfo\footnote{Se
adentrassem a cidade, os comandantes automaticamente abdicariam de seus poderes
militares e dos triunfos que pediam.} pela oposição de uma minoria afeita a
vender tudo o que é honroso ou desonroso ---, 5. mas os pretores foram
enviados, Quinto Pompeu Rufo a Cápua, Quinto Metelo Céler, ao território do Piceno,
e foi-lhes dado o encargo de aprestar um exército de acordo com a situação e o
perigo. 6. Além disso, a todo aquele que delatasse ações concernentes à
conjuração realizada contra a República, a recompensa seria, para um escravo, a
liberdade e cem mil sestércios, para um homem livre, impunidade por seu ato e
duzentos mil sestércios; 7. e decidiram, igualmente, que as companhias de
gladiadores fossem distribuídas por Cápua e pelos demais municípios de acordo
com os recursos de cada; em Roma, que houvesse guardas por toda a cidade e que
fossem comandadas pelos magistrados menores\footnote{Particularmente pelos
triúnviros capitais.}.

XXXI. 1. Tais eventos turbaram a cidade e mudaram a face de Roma. De um estado
de alegria e divertimento extremos, que o longo período de paz
gerara\footnote{Desde a guerra civil dos anos 80.}, súbito a tristeza tomou a
todos de assalto: 2. agitam-se, tremem, não confiam o bastante em lugar ou
homem algum, não fazem guerra nem têm paz, cada um mede os perigos pelo medo.
3. Além disso, as mulheres, tomadas de um temor inusitado da guerra, dada a
grandeza da República, carpem-se, erguem as mãos súplices aos céus,
comiseram-se de seus filhos pequenos, oram, tudo temem, abandonam a soberba e
os prazeres e desconfiam de si e da pátria. 4. No entanto, era bem isso o que o
ânimo cruel de Catilina provocava, embora se aprestassem guarnições e ele
próprio houvesse sido acusado por Lúcio Paulo de acordo com a lei
Pláucia\footnote{A lei Pláucia concernia a crimes por uso de violência.}. 5.
Por fim, por dissimulação ou para se justificar, como se fora convocado por
calúnia, foi ao Senado. 6. Então o cônsul Marco Túlio, fosse por temer sua
presença, fosse movido pela ira, proferiu um discurso esplêndido e útil à
República, que depois publicou em versão escrita\footnote{Salústio refere-se,
com parcimônia, à primeira \emph{Catilinária} de Cícero.}. 7. Mas, quando se
sentou, Catilina, preparado que estava para todo tipo de dissimulação, com o
rosto baixo, com a voz suplicante, começou a pedir aos senadores que não
acreditassem cegamente em nada a seu respeito; era oriundo de família tal, de
tal modo conduzira sua vida desde a juventude, que ansiava tudo quanto fosse
honesto; que não julgassem que ele, um patrício, de quem, tal como de seus
antepassados, provieram inúmeros benefícios à plebe romana, carecia da ruína da
República, enquanto esta era salva por um Marco Túlio, cidadão inquilino da
cidade de Roma\footnote{O ultraje de Catilina diz respeito às origens de
Cícero, que não era romano, como se sabe, mas de Arpino, cidade a cerca de 110
km a sudeste de Roma.}. 8. Quando somava a essa outras afrontas, todos
protestam, chamam-no de inimigo público e parricida. 9. Diz ele então,
enfurecido: --- Uma vez que, cercado de inimigos, sou lançado ao precipício,
apagarei o incêndio provocado contra mim com ruínas!

XXXII. 1. Em seguida precipitou-se da Cúria a sua casa. Ali, muito volvendo em
seu íntimo, porque não tinham sucesso as insídias contra o cônsul e sabia que a
cidade estava protegida do incêndio pelas guardas, crendo que o melhor a fazer
era ampliar o exército e, antes que se recrutassem as legiões, antecipar muito
do que seria útil na guerra, alta noite partiu com uns poucos para o
acampamento de Mânlio. 2. Mas ordena a Cetego, Lêntulo e os mais cuja audácia
manifesta conhecia, que fortaleçam os recursos da facção com os meios
possíveis, apressem as insídias contra o cônsul, aprestem a matança, os
incêndios e os mais crimes de guerra: muito em breve chegaria à cidade com um
grande exército. 3. Enquanto isso ocorre em Roma, Caio Mânlio envia, de seu contingente, uma delegação  a Márcio
Rex com instruções do seguinte teor:

XXXIII. 1. “Invocamos os deuses e os homens por testemunha, comandante, de que
não pegamos em armas contra a pátria ou para colocar os outros em perigo, mas
para que nossos corpos se salvassem da injustiça, nós que, miseráveis, empobrecidos 
pela violência e crueldade dos usurários, estamos na maior parte privados de
nossa morada ancestral, todos, de nossa reputação e fortuna; nem foi permitido
a qualquer de nós fazer uso da lei segundo a tradição ancestral, nem, perdido
nosso patrimônio, manter o corpo em liberdade, tamanha foi a crueldade dos
usurários e do pretor. 2. Não raro vossos antepassados, compadecidos da plebe
romana, socorreram-na, em sua pobreza, por seus decretos; e mais recentemente,
em nosso tempo, em virtude da grandeza das dívidas, a prata foi saldada com
bronze por consenso de todos os bons cidadãos. 3. Não raro a própria plebe,
fosse movida pelo desejo de dominação, fosse pela arrogância dos magistrados,
separou-se em armas dos patrícios. 4. Nós, porém, não buscamos poder ou
riquezas, em virtude dos quais ocorrem todas as guerras e combates entre os
mortais, mas liberdade, que homem honesto algum perdeu senão com a vida. 5.
Rogamos a ti e ao Senado, velai pelos cidadãos miseráveis, restituí o amparo da
lei, que a iniquidade do pretor arrebatou, para não nos impor a necessidade de
buscar um modo de morrer depois de vingar plenamente o nosso sangue.”

XXXIV. 1. A ela Quinto Márcio respondeu que, se desejavam solicitar algo ao Senado, depusessem as armas, partissem súplices para Roma; que o Senado do Povo
Romano sempre mostrara brandura e misericórdia tais, que ninguém jamais lhe
pedira auxílio em vão. 2. Porém, de seu trajeto, Catilina envia cartas à maioria
dos consulares, sobretudo a todos os bons cidadãos, em que afirma estar
cercado de falsas acusações e que, por não ter conseguido resistir à facção dos
inimigos, cedia à fortuna, partia para o exílio em Massília\footnote{Massília,
hoje Marselha, na França, era cidade que com frequência recebia os exilados de
Roma.}, não por ser cúmplice de crime tamanho, mas para que a República tivesse
paz e que de sua resistência não nascesse uma sedição. 3. Quinto Cátulo leu no
Senado uma carta bastante diferente desta, que afirmava haver recebido da parte
de Catilina. Uma cópia dela é transcrita abaixo.

XXXV. 1. “Lúcio Catilina a Quinto Cátulo. Tua notável lealdade, reconhecida
quando posta à prova, cara a mim em meio a meus graves perigos, trouxe
confiança a esta minha recomendação. 2. Por isso, decidi não preparar uma defesa
neste meu novo plano: determinei, por não ter nenhuma cumplicidade neste crime,
apresentar uma justificativa, a qual, valha-me o deus da boa-fé, talvez possas
reconhecer como verdadeira. 3. Movido por injustiças e ultrajes, uma vez que,
privado do fruto de meu trabalho e indústria, não alcançava preservar a posição
de minha dignidade, assumi a causa pública dos miseráveis, segundo meu costume;
não que não pudesse saldar as dívidas em meu nome com minhas posses --- e
também aquelas em nome de outros seriam totalmente saldadas pela liberalidade
de Orestila, por meio de seus bens e dos de sua filha ---, mas por perceber
homens indignos sendo honrados com honrarias\footnote{Trata-se da chamada
\emph{figura etymologica}.} e sentir que fora abandonado em virtude de uma
falsa suspeita. 4. Em nome disso, segui a esperança bastante honesta, dada a
minha condição, de preservar o restante de minha dignidade. 5. Embora deseje
escrever mais, anunciam-me que se prepara o uso da força contra mim. 6. Agora
recomendo-te Orestila e a confio a tua lealdade; imploro, por teus filhos, que
a defendas de injustiças. Adeus.”

XXXVI. 1. Ora, enquanto o próprio Catilina, depois de demorar-se uns poucos
dias na casa de Caio Flamínio, no território de Arécio, fornece armas à vizinhança
que já antes incitara, dirige-se ao acampamento de Mânlio com os fasces e
outras insígnias de poder. 2. Quando se sabe disso em Roma, o Senado julga
Catilina e Mânlio inimigos públicos, estabelece, para a restante multidão, um
prazo dentro do qual seria possível depor as armas impunemente, à exceção dos
condenados à morte. 3. Demais, decide que os cônsules recrutem tropas, Antônio
se apresse a perseguir Catilina com um exército, Cícero defenda a cidade. 4.
Naquele tempo, o império do povo romano pareceu-me num estado absolutamente
deplorável. Embora todas as terras domadas pelas armas lhe obedecessem do
nascer ao pôr do sol, em Roma afluíssem a paz e as riquezas, que os mortais
consideram o bem supremo, houve cidadãos que buscavam arruinar a si próprios e
à República com obstinação. 5. De fato, apesar de dois decretos do Senado,
ninguém, de tamanha multidão, revelara a conjuração movido pela recompensa,
nem um único sequer de todos eles\footnote{Construção \emph{ad sensum}: a
expressão `` todos eles'', os conjurados, é sugerida pelo termo
``conjuração''.} abandonara o acampamento de Catilina: tão  grandes eram a
força da doença e, por assim dizer, a peste\footnote{Retomada da ideia de peste
e doença, que Salústio apresentara na digressão moral sobre o passado de Roma
(em X, 6).} que haviam assolado a maioria dos ânimos dos cidadãos.

XXXVII. 1. E não eram apenas os que haviam sido cúmplices da conjuração que
tinham a mente perturbada, mas a plebe toda, de maneira geral, aprovava os
desígnios de Catilina pelo desejo de revolução. 2. De resto, parecia fazê-lo
segundo sua tradição, 3. pois sempre, numa cidade, os que não têm recurso algum
invejam os bons, exaltam os maus; odeiam o antigo, anseiam o novo; por
odiarem sua própria condição, desejam mudar tudo; alimentam-se, irrefletidamente,
de desordem e sedições, uma vez que a pobreza é uma posse que se conserva
facilmente, sem dano. 4. Ora, a plebe urbana, ela, sim, precipitava-se sobre
muitas causas. 5. Antes de tudo, aqueles que em toda parte se sobressaíam
principalmente pela desonestidade e petulância, outros, do mesmo modo, que
haviam perdido o patrimônio em situações desonrosas, por fim, todos os que o
escândalo e o crime haviam levado ao exílio haviam afluído a Roma como a um
cavername\footnote{Vale dizer, como a um esgoto. A \emph{sentina}, traduzida
aqui por ``cavername'', era o porão do navio e, por metonímia, designava os
dejetos e imundícies ali acumulados.}. 6. De resto, muitos, lembrados da
vitória de Sula, por perceberem que alguns dos soldados rasos eram senadores,
outros eram tão ricos que passavam a vida em fausto e padrão de reis, cada um
deles tinha a esperança, caso pegasse em armas, de obter, com a vitória, coisas
tais para si. 7. Além disso, a juventude, que suportara a falta de recursos nos
campos com os ganhos das próprias mãos, preferira, movida pelas distribuições
privadas e públicas, o ócio urbano a um trabalho ingrato. Eles e os mais todos
alimentavam-se do mal público. 8. Daí não ser muito de admirar que homens
pobres, de mau caráter, enorme esperança, cuidassem da República e de si mesmos
igualmente. 9. Demais, aqueles que, com a vitória de Sula, tiveram seus
parentes proscritos, seus bens confiscados, seu direito de liberdade diminuído,
aguardavam o resultado da guerra com ânimo nada diferente. 10. Além disso,
todos aqueles que eram de facções diferentes da do Senado preferiam a agitação
da República a ter menor poder. 11. Foi assim que esse mal voltou-se contra a
cidade pós tantos anos.

XXXVIII. 1. De fato, depois que se restituiu o poder dos tribunos, no consulado
de Gneu Pompeu e Marco Crasso\footnote{Em 70. Os tribunos haviam perdido boa
parte de seu poder com a legislação de Sula, em 81.}, homens jovens, obtendo
enorme poder quando a idade e o ânimo eram inda impetuosos, passaram a agitar a
plebe com acusações contra o Senado, depois, a incendiá-los ainda mais com
distribuições e promessas, tornando-se eles próprios, assim, ilustres e
poderosos. 2. Contra eles empenhava-se com máximo afinco a maior parte dos
senadores em defesa da própria grandeza, na aparência, a do Senado. 3. De fato,
para dizer a verdade em poucas, quantos, a partir dali, perturbaram a República
sob pretextos honrosos, uns, como a defender os direitos do povo, parte, para
que a autoridade do Senado fosse a maior possível, lutavam, individualmente, por seu
próprio poder, simulando fazê-lo pelo bem público. 4. Não tinham moderação ou medida em
seus esforços; ambos exerciam a vitória com crueldade. 

XXXIX. 1. No entanto, depois que Gneu Pompeu foi enviado à guerra nos
mares\footnote{Para combater os piratas do Mediterrâneo, em 67.} e contra
Mitridates\footnote{Na já mencionada guerra contra o rei Mitridates do Ponto,
em 66 (cf. XVI, 5).}, diminuíram os recursos da plebe, cresceu o poder de uns
poucos. 2. Eles detêm as magistraturas, as províncias e tudo mais; seguros, no
ápice do poder, passam a vida sem medo e aterram com processos os demais, a fim
de manter bem calma a plebe em sua magistratura. 3. Porém, naquela situação
incerta, tão logo se lhes ofereceu uma esperança de revolução, a antiga rixa
deu-lhes coragem. 4. É que, se no primeiro combate Catilina tivesse saído vencedor ou
em igualdade, sem dúvida grande ruína e calamidade se teriam abatido sobre a
República, e não teria sido possível aos vencedores gozar a vitória por muito tempo
sem que alguém, que pudesse mais que os débeis e fracos, tomasse o poder e a
liberdade. 5. Houve, no entanto, conjuração à parte, diversos homens que de
início debandaram para o lado de Catilina. Entre eles estava Fúlvio, filho de
um senador, a quem seu pai mandara matar depois de o interceptar em seu
trajeto. 6. Na mesma época, em Roma, Lêntulo, tal como Catilina ordenara,
incitava, fosse por si mesmo, fosse por meio de outros, a todos aqueles que,
por caráter ou fortuna, cria aptos à revolução --- e não apenas cidadãos, mas
qualquer tipo de homens, contanto que fossem úteis na guerra\footnote{Ou seja,
Lêntulo pretendia fazer uso mesmo de escravos em seu contingente. A postura de
Catilina será diferente, como se lê em LVI, 5.}. 

XL. 1. Então encarrega certo Públio Umbreno de procurar os embaixadores dos
alóbroges\footnote{Tribo celta da Gália Narbonense.} e, se possível,
convencê-los a uma aliança na guerra, julgando que estavam esmagados por
dívidas em âmbito público e privado e, demais, por ser o povo gaulês guerreiro
por natureza, que fácil os poderia persuadir a tal desígnio. 2. Umbreno, por já
haver feito negócios na Gália, era conhecido entre os principais chefes de suas
cidades e os conhecia. Assim, sem demora, tão logo notou os embaixadores no
fórum, fez umas breves perguntas acerca da situação da cidade e, como que se
doendo de sua desventura, começou a inquirir-lhes que solução esperavam para
tamanhos males. 3. Depois de vê-los a queixar-se da ganância dos magistrados, a
acusar o Senado de não lhes oferecer auxílio algum, a esperar a morte como
remédio de seus males, diz-lhes: --- Ora, eu vos mostrarei, contanto que
desejeis ser homens de verdade, um meio de escapar desses terríveis males. 4.
Assim que o disse, os alóbroges, induzidos a grande esperança, rogam a Umbreno
que se compadeça deles: não havia nada tão árduo ou difícil que não fariam com
toda a  vontade, contanto que isso libertasse sua cidade das dívidas. 5. Ele os
leva à casa de Décimo Bruto, porque era próxima ao fórum e estava a par do
plano em virtude de Semprônia\footnote{O sujeito implícito de ``estava a par''\
continua sendo ``a casa de Décimo Bruto'', referindo-se, metonimicamente, à família que a habitava.}, pois Bruto encontrava-se, então,
fora de Roma. 6. Além disso, manda chamar Gabínio para conferir maior
autoridade à conversa; em sua presença, revela a conjuração, menciona os
aliados, diversos deles inocentes, ademais, de qualquer crime, para que os
enviados tivessem maior coragem; em seguida, depois de prometerem cooperar,
encerra a reunião. 

XLI. 1. Mas longo tempo os alóbroges não souberam ao certo que decisão tomar.
2. De um lado havia as dívidas, o gosto pela guerra, uma grande recompensa na
esperança de vitória; de outro, porém, maiores recursos, decisões seguras,
prêmios certos em lugar de uma esperança incerta. 3. Enquanto revolviam tais
questões, venceu, por fim, a fortuna da República. 4. Assim, tal como o
ouviram, revelam todo o caso a Quinto Fábio Sanga\footnote{Patrono da tribo,
que fora subjugada em 121 a.C. por um antepassado de Sanga, Quinto Fábio
Máximo, por isso mesmo cognominado Alobrógico.}, de cuja proteção a cidade mais
fazia uso. Cícero, conhecendo o plano por Sanga, recomenda aos embaixadores que
simulem um forte interesse pela conjuração, frequentem os demais, façam
muitas promessas, esforcem-se por mantê-los o mais à vista possível.

XLII. 1. Quase na mesma época, havia insurreições na Gália citerior e na
ulterior, bem como no território do Piceno, no de Brútio, na Apúlia. 2. O fato
é que aqueles que Catilina enviara de antemão faziam tudo a um tempo,
irrefletidamente e como que por demência: haviam, com suas reuniões noturnas,
transportes de armas e lanças, com sua pressa e completa agitação, provocado
mais temor do que real perigo. 3. O pretor Quinto Metelo Céler, apoiado num
decreto do Senado emitido depois de uma audiência formal, prendera diversos
desse número, assim como fizera Caio Murena na Gália ulterior, que governava como
lugar-tenente.

XLIII. 1. Já em Roma, Lêntulo, com os demais líderes da conjuração, preparados,
ao que tudo indicava, com grandes tropas, decidira que, quando Catilina
chegasse ao território de Fésulas com seu exército, o tribuno da plebe Lúcio
Béstia convocaria uma assembleia popular, faria queixas acerca das medidas de
Cícero e atribuiria ao excelente cônsul a odiosidade de uma guerra gravíssima;
a esse sinal, na noite seguinte a restante multidão executaria as tarefas
atribuídas a cada um. 2. Ora, dizia-se que elas haviam sido assim divididas:
Estatílio e Gabínio, com um grande contingente, incendiariam doze lugares
estratégicos da cidade, a fim de tornar mais fácil, pelo tumulto, o acesso ao
cônsul e aos demais contra quem tramavam; Cetego bloquearia a entrada da casa
de Cícero e o atacaria; cada um o faria a uma pessoa diferente, mas os filhos
das famílias, a maior parte dos quais pertencia à nobreza, matariam seus pais;
a um tempo, mortos todos com a carnificina e o incêndio, irromperiam em direção
a Catilina. 3. Em meio a tais preparativos e deliberações, Cetego queixava-se a
todo momento da apatia dos aliados: eles teriam, hesitando e adiando, grandes
oportunidades esperdiçado; era preciso agir, não deliberar, em meio a perigo
tal e, se uns poucos o ajudassem, mesmo que os demais nada fizessem, atacaria a
Cúria. 4. De natureza feroz, ardente, de coragem manifesta, via na rapidez o
maior bem.

XLIV. 1. Ora, os alóbroges, seguindo as recomendações de Cícero, reúnem os
demais na casa de Gabínio. Exigem de Lêntulo, Cetego, Estatílio, bem como de
Cássio, um juramento para levarem, selado, a seus concidadãos: de outro modo,
não será fácil conseguir persuadi-los a tamanha empresa. 2. Os demais, nada
suspeitando, o concedem; Cássio promete em breve ir para lá e deixa a cidade
pouco antes dos embaixadores. 3. Lêntulo envia com eles certo Tito Voltúrcio,
de Crotona, a fim de que os alóbroges, antes de chegar a sua terra,
confirmassem sua aliança com Catilina com oferecimento mútuo de lealdade. 4.
Ele próprio entrega a Voltúrcio uma carta para Catilina, cuja cópia é
transcrita abaixo: 5. “Ficarás sabendo quem sou por aquele que enviei a ti.
Pensa em que calamidade te encontras e lembra-te de que és um homem. Considera
o que teus interesses demandam; pede auxílio a todos, mesmo dos mais
insignificantes.” 6. Além disso, dá ordens oralmente: uma vez considerado
inimigo público pelo Senado, qual será o sentido de rejeitar a escravaria? Na
cidade, está pronto o que ordenara; que não tarde a se aproximar.

XLV. 1. Tomadas tais medidas e decidida a noite em que partiriam, Cícero,
informado de tudo pelos embaixadores, ordena aos pretores Lúcio Valério Flaco e
Caio Pontino que prendam a comitiva dos alóbroges com uma emboscada na ponte
Múlvia. Revela inteiramente o motivo de estarem sendo enviados; permite que
tomem as demais medidas segundo a necessidade. 2. Eles, homens com experiência
no exército, dispondo as guardas sem alvoroço, bloqueiam a ponte secretamente,
tal como ordenado. 3. Depois que os embaixadores chegaram àquele local com
Voltúrcio e a um tempo fez-se um clamor de ambas as partes, os gauleses, logo
conhecendo o plano, entregam-se aos pretores sem demora. 4. Voltúrcio, de
início, exorta os demais e defende-se da multidão com o gládio; em seguida,
quando foi abandonado pelos embaixadores, fazendo a Pontino várias súplicas por
sua salvação, porque era seu conhecido, por fim, temeroso e desesperando de sua
vida, rende-se aos pretores como a inimigos.

XLVI. 1. Isso feito, tudo é rapidamente relatado ao cônsul por meio de
mensageiros. Porém, grande aflição e alegria o tomaram a um tempo. De fato,
alegrava-se ao perceber que, revelada a conjuração, a cidade estava livre de
perigo; mas, por outro lado, estava inquieto, ponderando o que deveriam fazer a
cidadãos tão importantes, presos por um crime tão terrível; acreditava que era
seu dever puni-los, que a impunidade causaria a ruína da República. 3. Então,
recobrando a confiança, ordena que se chamem a sua presença Lêntulo,
Cetego, Estatílio, Gabínio, bem como Cepário de Terracina, que se preparava
para partir para a Apúlia a fim de inflamar a escravaria. 4. Os demais chegam
sem demora; Cepário, que partira de casa pouco antes, fugira da cidade ao tomar
conhecimento da denúncia. O cônsul em pessoa conduziu Lêntulo ao Senado
segurando-o pela mão, por tratar-se de um pretor; ordena aos restantes que
acompanhem os guardas até o Templo da Concórdia. 5. Convoca o Senado para lá e,
com uma grande afluência dessa ordem, apresenta Voltúrcio juntamente com os
embaixadores; ordena ao pretor Flaco que leve para o mesmo local a caixa com a
carta que recebera dos embaixadores.

XLVII. 1. Voltúrcio, interrogado sobre sua viagem, sua carta, por fim, quais
eram seus planos ou motivações, de início inventa fatos outros, dissimula a
respeito da conjuração; depois, quando se ordenou que falasse com garantia
pública de imunidade, revela como tudo havia sido feito, apontando que,
cooptado havia poucos dias por Gabínio e Cepário, nada sabia mais do que os
embaixadores, apenas ouvira de Gabínio mais de uma vez que Públio Autrônio,
Sérvio Sula, Lúcio Vargunteio, além de muitos outros, estavam naquela
conjuração. 2. Os gauleses admitem o mesmo e acusam Lêntulo, que dissimulava,
pela carta e pelas conversas que costumava ter: segundo os Livros
Sibilinos\footnote{Livros de teor profético atribuídos à Sibila, de  uso em
rituais expiatórios por parte dos \emph{quindecimviri sacris faciundis},
``quindecínviros encarregados dos rituais sagrados''.}, estaria previsto o
reinado de Roma a três Cornélios; Cina e Sula antes\footnote{Lúcio Cornélio
Cina, que exercera o consulado de 87 a 84, e Lúcio Cornélio Sula, que fora
ditador de 82 a 79.}, ele seria o terceiro a quem estava destinado apoderar-se
da cidade; de resto, aquele era o vigésimo ano desde o incêndio do Capitólio, o
qual, vaticinavam os adivinhos de acordo com os prodígios, seria um ano
sangrento em virtude da guerra civil. 3. Então, lida sua carta, depois de todos
reconhecerem seu selo, o Senado decide que Lêntulo abdique de sua magistratura
e que, do mesmo modo, os demais sejam mantidos sob livre custódia. 4. Assim,
Lêntulo é confiado a Públio Lêntulo Espínter, que então era edil, Cetego, a
Quinto Cornifício, Estatílio, a Caio César, Gabínio, a Marco Crasso, Cepário
--- pois este fora capturado pouco antes, enquanto fugia ---, ao senador Gneu
Terêncio.

XLVIII. 1. Enquanto isso, revelada a conjuração, a plebe, que, sedenta de
revolução num primeiro momento, era totalmente favorável à guerra, muda de
opinião, execra os planos de Catilina, exalta Cícero aos céus: como se livre da
escravidão, era tomada de contentamento e alegria. 2. De fato, julgava que os
demais crimes de guerra lhe trariam mais proveito do que prejuízo, mas
considerava que o incêndio seria cruel, desmedido e extremamente calamitoso a
si mesma, visto que todos os seus recursos diziam respeito às necessidades
cotidianas e aos cuidados do corpo. 3. No dia seguinte, fora levado ao Senado
certo Lúcio Tarquínio, que, diziam, teria sido capturado em viagem, quando
partia em direção a Catilina. 4. Ele, prometendo fazer revelações acerca da
conjuração se lhe fosse concedida a garantia pública de imunidade, recebe do
cônsul a ordem de dizer o que sabe e informa o Senado praticamente do mesmo que
Voltúrcio acerca dos incêndios que se aprestavam, do massacre dos bons
cidadãos, da marcha dos inimigos; além disso, que fora enviado por Marco Crasso
para anunciar a Catilina que não o atemorizasse a captura de Lêntulo, Cetego e
os demais da conjuração, mas que, por isso mesmo, se apressasse ainda mais por
chegar à cidade, a fim de elevar novamente o moral dos demais e mais fácil
tirar os primeiros de perigo. 5. Porém, quando Tarquínio mencionou Crasso, um
homem nobre, de enormes riquezas, extrema influência, alguns consideraram o
caso indigno de crédito, parte, embora o julgasse verdadeiro, por lhe parecer
que em tais circunstâncias cumpria antes abrandar que provocar o enorme poder
daquele homem, a maioria, devedora a Crasso por negócios particulares,
conclamam que a denúncia é falsa e solicitam que se submeta o caso a
deliberação. 6. Assim, consultado por Cícero, o Senado em peso decide que a
denúncia de Tarquínio era julgada falsa, que ele deveria ser mantido sob
custódia, não tendo mais a possibilidade de fazê-la, a não ser que delatasse aquele que
o mandara inventar um fato tão grave. 7. Houve, naquela época, quem julgasse
que aquilo fora maquinado por Públio Autrônio para que, no caso de se acusar
Crasso, ficasse mais fácil, com sua influência, proteger os demais com sua
cumplicidade. 8. Outros afirmavam que Tarquínio fora instigado por Cícero, a
fim de que Crasso não perturbasse a ordem pública ao assumir, como era seu
costume, a defesa de homens desonestos. 9. Eu ouvi o próprio Crasso afirmar em
público, posteriormente, que aquela terrível injúria fora-lhe imputada por
Cícero.

XLIX. 1. Ora, na mesma época, Quinto Cátulo e Caio Pisão não conseguiram
persuadir Cícero, fosse por pedidos, favor ou dinheiro, a que se implicasse
Caio César injustamente por meio dos alóbroges ou de outro delator. 2. De fato,
ambos nutriam por ele uma forte inimizade: Pisão, atacado, durante um processo
de extorsão, pela execução ilegal de um transpadano, Cátulo, abrasado de ódio
por sua candidatura ao pontificado, porque, apesar de sua idade avançada, de
suas grandes honrarias, saíra derrotado pelo jovem César. 3. A situação parecia
oportuna porque este, em virtude de sua célebre prodigalidade no âmbito
privado, de seus enormes encargos no público, devia uma grande soma de
dinheiro. 4. Porém, quando não conseguem persuadir o cônsul a crime tão grave,
cerceando pessoa por pessoa e inventando o que afirmavam ter ouvido de
Voltúrcio e dos alóbroges, suscitam grande ódio contra ele, a tal ponto que
alguns cavaleiros romanos, que por segurança estavam armados ao redor do Templo
da Concórdia, impelidos pela grandeza do perigo ou pela volubilidade de seu caráter, ameaçaram César com o gládio quando deixava o Senado, a fim de que
ficasse mais evidente o seu zelo pela República. 

L. 1. Enquanto isso se passa no Senado e se decidem as recompensas para os
embaixadores dos alóbroges e Tito Voltúrcio, uma vez que se comprovara sua
denúncia, os libertos e alguns dos clientes de Lêntulo, por ruas diversas,
aliciavam, nos vários bairros da cidade, os artesãos e a escravaria a fim de o
libertarem; outra parte procurava os chefes dos bandos, que costumavam agitar a
ordem pública mediante pagamento. 2. Já Cetego, por mensageiros, instava à
audácia seus escravos e libertos seletos e experientes, para que, agrupados e
com armas, forçassem caminho até ele. 3. Quando conhece tais preparativos, o
cônsul dispõe guarnições tal como a ocasião e as circunstâncias pediam, convoca
o Senado e submete a deliberação as medidas que cabia tomar em relação aos que
haviam sido postos sob custódia. Mas pouco antes o Senado em peso julgara que
eles haviam agido contra a República. 4. Então Décimo Silano, o primeiro a quem
se solicitou o parecer, por ser na época cônsul designado, propusera a pena
capital para os que eram mantidos sob custódia e, além disso, para Lúcio
Cássio, Públio Fúrio, Públio Umbreno, Quinto Ânio, caso fossem capturados;
depois, influenciado pelo discurso de Caio César, afirmara que votaria pela
proposta de Tibério Nero, que julgara que se devia aumentar o número de
guarnições antes que o caso fosse submetido a deliberação. 5. Ora, César,
quando chegou sua vez e o cônsul solicitou seu parecer, falou palavras do
seguinte teor:

LI. 1. --- A todos os homens, senhores senadores, que deliberam acerca de
questões incertas, cumpre estarem isentos de ódio, amizade, de ira ou
misericórdia. 2. Não é fácil à mente discernir a verdade quando tais emoções a
impedem, nem pessoa alguma obedece, a um tempo, à paixão e à utilidade. 3.
Quando se emprega a inteligência, ela tem força; se a paixão toma conta, passa
a comandar --- não tem a mente força alguma\footnote{As palavras de César ecoam
as do prefácio de Salústio, tanto pelo vocabulário como pela ênfase no uso da
razão.}. 4. Poderia facilmente rememorar, senhores senadores, as decisões
errôneas que reis e povos tomaram movidos por ira ou misericórdia; prefiro,
porém, mencionar o que nossos antepassados fizeram correta e ordenadamente,
contra a paixão que sentiam. 5. Durante a guerra da Macedônia, que travamos
contra o rei Perses\footnote{De 171 a 168.}, a cidade de Rodes, grandiosa e
magnífica, que crescera com a ajuda do povo romano, mostrou-se desleal e hostil
a nós. Depois da guerra, porém, quando se deliberou acerca dos rodienses,
nossos antepassados, para que não se dissesse que a guerra começara antes por
riquezas que por injúria, deixaram-nos partir impunes. 6. Do mesmo modo,
durante todas as guerras púnicas, embora não raro os cartagineses, tanto na paz
como durante as tréguas, cometessem inúmeros crimes abomináveis, nunca nossos
antepassados cometeram tais atos, ainda que houvesse ocasião: buscavam antes o
que era digno de si próprios que possíveis ações legítimas contra aqueles. 7.
Do mesmo modo, senhores senadores, cumpre a vós cuidar para que o crime de
Públio Lêntulo e dos demais não tenha, junto a vós, mais valor do que vosso
prestígio, e para que não leveis em conta antes a ira que a reputação. 8. De
fato, se encontramos uma pena digna, de acordo com suas ações, aprovo esta
proposta excepcional\footnote{Excepcional porque ilegal, já que, tecnicamente,
apenas o povo tinha o direito de decidir pela pena capital de cidadãos romanos.
Cf. Ramsey (2007) \emph{ad locum} para contextualização e referências.}; porém,
se a magnitude do crime supera toda imaginação, considero que se deve fazer uso
das penas que são previstas pelas leis. 9. A maior parte dos que deram seu
parecer antes de mim compadeceu-se da desgraça da República de maneira
elaborada e magnífica. Enumeraram qual seria a crueldade da guerra, o que
aconteceria aos vencidos: virgens, meninos raptados, filhos arrancados do
abraço dos pais, mães de família sofrendo o que aprouvesse aos vencedores;
templos e casas espoliados; massacres, incêndios; por fim, tudo ficaria repleto
de armas, cadáveres, de sangue e dor\footnote{Depreende-se que se trataria de
discursos de forte apelo às paixões.}. 10. Mas, pelos deuses imortais, a que
visava tal discurso? Tornar-vos hostis à conjuração? É claro que quem não foi
movido por fato tamanho e tão atroz será incendiado por esse
discurso!\footnote{César procura desfazer o efeito emocional causado pelos
discursos dos senadores que o antecederam.} 11. Não é o caso, nem, a qualquer
dos mortais, as injúrias sofridas parecem pequenas: muitos as recebem de modo
mais grave do que é justo. 12. Mas a cada um cabe uma licença diferente,
senhores senadores. Se homens de baixa condição, que passam a vida na
obscuridade, cometem algum delito em virtude da cólera, poucos ficam sabendo:
igualam-se em reputação e fortuna. Aqueles que, dotados de grande poder, passam
a vida numa posição elevada, todos os mortais conhecem seus atos. 13. Assim, na
maior fortuna há a menor licença: cumpre não favorecer ou odiar, muito menos
irar-se. 14. O que entre os outros chama-se cólera, no poder é denominado
soberba e crueldade. 15. De minha parte, senhores senadores, considero que
qualquer suplício é menor do que seus atos. Mas a maioria dos mortais lembra-se
do que vem por último e, tratando-se embora de homens ímpios, esquecida de seu
crime, questiona o castigo, se um pouco mais severo. 16. Sei bem que Décimo
Silano, bravo e corajoso varão, disse o que acaba de dizer por amor à
República, e que, numa crise tão grave, não dá vazão a favores ou inimizades:
conheço o caráter e a moderação desse homem. 17. Sua proposta, contudo, não me
parece cruel --- pois o que pode acontecer de cruel a tais homens? ---, mas
contrária aos interesses de nossa República. 18. Pois foi claramente o medo ou
a injúria que fizeram que tu, Silano,  cônsul designado, propusesses um tipo de
castigo excepcional. 19. Do temor é desnecessário tratar, sobretudo quando,
pela diligência do nosso ilustríssimo cônsul\footnote{César refere-se a Cícero,
que ficara em Roma para aprestar as defesas da cidade (cf. XXXVI, 3).}, tão
grandes são as nossas defesas em armas. 20. Quanto ao castigo, posso de minha
parte dizer o que se passa na realidade: na dor e nas misérias, a morte é um
descanso dos sofrimentos, não um suplício; ela cura todos os males dos mortais;
do outro lado não há lugar para inquietude ou alegria\footnote{O argumento de
César é de teor epicurista. Ernout (1996: 105, n. 2), aponta sua semelhança com o 
pensamento  de Lucrécio, 3, 830 ss.}. 21. Ora, pelos deuses imortais, por que
razão não acrescentaste a tua proposta que primeiro sofressem castigos
corporais? 22. Será porque a lei Pórcia\footnote{Lei  que protegia os cidadãos
de castigos corporais infligidos por magistrados. Cf. a excelente nota de
Ramsey (2007) \emph{ad locum} para contextualização e referências.} o proíbe?
Mas as outras leis, igualmente, ordenam que não se tire a vida de cidadãos
condenados, mas que se lhes conceda o exílio. 23. Será porque é mais grave ser
açoitado do que morto? Mas o que é cruel ou grave demais contra homens
condenados por tamanho crime? 24. Porém, se é por ser muito brando, de que modo
convém temer a lei numa questão de pouca monta, quando a desprezamos numa de
maior importância? 25. “Mas quem criticará que se tenha tomado uma decisão
contra os parricidas da República?”\footnote{César antecipa e refuta um
contra-argumento.} O tempo, os dias, a fortuna, cujo capricho governa os povos.
Receberão merecidamente o que quer que lhes aconteça; 26. no entanto,
considerai o que vós, senhores senadores, haveis de decidir contra os outros.
27. Todos os maus precedentes nasceram de situações favoráveis. Ora, quando o poder
chega a homens ignorantes ou não tão honestos, aquele precedente inusitado passa de
homens dignos e idôneos a indignos e não idôneos. 28. Os lacedemônios, depois
de vencer os atenienses, impuseram-lhes trinta homens para governar sua
república\footnote{Os chamados ``Trinta Tiranos''.}. 29. De início eles
começaram a matar sem julgamento os mais perversos e odiados por todos --- o povo
se alegra com isso e afirma ter acontecido merecidamente. 30. Depois, conforme
a licença foi aos poucos crescendo, assassinam igualmente os bons e os maus por
capricho, aterrorizam os demais pelo medo. 31. Assim, a cidade, esmagada pela
escravidão, sofreu graves castigos por sua estúpida alegria. 32. Em nossa
época, quando Sula, vencedor, mandou degolar Damasipo\footnote{Pretor em 82,
marianista, assassino de diversos simpatizantes de Sula, foi executado por este
depois de ser capturado na batalha de Porta Colina, em novembro daquele ano.} e
outros da mesma estirpe que haviam prosperado com a desgraça da República, quem
havia que não louvasse seu ato? Afirmavam que homens criminosos e violentos,
que haviam perturbado a ordem pública com sedições, haviam sido mortos
merecidamente. 33. Porém, esse fato foi o início de um grande massacre. Com
efeito, tão logo se desejava uma casa ou vila, enfim, o vaso ou a veste de
alguém, esforçava-se para que tal pessoa fosse incluída entre os
proscritos\footnote{Esse é o pano de fundo da \emph{Defesa de Sexto Róscio de
Améria}, realizada por Cícero por volta de 81, em que o Arpinate atribui a certo
Crisógono, um liberto de Sula, a atitude descrita aqui por César: a inserção
criminosa do pai de Sexto Róscio na lista de proscritos, com vistas à
apropriação indébita de seus bens.}. 34. Assim, aqueles a quem a morte de
Damasipo fora fonte de alegria eram, pouco tempo depois, eles mesmos
arrastados, e a degola só teve fim depois que Sula cobriu todos os seus de
riquezas. 35. E, contudo, não temo tais coisas da parte de  Cícero ou nesta
época; mas, numa grande cidade, muitas e variadas são as índoles. 36. É
possível que em outra ocasião, sob outro cônsul, que tenha, do mesmo modo, um
exército sob seu poder, algo falso seja tomado por verdade. Quando, com
este precedente, um cônsul desembainhar o gládio por decisão do Senado, quem
lhe estabelecerá um fim ou quem lhe porá medida?\footnote{Ramsey (2007)
\emph{ad locum} vê nesta observação uma alusão de Salústio à tomada de poder de
Otaviano, em agosto de 43. Se tal é o caso, o efeito de ironia é duplo, por se
tratar do filho adotivo de César --- nascido em 63, junto com o precedente do
Senado! --- e pelo fato de o próprio César ter ele mesmo desembainhado o
gládio, embora não com o aval do Senado, mas justamente contra o Senado. Em
ambos os casos ninguém foi capaz de lhes estabelecer um fim ou pôr medida.} 37.
Nossos antepassados, senhores senadores, jamais careceram de determinação ou
audácia, nem a soberba impedia que imitassem as instituições alheias, contanto
que fossem honestas. 38. Tomaram as armas de defesa e ataque aos samnitas, as
insígnias dos magistrados, em sua maioria, aos etruscos; enfim, aquilo que em
qualquer parte, entre aliados ou inimigos, consideravam idôneo, seguiam-no com
extremo ardor em Roma; preferiam imitar a odiar os bons. 39. Porém, na mesma
época, imitando o costume da Grécia, infligiam castigos corporais a cidadãos,
aplicavam o sumo suplício aos condenados. 40. Depois que a República se
desenvolveu e, em virtude do grande número de cidadãos, reinavam as facções,
passaram a oprimir os inocentes, a acontecer outras coisas tais. Então se
prepararam a lei Pórcia e outras leis, leis\footnote{A repetição enfática é do original.}
pelas quais foi concedido o exílio aos condenados. 41. Considero sobretudo este
um grande motivo, senhores senadores, para não tomarmos uma decisão
excepcional. 42. É evidente que tinham maior valor e sabedoria aqueles que com
parcos recursos criaram um domínio tão grande, do que nós, que mal conseguimos
sustentar tão belas conquistas. 43. Decidiremos, então, libertá-los e aumentar
o exército de Catilina? De forma alguma. Mas sou do seguinte parecer: seu
dinheiro deve ser confiscado, eles próprios devem ser mantidos sob custódia
pelos municípios de maiores recursos, para que ninguém, posteriormente,
consulte o Senado ou debata com o povo a seu respeito; aquele que agir
diferentemente, o Senado julga que agirá contra a República e a segurança
geral.

LII. 1. Depois que César terminou seu discurso, os demais puseram-se a concordar 
verbalmente com um ou outro, de maneira diversa. Porém, Marco Catão, consultado
acerca de sua proposta, proferiu um discurso do seguinte teor: 

2. --- Meu pensamento é bastante diferente, senhores senadores, quando considero a
situação e os nossos riscos e quando examino em meu íntimo as propostas de
alguns. 3. Parecem-me ter feito uma exposição acerca do castigo dos que
prepararam guerra contra a pátria, os pais, os altares e os Lares; a situação,
porém, sugere antes que nos acautelemos deles que deliberemos sobre o que
decidir contra eles. 4. De fato, podemos nos vingar dos demais crimes no
momento em que os cometem; quanto a este, se não impedirmos que ocorra, quando
acontecer, em vão imploraremos por julgamentos: tomada a cidade, nada resta aos
vencidos. 5. Ora, pelos deuses imortais, eu me dirijo a vós, que sempre
tivestes muito mais apreço por vossas casas, vilas, estátuas, quadros do que
pela República\footnote{Tal como acontece no discurso de César, as palavras de
Catão também ecoam as de Salústio, na digressão moral de VI--XIII.}. Se
desejais manter essas coisas que abraçais, seja qual for sua natureza, se
pretendeis dedicar o ócio a vossos prazeres, acordai de uma vez e apoderai-vos
da República! 6. Não se trata de impostos ou de injúrias cometidas por aliados:
a liberdade e nossa vida estão em jogo\footnote{Se o discurso de César primou
pelo aspecto racional, o de Catão primará pelo apelo às paixões.}. 7. Inúmeras
vezes, senhores senadores, fiz longos discursos nesta Ordem\footnote{Ou seja, a Ordem senatorial.}; não raro me
queixei do luxo e da ganância de nossos cidadãos, e por isso tenho muitos
mortais como inimigos. 8. Jamais tendo concedido a complacência de qualquer
delito a mim e a meu ânimo, não perdoava facilmente as faltas à paixão de um
outro\footnote{Catão busca conferir credibilidade a seu discurso fazendo uso de
sua autoridade, assentada na postura moral que tem caracterizado sua vida e sua
carreira.}. 9. Ora, embora tivésseis tais fatos em pouca consideração, a
República estava segura: a opulência suportava a negligência. 10. Agora, porém,
o que está em questão não é se vivemos segundo bons ou maus costumes, nem o tamanho
ou a grandiosidade do domínio do povo romano, mas se estes, como quer que nos
pareçam, caberão a nós ou a nossos inimigos junto conosco\footnote{A principal
paixão que Catão tenta incutir em seus colegas senadores é o medo. Para tal,
procura passar a ideia de premência da resolução imediata da situação. César,
em contrapartida, argumentara racionalmente que não precisava sequer mencionar
a questão do medo, dadas as providências tomadas por Cícero (LI, 19).}. 11.
Aqui alguém me vem falar de brandura e misericórdia?\footnote{A ironia desta
observação não terá escapado aos leitores contemporâneos de Salústio: César,
após a vitória na guerra civil, adotara a política de misericórdia para com os
vencidos. Catão, porém, recusara-se a sujeitar-se ao ditador, preferindo o
suicídio. Aqui temos um caso análogo: a recusa de Catão em aceitar a política
de misericórdia de César.} Já há muito, sem dúvida, perdemos os vocábulos
adequados às coisas: é por se chamar a dilapidação dos bens alheios de
generosidade, a ousadia em más ações, de bravura, que a República está à beira
do abismo. 12. Sejam, concedo, uma vez que é essa a tradição, generosos com as
fortunas dos aliados; tenham misericórdia dos ladrões do erário; mas que eles
não prodigalizem nosso sangue e, enquanto poupam uns poucos criminosos, venham
arruinar a todos os homens honestos. 13. Caio César, há pouco, discursou bem e
com arte, nesta Ordem, acerca da vida e
da morte, julgando falso, creio eu, o que se conta das regiões infernais: que
os maus, seguindo caminho diverso dos bons, deparam com lugares hediondos,
repugnantes, terríveis e temíveis. 14. Assim, considerou que o dinheiro deles deve
ser confiscado, que eles devem ser mantidos sob custódia pelos
municípios, sem dúvida por temor de que, estando em Roma, sejam libertados à
força pelo cúmplices da conjuração ou pela multidão reunida: 15. como se, na
verdade, existissem homens desonestos e criminosos apenas na Urbe e não por
toda a Itália, ou se a audácia não tivesse mais poder quando os recursos para
defesa são menores. 16. Por isso, não há dúvida de que este conselho é vão se
teme algum perigo da parte deles; se, em meio a tamanho temor generalizado,
apenas ele não tem medo, tanto mais é importante que eu tema por mim e por vós\footnote{Catão parece sugerir que a falta de medo de César seria um indício de seu
envolvimento na conjuração.}. 17. Por isso, ao decidirdes a respeito de Públio
Lêntulo e dos demais, tende certeza de que ao mesmo tempo estais tomando uma
decisão a respeito de Catilina e de todos os conjurados. 18. Quanto mais
atentamente o fizerdes, mais debilitada ficará sua coragem; se virem que
fraquejais apenas um pouco, atacarão já, todos, com ferocidade. 19. Não
julgueis ter sido pela armas que nossos antepassados tornaram grande a
República, de pequena que era. 20. Se assim fosse, nós a teríamos muito mais
bela, uma vez que possuímos maior abundância de aliados e cidadãos, bem como de
armas e cavalos, do que eles. 21. Mas outras foram as razões que os tornaram
grandes, que em absoluto não temos: internamente a indústria, no exterior um
justo domínio, ânimo livre para deliberar, não sujeito a delito ou paixão. 22.
Em lugar disso temos o luxo e a ganância, no âmbito público, a pobreza, no
privado, a opulência; louvamos as riquezas, buscamos a inação; não há diferença
alguma entre honestos e desonestos; a ambição tomou posse de todos os prêmios
que cabiam à virtude\footnote{Novo eco das ideias morais do prefácio  e da
digressão moral.}. 23. E não é de admirar: quando cada um de vós assume uma
conduta separadamente, quando em casa sois escravos dos prazeres, aqui, do
dinheiro e dos favores, por isso acontece que se ataque uma República vazia.
Mas deixo isso de lado. 24. Mui nobres cidadãos conjuraram incendiar a pátria;
convocam à guerra o povo gaulês, inimicíssimo do nome romano; o chefe dos
inimigos avança sobre nossas cabeças com seu exército: 25. vós estais ainda
agora indecisos e não sabeis o que fazer a cidadãos capturados dentro da
cidade? 26. Tende compaixão, eis o que penso\footnote{Catão distorce e ironiza
a posição de César.}: homens bastante jovens cometeram um delito por ambição; e
libertai-os armados, inda por cima; 27. que essa vossa brandura e misericórdia
não se transforme em miséria, se pegarem em armas. 28. Evidentemente a questão
em si é espinhosa, mas vós não a temeis. Espinhosa demais, na verdade; mas, por
inércia e fraqueza, hesitais, um esperando o outro, certamente fiados
nos deuses imortais, que não raro salvaram esta República em meio a seus
maiores perigos. 29. Não é com votos ou súplicas mulheris que se granjeia a
ajuda dos deuses: pela vigília, pela ação, pela boa deliberação é que tudo tem
bom êxito. Quando nos entregamos à indolência e à apatia, em vão imploramos aos
deuses --- eles estão irados e hostis. 30. Entre nossos antepassados, Aulo
Mânlio Torquato, durante a guerra da Gália, mandou matar seu filho porque
lutara contra o inimigo desobedecendo a uma ordem, 31. e aquele ilustre jovem
de desmedida bravura foi punido com a morte\footnote{Leia-se, a respeito, o
relato de Tito Lívio, 8, 7, 1. O \emph{praenomen} de Torquato seria Tito, não
Aulo. Ernout (1996: 113, n. 1) bem observa que o episódio mencionado por
Salústio não aconteceu na guerra contra os gauleses, mas contra os latinos.};
32. vós não sabeis o que decidir a respeito de parricidas cruéis ao extremo?
33. Sem dúvida suas vidas pregressas vão contra este crime. Na verdade, poupai
o prestígio de Lêntulo, se ele mesmo alguma vez poupou o pudor, se poupou sua
reputação, se poupou os deuses ou qualquer homem; perdoai a juventude de
Cetego, caso não tenha feito guerra contra a pátria uma segunda vez. 34. O que
direi de Gabínio, Estatílio, Cepário? Se tivessem tido alguma consideração, não
teriam tomado tais decisões a respeito da República. 35. Por fim, senhores
senadores, se, por Hércules, houvesse lugar para erro, facilmente sofreria que
fôsseis corrigidos pelos fatos, uma vez que desprezais as palavras. Mas estamos
cercados por todos os lados; Catilina ameaça nossas gargantas com seu exército;
há outros inimigos dentro das muralhas e no seio da cidade, não é possível
fazer quaisquer preparativos ou deliberações secretamente: um motivo a mais
para ter pressa. 36. Por isso, meu parecer é o seguinte: uma vez que a
República foi posta em enorme perigo pelo desígnio abominável de cidadãos
criminosos e eles foram condenados por denúncia de Tito Voltúrcio e dos
embaixadores alóbroges, confessando haver preparado massacres, incêndios e
outros crimes terríveis e cruéis contra os cidadãos e a pátria, como se pegos
em flagrante delito de crime capital, deve-se aplicar, segundo a tradição
ancestral, a pena de morte.

LIII. 1. Depois que Catão se sentou, todos os consulares e, do mesmo modo,
grande parte do Senado, elogiam sua proposta, exaltam aos céus o valor de seu
ânimo, censuram uns aos outros de covardes. Catão é considerado ilustre e
grandioso; aprova-se o decreto do Senado tal como julgara. 2. Ora, quanto a
mim, que muito li, muito ouvi sobre as façanhas que o povo romano realizou na
paz como na guerra\footnote{Ao apontar o seu amplo uso de fontes orais e
escritas, Salústio reforça sua autoridade de historiador.}, em mar e em terra,
calhei de achar por bem observar que fatores sobretudo contribuíram para que se
sustentassem tamanhas empresas. 3. Sabia que inúmeras vezes lutara com um
pequeno contingente contra grandes legiões de inimigos; tomara conhecimento de
que guerras haviam sido travadas com pequenas tropas contra reis opulentos;
que, além disso, não raro suportara a violência da fortuna; que pela eloquência
os gregos estavam à frente dos romanos, pela glória de guerra, os gauleses. 4.
E para mim, que muito refletia, era evidente que o egrégio valor de uns poucos
cidadãos levara tudo a cabo, e por isso acontecera que a pobreza superasse as
riquezas, que um pequeno número vencesse uma multidão. 5. Porém, depois que a
Urbe foi corrompida pelo luxo e pela indolência, a República, retrocedendo,
passou a sustentar os vícios de comandantes e magistrados com sua grandeza e,
como se estivesse esgotada pelo parto, em muitas ocasiões não houve uma única
pessoa sequer, em Roma, de grande virtude. 6. Ora, segundo me lembro, houve
dois homens de enorme virtude, de caráter diverso, Marco Catão e Caio César.
Uma vez que o tema os apresentara, decidi não deixar que passassem em
silêncio\footnote{Baker (1982), pp. 801--802, chama a atenção para o mesmo uso
ambivalente, ativo e passivo, de \emph{silentio} que observamos no prefácio (I,
1): neste caso, é passivo no que concerne a César e Catão, ativo no que
concerne a Salústio.}, revelando, na medida de meu engenho, a natureza e o
caráter de um e outro\footnote{Na síncrise do capítulo seguinte.}. 

LIV. 1. Então, eram de estirpe\footnote{César pertencia à \emph{gens Iulia},
família patrícia que remetia seus ancestrais a Iulo, filho de Eneias, e, em
última instância, à Vênus, mãe deste; Catão pertencia à \emph{gens Porcia},
família plebeia.}, idade\footnote{César nascera em 100, Catão, em 95.},
eloquência\footnote{Conforme demonstram os discursos que acabam de ser
reportados (embora se trate de reconstruções verossímeis, é claro, escritas
segundo o \emph{modus scribendi} de Salústio.)} quase iguais; semelhante era a
magnanimidade, igualmente a glória, mas em cada um de um tipo diferente. 2.
César era considerado grandioso pelos benefícios e pela generosidade; pela
integridade de sua vida, Catão. Aquele tornou-se célebre pela brandura e
misericórdia, a este a severidade conferira dignidade. 3. César conquistou a
glória distribuindo, ajudando, perdoando, Catão, nada prodigalizando. Num havia
o refúgio dos miseráveis, no outro, a ruína dos maus. Louvavam-se a afabilidade
daquele, a constância deste. 4. Por fim, César decidira-se pelo labor, pela
vigília, pela atenção aos negócios dos amigos em negligência dos seus, pela não
recusa do que merecesse ser concedido; buscava um grande comando, um exército,
uma nova guerra onde seu valor pudesse brilhar. 5. Catão, por outro lado, tinha
um grande desejo de moderação, de glória, mas, sobretudo, de severidade. 6. Não
disputava em riquezas com o rico, nem em violência com o violento, mas em valor
com o bravo, em pudor com o modesto, em desinteresse com o inocente. Preferia
ser a parecer bom; assim, quanto menos buscava a glória, tanto mais ela o
seguia. 

LV. 1. Depois que, como disse, o Senado acatou a proposta de Catão, o cônsul,
considerando que o melhor a fazer era aproveitar a noite que avançava para que
não ocorresse alguma mudança naquele intervalo, manda preparar os triúnviros
que a pena exigia\footnote{Trata-se dos \emph{tresviri capitales}, ``triúnviros
capitais'', responsáveis pela segurança de Roma e pela execução das penas
capitais, como sugere seu próprio título.}; 2. ele em pessoa, depois de dispor
as guardas, conduz Lêntulo ao cárcere; os pretores fazem o mesmo com os demais.
3. Há um lugar no cárcere, quando se sobe à esquerda, que é chamado
Tuliano\footnote{Localizado originalmente no Fórum Romano, hoje o Tuliano
encontra-se embaixo da Igreja de San Giuseppe dei Falegnami, construída no
século XVI.}, rebaixado cerca de doze pés no solo. 4. Protegem-no paredes de
todos os lados e, por cima, uma abóbada ligada por arcos de pedra; mas seu
aspecto é repugnante e terrível em virtude do desmazelo, da escuridão, do odor.
5. Depois que Lêntulo foi enviado a esse local, os algozes dos crimes capitais,
conforme lhes fora ordenado, estrangularam-no. 6. Dessa forma, aquele patrício
da ilustríssima família dos Cornélios, que detivera o poder consular em
Roma\footnote{Lêntulo fora cônsul menos de uma década antes da conjuração, em
71, juntamente com Cneu Aufídio Orestes.}, encontrou, para sua vida, uma ruína
digna de seu caráter e seus feitos. A Cetego, Estatílio, Gabínio, Cepário,
aplicou-se igualmente o castigo.

LVI. 1. Enquanto isso ocorre em Roma, Catilina forma duas legiões com o
contingente que  levara em pessoa e que Mânlio obtivera; preenche as coortes de
acordo com o número de soldados\footnote{No final da República, cada legião era
composta de 4800 homens, divididos em 10 coortes, compostas, cada uma, de 6
centúrias de 80 homens.}. 2. Em seguida, conforme cada um, voluntário ou do
número dos aliados, chegara ao acampamento, distribuíra-os uniformemente e logo
preenchera as legiões com o número de homens, embora de início não tivesse mais
que dois mil. 3. Ora, de todo o contingente, cerca da quarta parte estava
preparada com armas regulares do exército; os demais, segundo o acaso armara
cada um, portavam venábulos ou lanças, outros, dardos pontiagudos. 4. Mas
depois que Antônio começou a se aproximar com seu exército, Catilina marcha
pelas montanhas; desloca seu acampamento ora na direção da Urbe, ora na da
Gália; não concede aos inimigos a oportunidade de lutarem: tinha a esperança de
logo dispor de grandes tropas, se em Roma os aliados levassem os planos a cabo.
5. Entrementes, repudiava a escravaria (da qual grandes multidões, de início,
acorriam a ele), confiante nos recursos da conjuração, julgando, ao mesmo
tempo, que pareceria estranho a seus interesses misturar a causa de cidadãos a
escravos fugitivos\footnote{Ou seja, fazer uso de escravos fugitivos tiraria de
Catilina a pretensa legitimidade de sua causa, revelando suas intenções
tirânicas.}.
 
LVII. 1. Mas depois que um mensageiro chegou ao acampamento anunciando que em
Roma a conjuração fora descoberta, que a pena de morte fora aplicada a Lêntulo,
Cetego e aos demais que  citei acima, a maior parte, que havia sido atraída para
%acrescei 'a' 'aos demais' porque parece que os 'demais citados acima' também foram mortos, mas 'os demais' pode estar fazendo referência aos que se dispersaram - o que não parece ser o caso pela construção da frase.
a guerra pela esperança de pilhagens ou pelo desejo de revolução, dispersa"-se;
Catilina conduz os demais para o território de Pistório\footnote{Atual
Pistoia.} em grandes marchas por montanhas escarpadas, com a finalidade de
fugir secretamente, por atalhos, para a Gália Transalpina. 2. Mas Quinto Metelo
Céler defendia o território do Piceno com três legiões, por julgar que, dada a
dificuldade da situação, Catilina faria exatamente aquilo que mencionamos
acima. 3. Então, quando soube de sua marcha por desertores, levantou
rapidamente acampamento e tomou posição nos próprios sopés das montanhas, por
onde  Catilina descia apressadamente para a Gália. 4. E, contudo, Antônio,
desobstruído, dada a planura do terreno, não estava muito distante, visto que o
seguia, durante sua fuga, com um grande exército. 5. Ora, Catilina, tão logo percebe estar
cercado pelas montanhas e pelas tropas de inimigos, a situação desfavorável em
Roma, nenhuma esperança de fuga ou defesa, considerando que o melhor a fazer em
tal situação era tentar a fortuna na guerra, decidiu combater com Antônio o
quanto antes. 6. Assim, convocando uma assembleia, proferiu um discurso do
seguinte teor:

LVIII. 1. --- Estou convencido, soldados, de que palavras não aumentam a
bravura, e de que o discurso de um comandante não faz um exército passar de
indolente a incansável, ou de covarde a corajoso. 2. Quanta audácia há no ânimo
de cada um por natureza e caráter costuma ficar evidente na guerra. Aquele a
quem nem a glória, nem os riscos inflamam, em vão se exortaria; o temor em seu
ânimo obstrui seus ouvidos. 3. Mas eu vos convoquei para vos advertir
brevemente e, ao mesmo tempo, revelar a motivação de meus desígnios. 4. Sem
dúvida sabeis, soldados, quanta ruína a apatia e a falta de coragem de Lêntulo
trouxeram a ele próprio e a nós, e de que modo, enquanto aguardo defesas da
cidade, não pude partir para a Gália. 5. Agora, na verdade, todos sabeis, junto
comigo, em que pé está a nossa situação. 6. Dois exércitos de inimigos, um da
cidade, outro da Gália, nos bloqueiam. A carência de grãos e demais suprimentos
nos impede de ficar mais tempo neste local, por mais que nosso ânimo o
suporte\footnote{Cf. V, 3.}. 7. Para onde quer que se decida ir, o caminho terá
de ser aberto a ferro. 8. Por isso vos aconselho que sejais bravos e dispostos
e, ao entrar em combate, que vos lembreis que portais as riquezas, a honra, a
glória, bem como a liberdade e a pátria, em vossas destras. 9. Se vencemos,
tudo estará seguro para nós; as vias em grande número, os municípios e as
colônias se abrirão. 10. Se cedermos ao medo, essas mesmas coisas nos serão
adversas, nem lugar ou amigo algum protegerá a quem as armas não protegerem.
11. De resto, soldados, não pesa a mesma necessidade sobre nós e eles: nós
lutamos pela pátria, pela liberdade, pela vida; para eles, é inútil lutar pelo
poder de uma minoria. 12. Por isso, avançai com bastante audácia, lembrados de
vosso antigo valor. 13. Era-vos possível passar a vida com extrema torpeza no
exílio; alguns de vós podíeis, em Roma, livrando-se dos bons cidadãos, esperar
os recursos alheios. 14. Por tais ações parecerem terríveis e intoleráveis a
homens de verdade, decidistes seguir este caminho. 15. Se quereis deixá-lo para
trás, é preciso audácia; ninguém, senão o vencedor, troca a guerra pela paz.
16. De fato, ter esperança de salvação na fuga, depois de se afastarem dos
inimigos as armas com que se protege o corpo --- isso é realmente
demência\footnote{Como se depreende de LXI, 3, Catilina conseguiu persuadir os
soldados a não fugirem.}. 17. Sempre, em combate, enfrentam maior perigo os que
mais temem; a audácia é considerada uma muralha. 18. Quando vos observo,
soldados, e quando considero vossos feitos, toma-me uma grande esperança de
vitória. 19. Vosso ânimo, vossa idade, vosso valor me encorajam, para não falar
da necessidade, que mesmo os covardes torna corajosos. 20. De fato, a
estreiteza deste lugar impede que um grande número de inimigos nos possa
cercar. 21. Por isso, se a Fortuna invejar vosso valor, cuidai para não
perderdes a vida sem vingança, e para que não sejais trucidados como gado,
depois de capturados, antes que, combatendo à maneira de homens de verdade,
deixeis aos inimigos uma vitória sangrenta e dolorosa.  LIX. 1. Assim que disse
tais palavras, depois de aguardar por um breve momento, ordena que se deem os
sinais e conduz as fileiras ordenadas a um terreno plano. Então, afastando
todos os cavalos, a fim de que os soldados, pelo perigo igualado, tivessem
maior coragem, ele próprio, a pé, ordena o exército de acordo com o terreno e
as tropas. 2. De fato, como havia uma planície entre os montes à esquerda e um
rochedo escarpado à direita, posta oito coortes na linha de frente, dispõe em
formação mais cerrada os estandartes das demais como reserva. Leva os seus
centuriões, todos eles seletos e veteranos, além dos mais bem armados soldados
rasos, à primeira fileira. Ordena a Caio Mânlio que se encarregue do flanco
direito, a um fesulano, do esquerdo: ele mesmo, junto com os libertos e
colonos, posiciona-se diante do estandarte que, segundo se dizia, Caio Mário
mantinha em seu exército na guerra contra os cimbros. 4. Do outro lado, no
entanto, Caio Antônio, sofrendo de gota, impossibilitado de participar do
combate, confia seu exército ao lugar-tenente Marco Petreio\footnote{Como bem
observam os comentadores, segundo o relato do historiador Dião Cássio (37, 39,
4), a gota teria sido usada como pretexto para que Antônio não tivesse de
enfrentar Catilina, seu antigo aliado, que poderia recriminá-lo e revelar sua
participação na conjuração. Cf. XXI, 3.}. 5. Este posiciona na linha de frente
as coortes veteranas que recrutara em virtude da insurreição; depois delas, o
restante do exército, como reserva. Ele em pessoa, flanqueando-os a cavalo,
chama cada um pelo nome, encoraja-os, roga que se lembrem de que lutam pela pátria, pelos filhos, pelos altares e Lares contra
bandidos desarmados. 6. Homem
com experiência militar, porque estivera, com grande glória, por mais de trinta
anos no exército, como tribuno, comandante, lugar-tenente ou pretor, conhecia  pessoalmente a
maioria de seus homens e seus bravos feitos; rememorando-os,
inflamava os ânimos dos soldados.  LX 1. Ora, quando Petreio, depois de passar
tudo em revista, dá o sinal com a trombeta, ordena às coortes que avancem
lentamente; o mesmo faz o exército dos inimigos. 2. Assim que se chegou ao
ponto de onde a batalha poderia ser iniciada pelos soldados de armadura
ligeira, com enorme clamor, travam combate com as insígnias voltadas contra o
inimigo; renunciam aos pilos, luta-se com gládios\footnote{Ou seja, deixam-se
de lado as armas de arremesso, para o enfrentamento corpo-a-corpo.}. 3. Os
veteranos, lembrados de seu antigo valor, atacam brutalmente corpo-a-corpo;
aqueles opõem resistência sem temor -- combate-se com força máxima. 4.
Enquanto isso, Catilina encontra-se na primeira fileira com as forças ligeiras,
socorre os que estão em perigo, busca substitutos para os feridos, tudo provê,
muito combate ele próprio, não raro fere o inimigo; executava a um só tempo os
deveres de um soldado valeroso e de um bom comandante\footnote{Catilina parece
dotado de todos os aspectos da \emph{virtus} (bravura, coragem, valentia,
valor), exceto o aspecto moral. Cf. V, 1.}. 5. Petreio, ao perceber que
Catilina, ao contrário do que imaginara, combatia com grande força, conduz a
coorte pretoriana\footnote{A coorte pretoriana, no final da República, era a
guarda pessoal dos comandantes.} para o meio dos inimigos e mata-os em sua
desordem e resistência isolada; em seguida, ataca os demais por ambos os
flancos. 6. Mânlio e o fesulano, lutando nas primeiras linhas, tombam. 7.
Catilina, tão logo percebe suas tropas dispersadas e que sobrevive com poucos,
lembrado de sua estirpe e de sua antiga dignidade, lança-se contra o ponto mais
cerrado de inimigos e ali, combatendo, é traspassado. 

LXI. 1. Ora, só depois de encerrado o combate era possível perceber quanta
audácia e quanta bravura havia no exército de Catilina. 2. De fato, quase todas
as posições que cada um tomara lutando, quando vivo, estavam cobertas por seus
corpos depois de perderem a vida. 3. Uns poucos, que a coorte pretoriana, em
seu meio, dispersara, estavam um pouco mais distantes, mas todos haviam tombado
com ferimentos sofridos face à face. 4. Catilina, porém, foi encontrado longe
dos seus, em meio aos cadáveres dos inimigos, respirando ainda um pouco e
retendo em seu semblante a ferocidade que tivera em vida. 5. Por fim, de todo o
contingente, nenhum cidadão de nascimento livre foi capturado, fosse em combate
ou em fuga: 6. dessa forma, todos haviam igualmente poupado suas vidas e as dos
inimigos. 7. E, no entanto, o exército do povo romano não conquistara uma
vitória alegre ou sem sangue: os mais bravos haviam tombado em combate ou o
haviam deixado gravemente feridos. 8. Muitos homens, por sua vez, que haviam
deixado o acampamento para ver ou pilhar, encontravam, ao revolver os cadáveres
dos inimigos, alguns, um amigo, parte, um hóspede ou parente; houve também quem
reconhecesse seus inimigos. 9. Assim, diversamente, por todo o exército era-se
tomado de alegria, tristeza, de dor e felicidade. 

