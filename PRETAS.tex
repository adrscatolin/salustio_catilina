\textbf{Gaio Salústio Crispo} (86 a.C.--34 a.C.) foi um historiador e poeta
latino, nascido em Amiterno, na Sabina. Proveniente de família plebeia, mas com
posses, recebeu formação e cedo foi enviado para Roma, onde iniciou sua
carreira política. Com o apoio de Júlio César, foi eleito \emph{questor}, cargo que
lhe assegurou uma cadeira no senado romano, investindo contra adversários de
César, como Milão e Cícero. Ao fim da vida, dedica-se a escrever uma história da
República romana.


\textbf{A conjuração de Catilina} é considerada uma das primeiras obras de
história produzidas na Roma Antiga. O
texto relata o conjunto de eventos que constituíram o malogrado plano de Lúcio
Sérgio Catilina (108 a.C.--62 a.C.) para se assenhorear do poder em Roma, no
ano do consulado de Cícero, em 63 a.C. Catilina, de família patrícia romana,
provavelmente pretor em 68 a.C., governador da província da África no intervalo
de 67 a 66 a.C., tentara por duas vezes eleger-se cônsul, tendo sido derrotado
em ambas ocasiões. Depois do segundo insucesso, urde uma trama elaborada
para tomar o governo, que fracassa e termina com seu assassinato em uma batalha
de forças aliadas contra tropas republicanas. 


\textbf{Adriano Scatolin} é doutor em Letras Clássicas pela
Universidade de São Paulo (2009), com pós-doutorado pela Universidade Paris
\versal{IV} Sorbonne (2012--2013). Atualmente é professor assistente da Universidade de
São Paulo.

