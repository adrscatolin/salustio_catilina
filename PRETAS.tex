\textbf{Caio Salústio Crispo} (86 a.C. -- 34 a.C.) foi um historiador e poeta latino, nascido em Amiterno, na Sabina. Proveniente de família plebeia, mas com posses, teve uma formação requintada e foi cedo para Roma, onde iniciou sua carreira política. Com o apoio de Júlio César, foi eleito questor, cargo que lhe assegurou uma cadeira no senado romano, onde investiu contra adversários de César, como Milão e Cícero. Aposentado, dedicou-se a escrever uma história da República romana que estivessse à altura de sua memória.


\textbf{A conjuração de Catilina} é considerada uma das primeiras obras de História produzidas na Roma Antiga, frequentemente tida como a maior delas. O texto relata o conjunto de eventos que constituíram o malogrado plano de Lúcio Sérgio Catilina (108 a.C. -- 62 a.C.) para se assenhorear do poder em Roma, no ano do consulado de Cícero, em 63 a.C. Catilina, de família patrícia romana, provavelmente pretor em 68 a.C., governador da província da África no intervalo de 67 a 66 a.C., tentara por duas vezes eleger-se cônsul, tendo sido derrotado em ambas as ocasiões. Depois do segundo insucesso, urde uma trama elaborada para tomar o governo, que fracassa e termina com seu assassinato em uma batalha de forças aliadas contra tropas republicanas. Salústio escreve sobre o tema por ver nele uma metonímia da decadência romana.


\textbf{Adriano Scatolin} possui doutorado em Letras Clássicas pela Universidade de São Paulo (2009) e pós-doutorado pela Universidade Paris IV-Sorbonne (2012-2013). Atualmente é professor assistente da Universidade de São Paulo, atuando na área de Retórica Antiga, particularmente na obra de Marco Túlio Cícero.\par
